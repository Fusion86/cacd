\subsubsection {ln3x3}

Functie: NMOS compoundtransistor, basiscel voor analoge schakelingen

Terminals: (g, d, s, vss, vdd)

Equivalent chip oppervlak: 10

Beschrijving:

De NMOS compoundtransistor ln3x3 bestaat uit 9 NMOS basistransistoren
die in een matrix van 3x3 geschakeld zijn.\\
\\
\\

Circuit: \hspace{0.5\textwidth} Layout:\\


\makebox[0.98\textwidth]{
%figuur ln3x3.eps
{\callpsfig{ln3x3.cir.eps}{width=0.4\textwidth}}
\hfill
%figuur ln3x3.lay.eps
{\callpsfig{ln3x3.lay.eps}{width=0.4\textwidth}}
}

\clearpage

\subsubsection {lp3x3}

Functie: PMOS compoundtransistor, basiscel voor analoge schakelingen

Terminals: (g, d, s, vss, vdd)

Equivalent chip oppervlak: 10

Beschrijving:

De PMOS compoundtransistor lp3x3 bestaat uit 9 PMOS basistransistoren
die in een matrix van 3x3 geschakeld zijn.\\
\\
\\

Circuit: \hspace{0.5\textwidth} Layout:\\


\makebox[0.98\textwidth]{
%figuur lp3x3.eps
{\callpsfig{lp3x3.cir.eps}{width=0.4\textwidth}}
\hfill
%figuur lp3x3.lay.eps
{\callpsfig{lp3x3.lay.eps}{width=0.4\textwidth}}
}
\clearpage

\subsubsection {mir\_nin, mir\_nout}
\index{stroomspiegel!basiscel}
Functie: NMOS spiegel, basiscel voor stroomspiegel

Terminals mir\_nin: (i, g, vss, vdd)

Terminals mir\_nout: (i, g, o, vss, vdd)

Equivalent chip oppervlak: 19

Beschrijving:

De NMOS spiegel is een gecascodeerde stroomspiegel. 
Hij bestaat uit een ingang mir\_nin en een uitgang
mir\_nout.
Beide kunnen in een spiegel meerdere keren gerepeteerd
worden om schaling te verkrijgen.
De gebruikte transistoren zijn zgn. compound transistoren.\\

Circuit:

(a) mir\_nin \hspace{0.2\textwidth} (b) mir\_nout\\

\medskip
%figuur mir_n.eps
\begin{figure}[h]
{\callpsfig{mir_n.eps}{width=.5\textwidth}}
%\caption{}
\end{figure}
\newpage
Layout mir\_nin:
%figuur mir_nin.eps
\begin{figure}[h]
\centerline{\callpsfig{mir_nin.eps}{height=3.2in}}
%\caption{}
\end{figure}

Layout mir\_nout:
%figuur mir_nout.eps
\begin{figure}[h]
\centerline{\callpsfig{mir_nout.eps}{height=3.2in}}
%\caption{}
\end{figure}
\clearpage

\subsubsection {mir\_pin, mir\_pout}
\index{stroomspiegel!basiscel}
Functie: PMOS spiegel, basiscel voor stroomspiegel

Terminals mir\_pin: (i, g, vss, vdd)

Terminals mir\_pout: (i, g, o, vss, vdd)

Equivalent chip oppervlak: 19

Beschrijving:

De PMOS spiegel is een gecascodeerde stroomspiegel. 
Hij bestaat uit een ingang mir\_pin en een uitgang
mir\_pout.
Beide kunnen in een spiegel meerdere keren gerepeteerd
worden om schaling te verkrijgen.
De gebruikte transistoren zijn zgn. compound transistoren.\\

Circuit:

(a) mir\_pin \hspace{0.2\textwidth} (b) mir\_pout\\

\medskip
%figuur mir_p.eps
\begin{figure}[h]
{\callpsfig{mir_p.eps}{width=.5\textwidth}}
%\caption{}
\end{figure}
\newpage
Layout mir\_pin:
%figuur mir_pin.eps
\begin{figure}[h]
\centerline{\callpsfig{mir_pin.eps}{height=3.2in}}
%\caption{}
\end{figure}

Layout mir\_pout:
%figuur mir_pout.eps
\begin{figure}[h]
\centerline{\callpsfig{mir_pout.eps}{height=3.2in}}
%\caption{}
\end{figure}

\clearpage

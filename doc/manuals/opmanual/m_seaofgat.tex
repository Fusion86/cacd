\section{De Sea-of-Gates Chip}
\index{Sea-of-Gates|bold}
\label{SoGchip}
\subsection{Globale beschrijving}

De chip waar de OP practicum-ontwerpen op gemaakt worden
is van het "semi-custom" type. Dit houdt in dat de grootte
van de chip vast staat en dat ook de grootte, ligging en
aantal van de te gebruiken transistoren al bekend is. In feite
is de chip al voor een groot deel vervaardigd, alleen de
metallisatie die nodig is voor de gewenste schakeling, moet
nog worden aangebracht. De gekozen structuur op de reeds
gefabriceerde wafers (plakken silicium met een doorsnede
van 4 inch) is die van de moderne gate-array: de Sea-of-Gates
structuur. Deze structuur leent zich primair voor digitale
schakelingen, maar analoge zijn ook mogelijk.\\
\\
Fabricageproces:

\begin{quote}
Een 1.6 $\mu m$~nwell-CMOS proces (het Philips C3DM proces).
\end{quote}

Grootte van de chip:

\begin{itemize}
\item

10~x~10 mm, (46 exemplaren op een 4 inch plak)
\item
144 aansluitpinnen waarvan de functie programmeerbaar is
door middel van metallisatiepatronen,
\item
200.000 transistoren (de helft is pmos, de andere helft is nmos),
\item
afmeting nmos transistor: 1.6~x~23.2 $\mu m$,
\item
afmeting pmos transistor: 1.6~x~29.6 $\mu m$.
\end{itemize}

De metallisatie van de wafers, het uitzagen en afmonteren van de
chips gebeurt in het DIMES instituut. Voor de metallisatie zijn
4 maskerstappen nodig: 2 metaal- en 2 kontaktgaten maskers.

De grootte van de chip is zodanig gekozen dat meerdere typen 
ontwerpen gemaakt kunnen worden. Ook is het niet bij voorbaat
onmogelijk om een redelijk grote schakeling te realiseren. De
belang\-rijk\-ste voorwaarde is dat DIMES in staat moet zijn de
metallisatie aan te brengen.
Hieronder volgt een lijst met de maskers die DIMES voor de metallisatie
moet maken (en dus door de ontwerper gespecificeerd) en hun functie.
Voor de volledigheid is een opsomming toegevoegd van de maskers die
nodig waren om de sea-of-gates chip te maken.

DIMES maskers:
\index{maskers}
\begin{itemize}
\item
\layer{co} (\layer{con},~\layer{cop},~\layer{cps}) : Definieert posities van kontaktgaten in het
oxide tussen de onderste metaallaag(\layer{in}) en \layer{od} en \layer{ps}
gebieden.
\item
\layer{in} : Definieert de onderste (eerste) metaallaag.
\item
\layer{cos} : Definieert de positie van kontaktgaten in het oxide
tussen eerste en tweede(bovenste) metaallaag(\layer{ins}).
\item
\layer{ins} : Definieert de bovenste (tweede) metaallaag.
\end{itemize}
Overige maskers:
\begin{itemize}
\item
\layer{nw} :  nwell, definieert de gebieden waarin p-transistoren kunnen
worden gedefinieerd.
\item
\layer{od} : Definieert de actieve gebieden: plaats van source, gate en
drain van beide typen transistoren.
\item
\layer{sn} : Definieert die delen van de actieve gebieden die van het 
n-type zijn.
\item
\layer{sp} : Definieert die delen van de actieve gebieden die van het 
p-type zijn.
\item
\layer{ps} : Definieert gates van transistoren.
\end{itemize}
\subsection{De kern van de chip,
vanuit circuit-standpunt}

De chip heeft een sea-of-gates structuur en bestaat eenvoudig 
uit een groot aantal rijen transistoren zoals hieronder getekend.
Er zijn evenveel rijen n- als p-type transistoren (zie figuur~\ref{pnrij})

\begin{figure}[hbt]
\centerline{\callpsfig{pnrij.eps}{width=0.9\textwidth}}
\caption{Schematische voorstelling van een pmos- en een nmos-rij op de sea-of-gates chip
\label{pnrij}}
\end{figure}


Elke rij op de chip telt ongeveer 1000 transistoren. Elk knooppunt
in elke rij is toegankelijk vanuit de eerste metaallaag. De tweede
metaallaag kan gebruikt worden indien kruisingen tussen metaalsporen
nodig zijn. Naast een rij n-transistoren ligt altijd een rij 
p-transistoren.\\
De rijenstructuur lijkt onhandig maar is dit in de praktijk niet.
In een schakeling zijn namelijk altijd rijen transistoren te ontdekken.\\
Als voorbeeld een 3-input nand schakeling (zie figuur~\ref{nand3}).

\begin{figure}
  \makebox[0.98\textwidth]{
     {\callpsfig{nand3.eps}{height=0.4\textheight}}
     \hfill
     {\callpsfig{nand3a.eps}{height=0.4\textheight}}
  }
\caption{Schema van de 3-input nand,
de afbeelding hiervan
op een rijenstructuur en de layout op het sea-of-gates image
\label{nand3}}
\end{figure}

In de structuur is duidelijk dat alle transistoren keurig op een rij 
gelegd kunnen worden en met een eenvoudig verbindingspatroon 
(in metaal) verbonden kunnen worden tot een 3-input nand. De in de
structuur meest rechts getekende transistoren dienen alleen als
scheiding van de schakeling met de rest van de transistoren.
De herhaling in de y-richting van n- en p-rijen transistoren is
als volgt: {\tt ....nppnnppnnppn...} (zie ook figuur~\ref{fig-fishbone}).\\
Dit heeft o.a.\ als reden dat de p-transistoren op deze manier
meer gegroepeerd zijn voor grotere nwell gebieden.\\
De basisrepetitie van 4 rijen ({\tt nppn}) komt op de chip 44 keer voor.\\
In de tabellen~\ref{weerstand} en~\ref{capaciteit} staan enkele elektrische eigenschappen van het cmos proces.
\begin{table}[hbt]
\begin{center}
\begin{tabular}{lc}\\
\hline
masker & weerstand ($\Omega/\fbox{}$)\\
\hline\\
$n^+$ OD & 55\\
$p^+$ OD & 75\\
poly-Si & 25-60\\
\layer{in} & 0.045\\
\layer{ins} & 0.03\\
\hline
\end{tabular}
\caption{Nominale interconnectie weerstand cmos proces}
\label{weerstand} \index{weerstand!procesgegevens}
\end{center}
\end{table}

\begin{table}[hbt]
\begin{center}
\begin{tabular}{lccc}
\hline
& Oppervlakte capaciteit & Rand capaciteit &\\
& (aF$/\mu m^2$) & (aF$/\mu m$) & opmerking\\
\hline\\
$n^+$ OD naar substraat & 190 & 310 & junctie capaciteit\\
$p^+$ OD naar substraat & 450 & 570 & junctie capaciteit\\
poly naar substraat & 49 & 54 \\
metaal1 naar substraat & 25 & 45 \\
metaal 1 naar OD & 49 & 54\\
metaal1 naar poly & 49 & 55\\
metaal2 naar substraat & 13 & 49\\
metaal2 naar OD & 15 & 53\\
metaal2 naar poly & 22 & 62\\
metaal1 naar metaal2 & 43 & 81\\
\hline
\end{tabular}
\caption{Nominale interconnectie capaciteiten cmos proces} \index{capaciteit!procesgegevens}
\label{capaciteit}
\end{center}
\end{table}


\subsection{De kern van de chip,
vanuit een layout-standpunt}

In figuur~\ref{fig-fishbone} is een stukje layout gegeven van de basisstructuur.
Duidelijk zijn de rijen transistoren zichtbaar. De p-transistor
is breder dan de n-transistor. Het verschil in breedte is gelijk
aan de ruimte die nodig is voor een extra metaalspoor en kontaktgat  
op de actieve gebieden van de p-transistoren. Tussen de twee rijen 
p-transistoren is plaats voor het voedingsspoor met daaronder een
\layer{od} gebied om de nwell te kunnen verbinden met de hoogste spanning
in de schakeling (de voedingsspanning). Onder het aardspoor ligt
een \layer{od} gebied om het p-substraat aan aarde te kunnen leggen.

\begin{figure}
\centerline{\callpsfig{fishbone2.ps}{height=0.8\textheight}}
\caption{Layout van het fishbone sea-of-gates image. \index{image}
\label{fig-fishbone}}
\end{figure}

Ook valt nog op dat alle afmetingen rechthoekig zijn, behalve
de gate-aansluitingen waarin 45-graden lijnen zijn gebruikt.
Dit is gedaan uit efficiency overwegingen.\\
Wat de layout betreft:

\begin{quote}
De steek in de x-richting is 8.0~$\mu m$.\\
De steek in de y-richting is 197.6~$\mu m$.\\
De layout-eenheid is 1 lambda (1 lambda = 0.2 $\mu m$).
\end{quote}

\subsection{Eisen en afspraken met betrekking tot de metallisatie}
\label{eisen_metal}
\begin{itemize}
\item
Horizontaal lopend over het actieve gebied van een rij
n-transistoren is plaats voor maximaal 4 metaalsporen.\\
Horizontaal lopend over het actieve gebied van een rij
p-transistoren is plaats voor maximaal 5 metaalsporen.
\item
Source- en drain gebieden en ook gate-aansluitingen) kunnen
alleen maar verbonden worden met metaal1 (\layer{in}). 
Verbinding van source, drain en gate 
met metaal2 (\layer{ins}) kunnen alleen maar lopen via metaal1.
\item
Kontakten van metaal1 met source, drain of gate wordt gemaakt 
middels kontaktgaten.
Deze kontaktgaten mogen alleen gedefinieerd worden op de grid-punten
(zie figuur~\ref{fig-fishbone}).
Kontakt van metaal1 (\layer{in}) en pmos source en 
draingebied wordt gemaakt met \layer{cop}, kontakt van metaal1 (\layer{in}) 
en nmos source en draingebied wordt gemaakt met \layer{con} en 
kontakt van metaal1 (\layer{in}) met de gates (poly-silicium) 
wordt gemaakt met \layer{cps}.\\
Overigens mogen deze drie kontaktlagen (\layer{con, cop en cps}) bij het maken 
van de layout door elkaar worden gebruikt omdat bij het maken van het masker
voor deze gaten ook geen onderscheid wordt gemaakt.
\item
Kontakten tussen metaal1 (\layer{in}) en metaal2 (\layer{ins}) worden 
gemaakt met \layer{cos}.
\item
Een \layer{con}- of \layer{cop}- of \layer{cps}-kontakt en een \layer{cos}
kontakt mogen niet boven elkaar worden gedefinieerd.
\item
Op de plaats van een gate kontaktaansluiting mag geen \layer{cos} kontakt
worden gedefinieerd.
\item
Metaal1 (\layer{in}) loopt overwegend horizontaal, metaal2 (\layer{ins}) loopt
overwegend verticaal.
\end{itemize}

\subsection{Meetgegevens van de Sea-of-Gates chip}

\paragraph{Karakteristieken van de nmos en pmos transistoren}

In figuur \ref{npmosidvd} is de gemeten drainstroom $I_d$ uitgezet 
tegen de drain-sourcespanning $V_{ds}$ met de gate-sourcespanning $V_{gs}$ als parameter.\\
In figuur \ref{npmosidvg} is de gemeten drainstroom $I_d$ uitgezet tegen de gatespanning $V_{gs}$ met de drain-sourcespanning $V_{ds}$ als parameter.\\
De maximale stroom voor een enkele nmos transistor ($V_{gs}$ = $V_{ds}$ = 5 V) is ca.~5.5 mA. Voor een enkele pmos transistor is de maximale stroom ($V_{gs}$ = $V_{ds}$= -5 V) gelijk aan ca.~2.7 mA. 
\paragraph{Kleinsignaalmodel van de mos transistor}

In figuur \ref{ks-model}~ is een eenvoudig kleinsignaalmodel gegeven van de
mos transistor. De waarde van de verschillende capaciteiten zijn nog
niet gemeten.\\
Voor de drainstroom in verzadiging geldt ($V_{ds} > V_{gs} > V_t$, zie ook \cite{GS}):\\
\\
\centerline{$I_d = \frac{\beta}{2}(V_{gs}-V_t)^2(1+\lambda V_{ds}$)}\\
\\
De gemeten waarde van de verschillende parameters zijn te vinden in tabel \ref{mostabel}.

\begin{figure}[hbt]
\centerline{\callpsfig{mostm.ps}{width=0.8\textwidth}}
\caption{\label{ks-model} Een eenvoudig kleinsignaalmodel van de mos transistor}
\end{figure}

\begin{table}[hbt]
\begin{center}
\begin{tabular}{ccc}
\hline
Parameter & nmos & pmos\\
\hline\\
$V_t$ & 0.7 V & -1.2 V\\
$\beta$ & 0.65 mA/$V^2$ & 0.13  mA/$V^2$\\
$\lambda$ & 0.027 $V^{-1}$ & 0.096  $V^{-1}$\\
$r_d$ & 7.5 k$\Omega$ & 10  k$\Omega$\\
\hline
\end{tabular}
\caption{\label{mostabel} Enkele gemeten parameters van de nmos en pmos transistoren}
\end{center}
\end{table}
\paragraph{Weerstandswaarden} In tabel \ref{weerstandwaarden} zijn de gemeten waarden te vinden van een aantal weerstanden en vergeleken met de berekende waarden uit gegevens van het cmos-proces.

\begin{table}[hbt]
\begin{center}
\begin{tabular}{ccc}
\hline
Grootheid & Berekende waarde & Gemeten waarde\\
\hline\\
Contactgateweerstand\\
{\bf in-ins (cos)} & max. 0.2~$\Omega$ & 0.13~$\Omega$\\
{\bf in-ps (cps)} & max. 50~$\Omega$ & 22~$\Omega$\\
Metaalweerstand\\
metaal1 ({\bf in}) & 45 m$\Omega/\fbox{}$ & 56 m$\Omega/\fbox{}$\\
metaal2 ({\bf ins})& 30 m$\Omega/\fbox{}$ & 26 m$\Omega/\fbox{}$\\
Polysiliciumgateweerstand\\
nmos gate & 690 $\Omega$ & 700 $\Omega$\\
pmos gate & 790 $\Omega$ & 950 $\Omega$\\
\hline
\end{tabular}
\caption{\label{weerstandwaarden} Enkele weerstandwaarden van de sea-of-gates chip} \index{weerstand!gemeten waarde}
\end{center}
\end{table}

\subsection{Structuur in de rand van de chip}

Er zijn in de rand 144 aansluitposities, verdeeld in 8 groepen van elk 18.
Elke groep van 18 aansluitingen heeft een voedingspositie en een aardpositie.
De andere 16 zijn 
programmeerbaar d.m.v.\ metallisatiepatronen tot een aantal
bufferfuncties, b.v.: digitaal in, digitaal uit, analoog in, analoog uit.
Alle 8 aardposities in de rand zijn met elkaar verbonden (via de 
chiprand).
Voor de programmering van de rand van een chip wordt verwezen
naar een afzonderlijke handleiding. Voor de liefhebbers liggen
er in de ruimten tussen de buffers in de rand nog polyweerstanden.
In elke tussenruimte liggen er 8 van elk 2.5 k$\Omega$ (= 100 vierkanten).
Tenslotte ligt
in de rand linksonder een grote poly-rechthoek die nuttig kan zijn
voor een capaciteit. Deze weerstanden en capaciteiten zijn echter niet 
te gebruiken in het ontwerppracticum omdat ze vanwege de kwarten-indeling
van de chip niet zijn aan te sluiten.\\
\\
\\

\begin{figure}[hbt]
  \makebox[1\textwidth]{
     {\callpsfig{NMOS1-idvd.ps}{width=0.38\textwidth}}
     \hfill
     {\callpsfig{PMOS1-idvd.ps}{width=0.38\textwidth}}
  }
\caption{\label{npmosidvd} De gemeten drainstroom van een nmos en pmos transistor uitgezet tegen de drain-sourcespanning met de gate-sourcespanning als parameter}
\end{figure}

\begin{figure}[hbt]
  \makebox[1\textwidth]{
     {\callpsfig{NMOS1-idvg.ps}{width=0.38\textwidth}}
     \hfill
     {\callpsfig{PMOS1-idvg.ps}{width=0.38\textwidth}}
  }
\caption{\label{npmosidvg} De gemeten drainstroom van een nmos en pmos transistor uitgezet tegen de gate-sourcespanning met de drain-sourcespanning als parameter}
\end{figure}

\cleardoublepage

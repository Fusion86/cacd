\selectlanguage{british}
\title{Seadali: Your Gateway to Sea-of-Gates Layout}
\maketitle
\label{s-seadali}
\index{seadali@\tool{seadali}|(}
\index{editing layout|see{\tool{seadali}}}
\index{creating!layout|see{\tool{seadali}}}
This appendix deals with \tool{seadali}, the interactive tool at the very heart
of the \smc{ocean} layout system.  It is worthwhile to read this appendix
carefully, since it contains many useful hints on how to click the buttons
efficiently.
\tool{Seadali} is an interactive program which allows you to:
\begin{itemize}
\item
View or modify any existing layout cell.
\item
Create an empty layout cell from scratch with the help of \tool{fish}.
\item
Place instances in your layout manually or automatically using \tool{madonna}.
\item
Automatically route your layout using \tool{trout}.
\item
Verify the correctness of your layout against the design rules and the circuit
description.
\end{itemize}
There are many buttons and options to control this program.  But don't worry:
you'll get accustomed to them quite soon.  We try to describe only the most
vital buttons and functions, so you can get started quickly. All trivial or
less useful buttons are not described.

In all examples of this, and the following description we assume the {\sl fishbone} image.
Some mask names, design constrains and commands work slightly
different in the other images. 

\section{Starting \protect\tool{seadali}}
\label{s-start-seadali}
\index{seadali!starting|bold}
The best way to start \tool{seadali} is by typing:
\type{seadali \&}
The '\&' instructs the operating system (Linux) to start the program in the background.
Therefore the prompt returns immediately.
After starting a window pops up\footnote{If starting \tool{seadali} doesn't
work and no window appears, remember that you must be in 
a project directory. You can make a project directory (\smc{nelsis} database) using \tool{mkopr}.
Also, check whether your \smc{display} environment variable is
set to the proper screen (the name of your terminal, followed by ':0').},
which shows the main menu on the left.
From this menu you can select various sub-menus by clicking the
appropriate box.
The bottom-most command of any menu always brings you back
to the main menu. \tool{Seadali} is also sensitive to keyboard input.
Table~\ref{keytable} (page~\pageref{keytable}) shows all commands.

\section{Getting on-line help in \protect\tool{seadali}}
\index{seadali!on-line help}
In the main menu you'll find the button \button{help}. This will start a
hypertext on-line help page on \smc{ocean}.
You need the browser program \tool{mozilla} too for this help facility.
To use another browser, set the \file{BROWSER} environment variable.
Initially it will display the hypertextpage \file{\$ICDPATH/share/lib/seadali/help.html}.

\section{Reading and writing layout: \protect\button{database}}
Pressing \button{database} brings you in a special menu
for reading and writing cells from/into the database. Let's have a look
at the buttons:
\begin{description}
\item [\button{READ}]
Load a cell from the database into the editor. The list of all cells
in your database will be displayed. If this list is empty, your
database has no cells. Notice that it is not possible to read or
edit imported cells from libraries. You can only have a look
at them by loading them as instances in the menu~\button{instances}
(see~\ref{instances} on page~\pageref{instances}).
It is possible to start \tool{seadali} with a cell name as argument.
For instance: \type{seadali adder \&} causes the cell adder
to be read automatically at startup.
\item [\button{Expand all}]
\index{seadali!expanding}
Set the hierarchical level (that is, the amount of detail) you wish to see of
the cell. Initially all son-cells are expanded 'to the bottom'.  Alternatively,
you can just use the keys 1-9 on your keyboard (see table~\ref{keytable} on
page~\pageref{keytable}).  Initially all son-cells are expanded 'to the
bottom'. You can set this default expansion level for all cells in the
\file{.dalirc}-file (see~\ref{dalirc} on page~\pageref{dalirc}).  Initially all
son-cells are expanded 'to the bottom'.
\item [\button{Expand instance}]
Set expansion level of a specific instance.
This can be useful to prevent long drawing times on large designs.
\item[\button{show Sub terminals}]
\index{seadali!sub terminals}
Show the terminals of the sub-cells. The key 's' on your keyboard is doing the
same (see table~\ref{keytable} on page~\pageref{keytable}).
\item[\button{NEW}]
Erase the current cell in \tool{seadali}'s memory to start with a new cell. 
This only clears the memory of \tool{seadali}. It doesn't remove the cell
from the database.
\item[\button{Fish}]
If the design is empty (initially, or after pressing \button{NEW}) this button
creates a $20~\times~1$ array of the fishbone image.  This is useful to start a
new cell. If the design is not empty, this button will purify your layout (see
appendix~\ref{fish} on page~\pageref{fish}).
\item[\button{Fish -i}]
\index{Fish -i button}
This button performs the same as \button{Fish}, except that the image
is removed from the cell. Also, the cell is automatically moved as close as
possible to the origin. This button can be useful if you want to use the cell
as a child cell on a higher level of hierarchy.
\item[\button{WRITE}]
Write the cell into the database. If necessary, you can write it under 
a new name.
\end{description}
The title bar of \tool{seadali}'s window shows the name of the cell
which you are editing now, together with the expansion level.

\attention{You can interrupt \tool{seadali} while it is drawing!}{
Just hit any key to stop the drawing and regain control. This can be especially
useful for extremely large layouts and/or nervous designers. Hit the key
\button{r} or \button{next} (on the keyboard) to redraw the screen, 
in case you want to complete an interrupted drawing. During input activities
(reading, expansion) \tool{seadali} cannot be interrupted.}

\section{Running other programs: the \protect\button{automatic tools} menu}
This menu contains all automatic tools which can be run from \tool{seadali}.
Most of them are dealt with in other sections. 
In addition, this menu contains the
\button{Print hardcopy}-button, \index{hardcopy} which
prints the layout which is currently being edited. See
section~\ref{hardcopy} (page~\pageref{hardcopy})
for more details. 

\section{How to import cells from a library: the \protect\button{instances} menu}
\label{instances}
\index{instances}
Hierarchy is the most important way to handle the complexity of any
circuit. In the \button{instances} menu you can import other cells into
you design as 'son cells' (also called 'instances').  The son cells can
be imported from a remote library, which is nothing else than just
another project. Obviously you can also add son-cells which are in the
current project. 
\index{seadali!instance name|bold}
\tool{Seadali} displays the names of the instances in your design.
It also displays the cell name of each instance between '@' characters.
The following buttons are important to play around with instances:
\begin{description}
\item[\button{Madonna}]
\index{madonna@\tool{madonna}}
Automatically place the instances of a circuit.
See appendix~\ref{madonna} on page~\pageref{madonna} for 
all the naked details about \tool{madonna}.
\item[\button{ADD instance}]
\index{seadali!adding instances}
\index{instance}
Add a local instance to your design. After you have selected the instance
you must click its left bottom position. It is not necessary
to align the instances exactly because \tool{seadali} will
automatically snap them to the grid (if the 'image mode' is on). 
\index{seadali!image mode}
\tool{Seadali}
even performs automatic mirroring of small instances, so that they will
always match the grid. Also \tool{fish} can take care of the snapping.
\item[\button{ADD imported instance}]
Just like the previous command, but this button adds an imported
instance to your design. Use this button to add library cells. \tool{Seadali}
shows a list of all imported (library) cells. You can add libraries to your
project using \tool{mkopr}.
\item[\button{set instance name}]
%%\index{seadali!instance name}
Set the name of a specific instance. Notice that an instance which you
add by hand using \button{ADD instance} has no name yet. This button can be
useful to help the automatic router select the correct instance.
\item[\button{DELETE instance}]
Does just what you expect it to do: delete a specific instance.
\item[\button{move}, \button{mirror} and \button{rotate}]
Use these buttons to put the cell on its right place. In 'image mode',
%%\index{seadali!image mode}
\tool{seadali} will shift and/or mirror the cell automatically. 
On certain places in the fishbone the cells must be mirrored in the
x-axis.
\item[\button{Fish}]
Run the layout purifier, which will shift the instances to
a proper grid position. See appendix~\ref{fish} on page~\pageref{fish}
for more details.
\end{description}

\section{Editing mask layout: the \protect\button{boxes} menu}
\subsection{Selecting active masks}
To add a wire or any other mask pattern in \tool{seadali} you first
have to activate a mask.  
\index{seadali!active masks}
The bottom of the window shows the list of
available masks.  In the sea-of-gates process you must press
\button{in} to activate the first metal layer and \button{ins}
to activate the second metal layer. For your convenience 
the \mask{ins}-mask is automatically activated when 
\tool{seadali} is started.
\warning{
Only activate the \mask{in}, \mask{ins}, \mask{con}, \mask{cop},
\mask{cps} or \mask{cos} masks. On our sea-of-gates chip it is not
possible to add or remove patterns in any of the other masks.  If you
do so, \tool{fish} will complain and remove them automatically.}

\subsection{Making wires using \protect\button{APPEND}}
\index{creating!wires and boxes}
To add an arbitrary box in the active layer(s), just press 
\button{APPEND} and click at the appropriate positions in the 
layout. You can change the active masks also while this command is activated.
If \button{image mode} is activated,
\index{seadali!image mode} the boxes you
append are automatically snapped on the grid. Using the keyboard keys you can
zoom or pan the window using the keyboard keys (see table~\ref{keytable} on
page~\pageref{keytable}).

\subsection{Making wires with \protect\button{wire}}
\index{seadali!wire button}
Long, continuous wires can be added more easily with the \button{wire}
button.  Just enter the corners of the wire. Click \button{next} to
start a new wire or \button{ready} to return to the boxes menu.

\attention{Huh?! I am adding a wire but nothing is happening!}{
The answer is simple: you forgot to activate the layer, or you also
activated another more dominant layer (such as \mask{cps} or other
contacts). Also it is possible that you are adding layout over an
existing contact.  Since the contacts are displayed dominant, they hide
other layout which covers them.  Finally it is possible that you
accidently made a specific layer invisible in the \button{visible masks} menu
(see section~\ref{visible}, page~\pageref{visible}).}

\attention{Huh?! Some wires are drawn in a lighter color!}{
\index{seadali!hashed drawing style}
You cannot edit the layout of con-cells. To indicate the difference
between the different hierarchical levels, the layout of son-cells is
drawn in a {\bf hashed} style. See section~\ref{hashed} on
page~\pageref{hashed}.}

\subsection{Ensuring design-rule correctness}
\index{design rules}
For Sea-of-Gates layout it is useful to select the \button{image mode}
\index{seadali!image mode}
in the settings menu or the the \file{.dalirc}-file. If this mode is
enabled, the boxes you append are automatically snapped to the grid of
the image. Also, the wires have the proper width, and the vias
are aligned precisely. This behavior is disabled by switching off the
image mode. See section~\ref{sogmode} for more details.

Despite the features of image mode, the layout purifier~\tool{fish}
(see appendix~\ref{fish} on page~\pageref{fish}) should still be used before
writing anything to back to the database. \tool{Fish} performs more
elaborate design-rule checks on your layout.

\subsection{How to make contacts between layers?}
In the {\sl fishbone} sea-of-gates image there are basically only two types of
contacts: 
\index{seadali@contacts between layers}
\index{creating!contacts|bold}
\begin{enumerate}
\item
Between the image (\mask{ps} or \mask{od}) and metal~1 (\mask{in}):
\mask{con}, \mask{cop} or \mask{cps}
\item
Between metal~1 and metal~2 (\mask{ins}): \mask{cos}.
\end{enumerate}
To add a contact, use the \button{APPEND} command to make a small box
at the desired position.  For a contact between the image and metal~1,
you must activate the mask \mask{con}, \mask{cop} or \mask{cps}. Which
of these three you use doesn't matter because \tool{fish} takes care of
that.  For a contact between metal~1 and metal~2 you must use the
\mask{cos}-mask.

Press \button{Fish} to convert the contact mask into a design rule
correct pattern.

\warning{
There are some restrictions in the placement of the contacts. They may
not be stacked and they are not allowed in certain rows. See
section~\ref{eisen_metal} on page~\pageref{eisen_metal} for more details.}

In the {\sl octagon} image there are three layers and therefore masks for
contacts:
\begin{enumerate}
\item
Between the image (\mask{ps}, \mask{nn} or \mask{pp}) and metal~1 (\mask{in}):
\mask{co}
\item
Between metal~1 and metal~2 (\mask{ins}): \mask{cos}.
\item
Between metal~2 and metal~3 (\mask{int}): \mask{cot}.
\end{enumerate}
In the single-layer {\sl gate-array} image there is only one contact.
\subsection{Deleting layout: \protect\button{DELETE}}
\index{removing!wires}
Deleting is simple. Just activate the appropriate layers and click the
corners of the box to be deleted. To delete all the wires of you
current design, activate all masks and drag the box over your entire
design.

\attention{Huh?! I try to delete but it won't go away!}{
As usual, you are doing something wrong. First of all, check if you
have selected the right active layers. Secondly, remember that it is
NOT possible to edit the mask patterns of the son cells (instances). If
you want to change that you must edit that particular son cell. It is
convenient to use the hashed drawing style for instances to see which
\index{seadali!hashed drawing style}
wires belong to the son-cells (see section~\ref{hashed} on
page~\pageref{hashed}).}

\attention{Can I undo a deletion or any other command?}{
\index{seadali!undo}
No, unfortunately you cannot undo any action. Therefore you must be
careful. You should also be careful with the masks you activate.
Adding a big box in the wrong layer can destroy hours of work.  Save
your layout regularly to avoid frustration!}

\subsection{Copying layout: \protect\button{yank} and \protect\button{put}}
Sometimes it is convenient to copy or to move a part of the mask.  Just
like \button{DELETE}, \button{yank} puts the indicated box in a paste
buffer. The difference is that \button{yank} doesn't remove the box.
You can copy the contents of the paste buffer to a certain position by
pressing \button{put}.
\attention{Huh?! It doesn't seem to work!}{
Remember that it is not possible to yank the layout of son cells.  You
can only yank and put the layout of the (father) cell on which you are
currently working.}

\subsection{Purifying and checking your layout: \protect\button{Fish}}
\label{fish1}
\index{fish@\tool{fish}}
Pressing \button{Fish} makes your manual design design rule correct.
All wire segments will be shifted on the grid, and the instances will
be properly aligned. \tool{Fish} also shifts sufficient repetitions of
the fishbone image under your design.  A more elaborate description of
\tool{fish} will be given in appendix~\ref{fish} on page~\pageref{fish}.

\subsection{Automatically routing your design: \protect\button{Trout}}
\index{seadali!trout}
\index{trout@\tool{trout}}
This button will call the automatic router called \tool{trout}.
This is a rather complicated command. Pressing it without the
proper preparations will most likely result in an error message.
See appendix~\ref{routeman} for more details on how to use the router.

\subsection{Verifying layout connectivity: \protect\button{Check nets}}
\label{verify}
\index{design rules!checking connectivity|see{check nets}}
\index{seadali!check nets button}
\index{check nets|bold}
Pressing this button will verify your layout by comparing it with the
netlist in the circuit description. Your layout will be 'fished'
automatically if your press this button.  The check which is hidden
behind \button{Check nets} is quite strong. The following errors will
be detected:
\begin{itemize}
\item
All errors which \tool{fish} detects. In this case the errors of
\tool{fish} will be logged in the database file \file{seadif/sea.out}.
In contrast to the button \button{Fish}, no window with errors will pop up.
\item
Missing or superfluous instances in your layout. 
\item
Unconnected terminals of nets. The unconnects will be indicated in the
layout by a small box with the name of the net and an arrow.
The indicator will be placed on top of each terminal that has no
connection to another terminal of the same net.
\item
Short-circuits between different nets.  The program reports the names
of the nets which form a short-circuit.  These nets will also be
indicated in the layout in the familiar way: {\sf SHORT\_no} (\_no
is the number of the short circuit).  Notice that the indicators do not
indicate the exact spot of the short circuit. The actual short is
somewhere on the path between two indicators of the same short.
\end{itemize}

\attention{Huh?! How do I find the indicators in this huge layout?}{
In some cases the scale of the design is so large that the individual
indicators are hard to distinguish. One of the ways to find an indicator is by
zooming in and moving the window around.  For small layouts this is the easiest
way. For really large designs, however, this is a hell of a job, a bit like
finding a needle in a haystack.  The following trick will help you to find that
single stacked contact (or unconnect or whatever is indicated). Go to the main
menu and press
\button{visible masks} (see also section~\ref{visible}). Make everything invisible
except instances. The indicators will now be visible as little
white boxes (or dots). In this way you'll know where
to zoom.}

\section{Adding terminals: the \protect\button{terminals} menu}
\index{seadali!adding terminals|bold}
\index{creating!terminals in layout|bold}
\index{terminals!adding to layout|bold}
For simulation it is required to indicate the positions of the
terminals in the layout. If you routed the layout automatically using
\tool{trout} the terminals will also be placed automatically.  In the
menu \button{terminals} you can modify the terminal placement or make
one manually.

\subsection{How to add a terminal?}
Again, you must set the active layer of a terminal by clicking the
appropriate mask in the bottom row of the window. A terminal can only
be placed in the \mask{in} and \mask{ins} layer.  Next press
\button{ADD terminal} and drag a small rectangular box on the grid
position of the terminal. Next, \tool{seadali} will ask you to give a
name to this terminal. Enter it using the keyboard. The name must be
unique. Just give it the same name as in the circuit description you
made.

\subsection{How can I get rid of a terminal?}
\index{terminals!removing}
\index{removing!terminals}
That is very simple: use \button{DELETE terminal} and point at the terminal you
wish to delete. You can also delete all terminals within a specific area using
\button{DELETE terms in box}. Don't forget to set the proper active layer.

\section{Setting the visible layers: \protect\button{visible masks}}
\label{visible}
\index{seadali!setting visible masks}
In many cases it is not so useful to display all the masks of a cell.
Especially on our sea-of-gates many masks are not so relevant for the
electrical design, and will only make the picture unclear.  In the
\button{visible masks} menu you can switch on or of any mask you want. The
default settings of \tool{seadali} are such that the \mask{ps},
\mask{in} and \mask{ins} plus the contact masks are visible. All the
other masks are switched of. You can change the default settings by
editing the \fname{.dalirc} file in your project directory
(see~\ref{dalirc}).

\attention{How can I make the grid of the Sea-of-Gates image visible?}{
You can do this just by clicking \button{show Grid} of by pressing \button{g}
on the keyboard. The dots indicate the grid
positions (provided that {\bf image mode} is {\bf on}). The grid will not be drawn if
there are too many grid points in sight.}

\attention{When does \tool{seadali} display the image?}{
\index{image!displaying}
\index{seadali!displaying the image}
\tool{Seadali} will not draw the image if the scale of the layout is too big.
In this way long drawing times are prevented. If you zoom in enough
the image will become visible again. You can set the maximum amount of
image elements which should be drawn in the file~\file{.dalirc} (see
section~\ref{dalirc}). You you want to see the image at all times, switch of
the {\bf image mode}
%%\index{seadali!image mode}
in the settings menu (see section~\ref{sogmode}).}

\section{Some other nice features: the \protect\button{settings} menu}
\label{s-other-nice-features}
There are a few ways to customize the behavior of \tool{seadali}
according to your own wishes. In this section we'll list a few of them.
\subsection{Customizing \protect\tool{seadali}: the \protect\fname{.dalirc}-file}
\label{dalirc}
\index{dalirc@\fname{.dalirc} file}
\index{seadali!customizing}
Your project directory automatically contains a file called
\fname{.dalirc}.  This file is read by \tool{seadali} during startup.
You can edit this file to change various settings. Just have a look at
it and try change some parameters. Obviously, you must re-start
\tool{seadali} for the changes to have any effect.\\
\\
{\bf Huh?! I don't see any file called \fname{.dalirc} in my project!}\\
That is normal. In \smc{unix} all files which start with a '.' are not
visible. You can make them visible by typing:
\type{ls -a}
If it is still not there, copy the default file into your project:
\type{\mbox{cp \$ICDPATH/share/lib/process/fishbone/.dalirc .}}

\subsection{Switching on the Sea-of-Gates mode: \protect\button{image mode}}
\label{sogmode}
\index{seadali!image mode|bold}
\tool{Seadali} has a special {\bf image mode} for working with semi-custom
layout. This mode should always be turned on if you are working with
Sea-of-Gates. The {\bf image mode} has the following features:
\begin{itemize}
\item
Layout boxes and instances are snapped to the grid. \tool{Seadali} will
warn you if you try to place something at the wrong place. Library
cells are automatically mirrored in the x-axis if necessary or possible. 
\item
The position of the tracker is displayed in grid points, rather than in
lambda's. 
\item 
The grid of the image is displayed if you press \button{show Grid}.
The grid is only displayed if it is useful to do so.
\item 
The image is not drawn under your layout if the scale of the picture becomes
too big. If the {\bf image mode} is switched {\bf off}, the image is drawn
always\footnote{Provided its present as a instance in your layout.}.
\end{itemize}
This mode should generally be turned on if you are working with
Sea-of-Gates. Only in some exceptional cases (e.g. to force the
mirroring of a certain instance) you may want to turn it off.

\subsection{Setting drawing style for instances: \protect\button{draw hashed}}
\label{hashed}
\index{seadali!hashed drawing style|bold}
In many cases it is not so clear to which level of hierarchy a certain
wire belongs. Therefore \tool{seadali} displays the layout of all
son-cells in a less bright color than the wires at the current level.
The less bright color is achieved by hashing (45 degree lines). You can
switch this feature on or off by pressing \button{draw hashed} (also in the
\fname{.dalirc}-file). 

\subsection{Setting dominant or transparent drawing style:
\protect\button{Draw dominant}}
\index{seadali!dominant drawing style}
There is another drawing mode which can be set by you.  In the
transparent drawing mode the colors mix. In this way any crossing of
the masks results in a mix-color.  With many masks this makes the
picture rather confusing.

With the dominant drawing style, the layers are drawn as if they are
stacked on top of each other. In this case metal~2 (the top layer)
obscures any layout below it.  The default drawing mode is dominant.
You can switch by pressing the button \button{Draw dominant} or by setting
the parameter in the file \fname{.dalirc}. In the same file you can set the
order in which the layers should be drawn.

\subsection{Setting the print command for the \protect\button{Print hardcopy}-button}
\index{hardcopy} \label{hardcopy}
The button \button{Print hardcopy} in the \button{automatic tools} menu
causes the current layout to be printed by the laserprinter. 
It might be necessary to configure the proper print command to
perform this action. The proper command sequence can be 
set using the keyword \fname{Print\_command} in the file \fname{.dalirc}. 
To set the proper command, it is relevant to understand what \tool{seadali}
is doing after you pressed this command: 
\begin{enumerate}
\item
The current layout is written away in the \smc{nelsis} database 
using a temporary cell name.
\item
Any occurrences of \%s in the print command are 
replaced by the name of the temporary layout cell.
\item
The print command sequence is executed by a shell. 
\item
The temporary cell is removed.
\end{enumerate}
If no print command is set, \tool{seadali} will run the command \tool{playout}.
\index{playout@\tool{playout}}
\tool{Playout}, on its turn, will create a PostScript file using \tool{getepslay},
print it, and show the printer queue.

\subsection{X-window redrawing strategy: \protect\button{backing store}}
\index{backing store}
\tool{Seadali} asks the X-window server (your screen), to maintain a copy of the
window in its memory. This feature (called {\em backing store}) makes it
possible to move, obscure, iconify and then re-expose \tool{seadali}'s window
without redrawing the entire picture. Especially for large layouts this saves a
considerable amount of drawing time and \smc{cpu} load. The price we have to
pay for this nice feature is some extra memory consumption by the server. In
general this is not a problem, but for certain X-terminals without memory
management the extra (500K to 1M byte) memory load might cause some problems.
Therefore this feature can be switched off in the settings menu. You can also
switch it off in the file \fname{.dalirc} by adding the line:
``\fname{backingstore off}''.

\section{Interrupting \protect\tool{seadali}: keyboard input}
\index{seadali!keyboard input}
\index{seadali!interrupting}
\tool{Seadali} is sensitive for keyboard input. Table~\ref{keytable} shows all
the keycombinations which can currently be used. They serve as a convenient
alternative for clicking with the mouse in the menu. If \tool{seadali} is
drawing the picture, pressing {\em any} key interrupts the
drawing\footnote{There is one exception: during the reading or expansion
process \tool{seadali} cannot be interrupted.} within one second. 

\begin{table}
\begin{center}
\sf
\begin{tabular}{|l|l|} \hline  
key & function \\ \hline\hline
\button{space}/any key & interrupt (stop) drawing \\
'i' or '+'      & zoom in by a factor of 2\\
'o' or '-'      & zoom out by a factor of 2\\
\button{$\leftarrow$} or 'h'   & pan left \\
\button{$\rightarrow$} or 'l'  & pan right \\
\button{$\uparrow$} or 'k'    & pan up \\
\button{$\downarrow$} or 'j'   & pan down \\
'c' & center window around current cursor position\\
'b' & set window to bounding box \\
'p' & previous window \\
'r' & redraw screen \\
'1','2','3',$\ldots$,'9'  & set expansion level\\
'0'             & expand maximum \\
's'             & show sub terminals \\
'd'             & toggle hashed drawing style\\
'D'             & toggle dominant drawing style \\
'g'             & toggle display of grid\\
't'             & toggle tracker (cursor position display)\\ \hline
\end{tabular}
\rm
\end{center}
\caption{
The key commands for \protect\tool{seadali}.}
\label{keytable}
\end{table}
\index{seadali@\tool{seadali}|)}

\cleardoublepage

\section{Fish: The Layout Purifier and Design Rule Checker}
\label{fish}
\index{fish@\tool{fish}|(bold}
\index{design rules!checking|see{\tool{fish}}}
In some of the menus of \tool{seadali} you've encountered the button
\button{Fish}. In this appendix we'll describe a bit more
about what is behind this button.
\tool{Fish} is a layout purifying program for sea-of-gates layout.
It reads a layout, processes it, and produces a new enhanced cell. The
basic idea is to allow some design-rule errors during the manual design
of a gate-array cell.
\tool{Fish} corrects these errors and produces a design rule correct result.
In this way the process of manual design using \tool{seadali} can be
speeded up considerably. \tool{Fish} also prints warning messages for a
variety of design rule errors.  You can call \tool{fish} in two ways:
\begin{enumerate}
\item
By pressing the button \button{Fish} in \tool{seadali}. This is the
most common way of using it. What is actually happening is that
\tool{seadali} writes your design as a temporary cell into the
database, then it calls \tool{fish}. If everything goes well,
\tool{seadali} re-reads the purified cell. This will take a few
seconds, depending on the load of the computer.
\item
By typing {\tt fish} on the command line: \type{fish <cell\_name>} This
purifies the cell named {\tt <cell\_name>}. If you want to know more:
try {\tt fish -h} for a brief list of all options.
\end{enumerate}

\tool{Fish} needs an image description file. This file is
called \fname{image.seadif} 
\index{image.seadif@\fname{image.seadif} file}
\index{design rules!file|see{\fname{image.seadif}}}
and should be present in the directory
\file{seadif} in your project. It
contains data about the positions of grid points, the way in which the
image should be repeated, the kind of contact which should be used,
etc. If you like you can have a look at it.

\section{What errors does \protect\tool{fish} detect?}
\index{design rules!for contacts}
\tool{Fish} performs limited design rule checking of your layout.
It will report errors for:
\begin{itemize}
\item
Stacked contacts. 
\index{stacked contacts}
The process design rules of the fishbone image do not
allow to stack an \mask{in} to \mask{ins} contact (= \mask{cos}) on top
of another via. The wire would crack at that point.
\tool{fish} will indicate the error in the layout by a small
box with an arrow pointing to it. \tool{Fish} 
does NOT detect a pair of stacked contacts if one of the contacts is in
a son-cell.
\item
\index{design rules!illegal vias}
Contacts at an illegal position. A \mask{cos} contact is not allowed in
the rows of the polysilicon gate contacts.  The error is again
indicated in the layout by a small box with an arrow.
\item
Patterns (boxes) in illegal masks. \tool{Fish} removes any patterns in
\mask{ps}, \mask{od}, etc. because they may not be used on a
semi-custom chip.
\item
Patterns which are not (entirely) in the first quadrant. In our system
all layout boxes must have positive coordinates, that is, it must be to
the right-top of the origin. The illegal boxes will be removed.
\end{itemize}

Anyway, you must keep in mind that \tool{fish} is not too clever.  It
will not warn you of short-circuits and mis-aligned instances (such as
forgotten mirroring in the x-axis). For that you must use the button
\button{Check nets} (see section~\ref{verify} on
page~\pageref{verify}).  A list of all the errors which were detected by
\tool{fish} will be displayed in a pop-up window.

\section{How does \protect\tool{fish} handle instances?}
\tool{Fish} snaps instances to the nearest grid point. To
perform this snapping, \tool{fish} opens the instance and looks for
contacts. The first contact it encounters is used for the snapping.
Notice that in this way \tool{fish} assumes that the instances are on
grid (that is, fished).  If it can't find any contacts \tool{fish} gets
angry and it doesn't align the instance.

\section{How to create an empty array of image?}
\index{image!making an empty array}
Before editing it is convenient to have an empty array to indicate the
transistor and grid positions.  This can be done by pressing the button
\button{Fish} in \tool{seadali} when the design is empty. A $20 \times
1$ array will be generated.  You can also do the same from the command
line by creating a new (empty) cell into the database:
\type{fish -x 20 -y 1 -o nand2}
The option {\tt -o nand2} instructs \tool{fish} to write the empty
array into a cell called {\tt nand2}.
You can now start editing your cell by adding wires.
\attention{But what should I do if the $20 \times 1$ image doesn't fit anymore?}{
Should the underlying image become too small, simply hit the button
\button{Fish} to enlarge it automatically so that it fits the wire pattern.
Alternatively you can enlarge the basic
image by pressing press \button{set array parameters} in the \button{instances} menu.
Set \button{nx} or
\button{ny} to the new value. Do not touch the dx and dy buttons!}

\index{fish@\tool{fish}|)}

\cleardoublepage

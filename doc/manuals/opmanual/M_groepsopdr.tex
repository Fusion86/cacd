\section{De Groepsopdracht}

\subsection{Inleiding}

In dit hoofdstuk worden beschrijvingen gegeven van de mogelijke
hoofdopdrachten:
\begin{itemize}
\item
de Dialmemo,
\item
de Avant-Garde Klok,
\item
de Infrarood-Besturing,
\item
de Reactiemeter,
\item
de Pong Spelcomputer,
\item
Alternatieve opdracht.
\end{itemize}
E\'en van deze opdrachten zult u met uw groep gaan ontwerpen.\\
Naast de beschrijvingen van de opdrachten vind u ook bij elke opdracht
een aantal specificaties waaraan de schakelingen moeten voldoen.
Deze specificaties zijn niet volledig, maar bedoeld als eerste aanzet.
Zelf zult u nauwkeuriger moeten specificeren. \\
Nadat de opdrachten afzonderlijk zijn beschreven worden er er vervolgens nog
enige algemene randvoorwaarden aangegeven waar u bij het ontwerpen terdege
rekening mee dient te houden.\\
Tenslotte volgen nog een paragraaf met enige
adviezen over hoe u de systeemspecificatie
het best aan kunt pakken, een paragraaf over de beschikbare ruimte
op de chip, een paragraaf die beschrijft hoe tijdens het ontwerpen
al rekening moet worden gehouden met het testen dat later plaatsvindt,
en een paragraaf
over de afronding van het ontwerp. \\
Voor analoge deelschakelingen die mogelijk voorkomen in 
het ontwerp wordt men verwezen naar appendix \ref{analoog}.

\subsection{De Dialmemo}

\subsubsection{Inleiding}

Beeld u eens in dat u werkt bij de firma TECHNOGADGET.
Zoals de naam al suggereert zit uw bedrijf in de
relatiegeschenkenbranche. 
Een aantal jaren geleden vond de oprichter van TECHNOGADGET
(een oud-TU-student) de alom bekende ''keyfinder'' uit. De keyfinder
is een groot model sleutelhanger (met firma-opdruk) die gaat piepen
bij een plotseling hard geluid (b.v.\ fluiten). Het succes van dit idee
was zo enorm dat deze student z'n lelijke eend al spoedig kon
inwisselen voor een glimmende Porsche.\\
Elke doorgewinterde beursbezoeker heeft ondertussen al hele
sleutelbossen vol keyfinders. De markt raakt daardoor een beetje
verzadigd en bovendien is het nieuwtje er langzamerhand een beetje van
af. Als gevolg daarvan gaan de zaken voor TECHNOGADGET momenteel
slecht. Het is duidelijk dat het bedrijf snel een nieuw produkt nodig
heeft, anders moet binnenkort de lelijke eend weer van stal gehaald
worden.\\
Op een blauwe maandagmorgen kreeg de directeur weer een briljant idee:
de DIALMEMO. Het is weer een groot model sleutelhanger, maar deze keer
bevat hij een heel klein telefoontje.
De DIALMEMO bestaat uit een toetsenbord van een druktoets-telefoon,
een chipje, een luidsprekertje en een batterij, alles in een klein
elegant huisje met firma-opdruk. Door het apparaatje tegen de
microfoonkant van een telefoonhoorn te houden en vervolgens op een
knop te drukken, wordt het firma-nummer automatisch gebeld. Ook is het
met de DIALMEMO mogelijk om een draaischijf-telefoon als
druktoets-telefoon te gebruiken.
Technisch is het principe eenvoudig. Moderne telefoon-centrales werken
met een toonkies-systeem. Hierbij wekt elke knop van een
druktoets-telefoon een specifieke toon-combinatie op die in de centrale
herkend wordt. In plaats van de toetsen \'e\'en voor \'e\'en in te
drukken, fluit de DIALMEMO het nummer rechtstreeks in de microfoon van
de hoorn. Door de eenvoudige opbouw moet de DIALMEMO voor rond de
1 euro per stuk in China te produceren zijn. \\
Het is duidelijk dat dit een revolutionaire vinding is. Zo kan
bijvoorbeeld de slogan ''even Apeldoorn bellen'' een volledig nieuwe
dimensie krijgen. Iedere verzekerde krijgt nu als welkomsgeschenk een
DIALMEMO met daarin opgeslagen het nummer in Apeldoorn. Centraal
Beheer Verzekeringen was waanzinnig enthousiast!\\
Voor allerlei andere
bedrijven die sterk van de telefoon afhankelijk zijn, is de DIALMEMO
een uitkomst. Vooral voor 06-nummers ligt een enorme markt braak. Zo
zou de exploitant van ''Harry's gay box'' (een homo-babbelbox) er al
over denken om een DIALMEMO (met daarin zijn nummer) gratis toe te
sturen aan alle abonnees van de gay-krant. Bij gemiddeld 15 minuten
babbelen (\`a 50 cent per minuut) per DIALMEMO is de exploitant al uit
de kosten!\\
U krijgt als groep de opdracht van TECHNOGADGET om de chip te maken
die het hart vormt van de DIALMEMO.


\subsubsection{Specificaties Dialmemo-chip}


\begin{itemize}
\item
Als invoer dient een toetsenbord, een losse druktoets en een
aan/uit-schakelaar. Het toetsenbord heeft de bekende telefoon-layout,
d.w.z.\ de cijfers 0 t/m 9, \# en *. De laatste toetsen dragen in ons
geval resp. de namen ''read'' en ''dial''.
De losse toets draagt de naam ''store''.\\
Omdat het toetsenbord en ook de losse druktoets buitengewoon low-cost
zullen zijn, moet er rekening gehouden worden met het ''denderen'' van
de contacten. Dit ''denderen'' houdt in dat de impedantie-verandering
bij het indrukken van een toets niet monotoon verloopt. In de
paragraaf ''Overige Specificaties'' wordt hier verder op ingegaan. De
toetsen van het toetsenbord zijn in een 3x4-matrix geschakeld.\\
De uitgangen van het toetsenbord en de losse druktoets worden direct
verbonden met de pinnen van de chip.
\item
Aan de uitgang van de chip verschijnen toon-combinaties volgens de
offici\"ele CCITT-norm voor telefonie. Zie hiervoor de ''Overige
Specificaties''.
\item
Om de DIALMEMO als gewone telefoon te gebruiken, dient eerst een
nummer ingetoetst te worden, waarna door de dial-toets (*) te
drukken, dit nummer afgespeeld wordt. Als vervolgens nogmaals dial
wordt gedrukt, wordt het laatst ingetoetste nummer opnieuw afgespeeld.
Op deze manier is er dus een ''redial''-mogelijkheid.\\
Dit ''dial-geheugen'' dient maximaal 16 cijfers te kunnen opslaan.\\
FACULTATIEF: Bij intoetsen van het nummer dienen de betreffende
toon-combinaties hoorbaar te zijn.
\item
De schakeling heeft naast het dial geheugen ook een ''vast'' geheugen
dat ook een telefoonnummer van 16 cijfers kan opslaan. Dit ''store''
geheugen kan worden geprogrammeerd door de (losse) store toets in te
drukken na invoer van een nummer. Oproepen van dit geheugen geschiedt
door de read toets te drukken.\\
Het is de bedoeling dat de store toets niet in te drukken is door Jan
de Beursbezoeker.
\item
De DIALMEMO is een wegwerpartikel en zal geen verwisselbare batterij
hebben. Toch moet het apparaat enkele jaren mee kunnen gaan. Hiervoor
is vooral in rusttoestand een laag stroomverbruik noodzakelijk.
\item
De inhoud van zowel het dial als het store geheugen mag niet verloren
gaan wanneer het apparaat zich in rusttoestand bevindt of geactiveerd
wordt.
\item
FACULTATIEF: De schakeling moet bestand zijn tegen het tegelijkertijd
indrukken van meerdere toetsen.\\
Bedenk zelf wat er moet gebeuren als een telefoonnummer wordt ingetoetst
dat langer is dan 16 cijfers.
\end{itemize}




\subsubsection{Overige Specificaties}

\paragraph{CCITT-norm voor telefonie}

Bij het zgn.\ Dual-Tone Multiple-Frequency (DTMF)-systeem voor telefonie hoort bij elke toets een combinatie van een lage en een hoge toon.
Dit is schematisch in figuur \ref{CCITT} aangegeven.

%figuur CCITT.eps
\begin{figure}[h]
\centerline{\callpsfig{CCITT.eps}{width=.22\textwidth}}
\caption{Frequentie verdeling}
\label{CCITT}
\end{figure}

Hiervoor gelden de volgende specificaties:
\begin{tabbing}
xxxxxxxxxxxxxxxxxxxxxxxxxxxxxx\=xxxxxxxxxxxxxxx\=xxxxxxx\=\kill
gebruikte frequenties: 	\>laag: 	\>697, 770, 852, 941 (Hz)\\
			\>hoog:		\>1209, 1336, 1477 (Hz)\\
onnauwkeurigheid:	\>$<$ 1.5 \%	\\
vermogensverschil tussen tonen:  	\>$<$ 6 dB\\
signaalduur:		\>$>$ 40 ms\>$<$ 2 s\\
signaalpauze:		\>$>$ 70 ms\>$<$ 2 s\\
signaal(duur+pauze):	\>$>$ 120 ms\>$<$ 2 s\\
signaalvermogen:	\>100 $\mu\!$W ($\pm$ 30 \%)\\
harmonische distorsie:	\>$<$ 0.01 of -20 dB\\
\end{tabbing}

\paragraph{Kristal-oscillator}

De onnauwkeurigheid van een kristal-oscillator is kleiner dan 1 \%.



\paragraph{Toetsenbord}

\begin{tabbing}
xxxxxxxxxxxxxxxxxxxxxxxx\=xxxxxxxxxxxxxxx\=xxxxxxx\=\kill
vorm:			\>3x4-matrix\\
impedantie:		\>uit: 		\>$>$ 10 M$\Omega$\\
			\>aan:		\>$<$ 100 $\Omega$\\
stabiliseringstijd:	\>$<$ 50 ms\\
\end{tabbing}



\paragraph{Luidspreker}

\begin{tabbing}
xxxxxxxxxxxxxxxxxxxxxxxx\=xxxxxxxxxxxxxxx\=xxxxxxx\=\kill
impedantie:		\>50 $\Omega$ ($\pm$ 10 \%)\\
resonantie-frequentie:	\>440 Hz\\
bandbreedte:		\>350-5000 Hz\\
vermogen:		\>$<$ 0.3 W\\
rendement:		\>100 \% (aanname)\\
\end{tabbing}


\subsection{De Avant-Garde Klok}

\subsubsection{Inleiding}
Klokken zijn behalve technische hoogstandjes ook altijd mode-objecten geweest.
Iedere periode in de geschiedenis kent zijn eigen karakteristieke uitvoering van een tijdmeter.
Het ontwerpen van zo\-iets ``gewoons'' als een klok kan daarom nog heel bijzonder zijn.
De informatie-overdracht van klok naar mens kan op talloze manieren verzorgd worden.
Dit loopt van de ouderwetse wijzerplaat met wijzers tot aan de elektroden op de grote hersenen met een bionisch RS232-interface.
Puur technisch gezien wordt van klokken een steeds grotere nauwkeurigheid verlangd.
Dit alles in beschouwing genomen blijft het nog steeds een uitdaging om een klok te ontwerpen die behalve als functio\-neel technisch object ook zijn plaats in de wereld verdient door zijn uitvoeringsvorm.
De directeur van TECHNOGADGET had op een druilerige zondagmiddag het idee
gekregen om een klok met een ultieme nauwkeurigheid te maken. Hij dacht
ineens aan een zender die in Duitsland staat, die met atoomnauwkeurigheid
de tijd gedecodeerd uitzendt, de zogenaamde DCF-zender.
Na verdere ontwikkeling van zijn idee\"en, heeft hij contact opgenomen met
de beroemde Italiaanse designer Verstamo, om
een klok te ontwerpen die eeuwigheidswaarde heeft.

Er zijn talloze manieren om de tijd zichtbaar te maken. De ontwerper
heeft gekozen moderne middelen  te gebruiken, om de dynamiek van
de techniek op de bezitter van de klok over te dragen. Hiervoor gebruikt hij
een LED-display voor de aanduiding van de uren, zeven LEDs voor de
aanduiding van de weekdag en twee draaispoelmeters voor de aanwijzing
van de minuten en seconden.
Om de gebruikers te helpen de klok zo te plaatsen dat deze het DCF-signaal
kan ontvangen, is ook op een strategische plaats een LED aangebracht,
die aangeeft of het DCF-signaal wordt ontvangen.
Zo mogelijk zal de klok ieder kwartier het wijsje spelen dat op dat moment ook door de Big Ben in Londen wordt gespeeld.
Speciaal voor mensen die ooit in Londen geweest zijn maakt dit de klok natuurlijk extra aantrekkelijk.

Helaas heeft Verstamo geen verstand van elektronica, dus de chip die dit alles moet besturen zal door de echte deskundigen ontworpen moeten worden.
Hier is natuurlijk haast bij, want zoals alle leken denkt ook Verstamo dat elektronisch alles zo maar kan.
Hij heeft zijn klok dus al verkocht aan een grote fabrikant die reeds een grote reclamecampagne heeft opgezet.
Verstamo kan dan ook een claim van ongeveer 100 miljoen euro aan ge{\ii}nvesteerd reclamegeld verwachten indien het ontwerp niet op tijd komt.
Zijn adviseurs hebben hem al gewaarschuwd dat simulaties ter verificatie van de chip veel tijd kunnen gaan kosten.
Wordt bijvoorbeeld een schakeling gesimuleerd met een klokfrequentie van 6 MHz waaruit een puls per week moet komen, dan moet hij beseffen dat de simulator niet veel meer dan 6 miljoen gebeurtenissen per seconde rekentijd kan simuleren en dat de simulatie dus ook makkelijk een week zal kunnen gaan duren.
Verder zitten er in een week bijzonder veel seconden, zodat data-opslag voor de simulator ook wel eens een probleem zou kunnen gaan worden.
Ook later, wanneer de chip uit de fabriek komt zal het ondoenlijk zijn de chip een week lang in een tester te stoppen om te kijken of de weekaanduiding werkt.
De chip zal niet op een veel hogere klokfrequentie kunnen werken, dus hier zal weinig winst te behalen zijn.
\\
{\em Het is daarom zaak ruime aandacht te besteden aan simuleerbaarheid en testbaarheid en niet aan een grote hoeveelheid (ontestbare) elektronische grapjes.}
Het aanbrengen van verschillende testmogelijkheden zoals het ``opdelen'' van delerketens die los testbaar zijn binnen een redelijke tijd is van zeer groot belang.
\subsubsection{Specificaties DCF-ontvanger}
\begin{itemize}
\item Als invoer dient een antenne-signaal, dat de DCF-code bevat en
door de al bestaande ontvanger wordt omgezet naar een signaal
dat direct als invoer kan dienen voor de schakeling, die voor de
verwerking van dit signaal zorgt.
Deze signalen hebben logische niveaus die door deze schakeling
verwerkt kunnen worden, maar
nog niet gesynchroniseerd en gedecodeerd zijn.
\item Als uitgang van de DCF-klok dienen enige displays die de
tijd en ook een luidspreker die de tonen
van de Big Ben melodie hoorbaar maakt. Bij het
genereren van deze tonen kunnen verschillende luxe-klassen worden onderscheiden.
De tonen kunnen een verschillende tijdsduur hebben, maar ook een volume
dat in de tijd afneemt. Dit geeft het effect of een echt klokkenspel
wordt bespeeld. Deze opties brengen wel een ingewikkelder en groter
ontwerp met zich mee!
\item De nauwkeurigheid van de klok wordt geleverd door de
beroemde DCF-tijdzender.
Deze zender zendt op 77.5 kHz een signaal uit in de vorm van een serie bits.
Elke minuut wordt er een reeks van 59 bits uitgezonden die de complete tijd weergeeft (tijd, dag, maand, jaar en controle-bits).
De afwijking is ongeveer 1 seconde per 300.000 jaar, dus dat is net voldoende.
Een bijzonder simpele ontvanger (die niet ontworpen hoeft te worden) levert aan de uitgang een digitaal signaal (pulsen van 100 ms en 200 ms), waaruit de tijd bepaald kan worden.
\item De ontvangst van het DCF-signaal zal doorgaans goed zijn.
Als de zender echter niet ontvangen wordt, of als er storingen
worden gedetecteerd, dan zal de klok moeten besluiten het voorlopig
maar met het eigen kristal te doen.
Zodra goede ontvangst weer mogelijk is, zal de klok zich weer
precies gelijk zetten.
Wanneer een DCF-signaal wordt ontvangen, moet de klok dit aangeven
door het laten branden van een LED.
\item De firma TECHNOGADGET verwacht dat de eerste serie exclusieve klokken
nog voor grote bedragen verkocht kunnen worden. Wanneer de klokken bekend
worden, wil iedereen zo'n klok kopen. Om iedereen in staat te stellen zo'n
klok te kopen, moeten ook goedkopere versies gemaakt kunnen worden.
Door verschillende mogelijkheden niet te gebruiken, kan een goedkope versie worden aangeboden.
Door een modulair ontwerp te maken, is het eenvoudig om bepaalde
onderdelen weg te laten. Dit heeft als voordeel dat
de verschillende circuitdelen los van elkaar getest kunnen worden, omdat ze
ook los van elkaar kunnen werken. Dit is ook prettig voor de fabrikant, omdat
deze bij uitval tijdens de fabricage, wanneer bijvoorbeeld
het speelwerk defect is, de klokken kan gebruiken
in een wat goedkopere versie van de DCF-klok zonder geluid.
\end{itemize}

\subsubsection{Overige specificaties}

\paragraph{DCF-signaal}

De codering van het DCF-signaal ziet er als volgt uit:

\begin{figure}[bth]
\centerline{\callpsfig{dcfcode.eps}{width=.6\textwidth}}
\caption{Codering van het DCF-signaal}
\label{dcfcode}
\end{figure}

\begin{itemize}
\item
Alle waarden worden gegeven in BCD-code.
\item
Er zijn drie pariteitsbits (controle-bits: P1, P2, P3) aanwezig. Deze kunnen
genegeerd worden.
\item
Bij de weekdag is maandag als ''001'' gecodeerd. De overige coderingen zijn
triviaal.
\end{itemize}
\begin{figure}[bth]
\centerline{\callpsfig{dcfsignaal.eps}{width=.5\textwidth}}
\caption{Vorm van het DCF-signaal}
\label{dcfsignaal}
\end{figure}


Het signaal zelf ziet er ongeveer zo uit als bovenaan aangegeven is in figuur
\ref{dcfsignaal}

Een (overigens vrij eenvoudige) ontvanger, die u niet zelf hoeft te maken, maakt
 er een signaal van dat lijkt op het middelste signaal. Dit signaal wordt
aangeboden aan de chip.\\
De DCF-zender genereert om de seconde een puls
met een breedte van 100 ms (voor een '0') of 200 ms (voor een '1').
Op de 60-ste seconde wordt geen puls uitgezonden.\\
Het onderste signaal geeft aan hoe u dit zou kunnen decoderen.

\paragraph{Big Ben melodie}

de Big Ben melodie ziet er als volgt uit:\\


\begin{figure}[bth]
\centerline{\callpsfig{bigben.ps}{width=.8\textwidth}}
\caption{Big Ben melodie}
\label{Big_Ben}
\end{figure}


De toonhoogten die gebruikt worden zijn:

\begin{tabular}{||l|l||} \hline
C & 1046.5 Hz \\ \hline
F & 1396.9 Hz \\ \hline
G & 1567.9 Hz \\ \hline
A & 1760.0 Hz \\ \hline
\end{tabular}


\paragraph{Draaispoelmeter}

De draaispoelmeters zijn lineair en vragen voor een volledige uitslag
een stroom van 2 mA. De impedantie is kleiner dan 100 $\Omega$.

\paragraph{Uitgangen}

De uitgangen van de DCF-klok kunnen bij een te grote belasting
gebufferd worden. Om een goede aansturing van de buffers mogelijk te
maken, moet de uitgangsstroom van de (analoge) signalen (extern)
regelbaar zijn.

\paragraph{Toetsen}

\begin{tabbing}
xxxxxxxxxxxxxxxxxxxxxxxx\=xxxxxxxxxxxxxxx\=xxxxxxx\=\kill
impedantie:             \>uit:          \>$>$ 10 M$\Omega$\\
                        \>aan:          \>$<$ 100 $\Omega$\\
stabiliseringstijd:     \>$<$ 50 ms\\
\end{tabbing}


\paragraph{Luidspreker}

\begin{tabbing}
xxxxxxxxxxxxxxxxxxxxxxxx\=xxxxxxxxxxxxxxx\=xxxxxxx\=\kill
impedantie:             \>50 $\Omega$ ($\pm$ 10 \%)\\
resonantie-frequentie:  \>440 Hz\\
bandbreedte:            \>350-5000 Hz\\
vermogen:               \>$<$ 0.3 W\\
rendement:              \>100 \% (aanname)\\
\end{tabbing}



\paragraph{Functies}

Er zijn een aantal duidelijk te onderscheiden functies in de klok
waardoor de firma TECHNOGADGET in staat is deelontwerpen uit te besteden
aan verschillende teams:
\begin{itemize}
\item
De DCF-decoder, die in staat is de dcf-pulsen te decoderen en om te zetten in logische niveaus.
\item
De schakeling die besluit of er een geldige datareeks is binnengekomen uit
de ontvanger.
\item
Een ``autonome klok'' die parallel de DCF-data kan laden als deze geldig blijkt te zijn, maar die zelfstandig kan lopen in andere gevallen.
\item
Een display sectie, die de LEDs en de draaispoelmeters kan aansturen.
\end{itemize}

\subsection{De Infrarood-Besturing}
\subsubsection{Inleiding}
Een instituut beschikt over een ionenbron, die ingesteld wordt m.b.v.
een vijftal parameters, die alle kunnen worden ingesteld door een
getal tussen de 0 en 99. Deze ionenbron is geplaatst in een afgeschermde
ruimte omdat hij op een hoge spanning moet zijn aangesloten.
Probleem is nu, dat steeds wanneer voor een proef met deze bron een
bepaalde parameter moet worden gewijzigd, de gehele opstelling moet worden
uitgeschakeld om de betreffende parameter te kunnen instellen.
Gelukkig voor dit instituut liep er een slimme TUD-student stage, die
het idee opperde om een infrarood kanaal te maken, waarmee de waarden voor
de parameters naar de bron-opstelling zouden kunnen worden overgebracht,
zonder de opstelling uit te schakelen.
Hij merkte op dat de devices voor het instellen van de parameters
reeds een busaansluiting hadden, waarop, in binaire code de waarde van
de parameter kon worden geplaatst, waarna deze waarde werd overgenomen wanneer
de enable lijn van het betreffende device gedurende tenminste
100 us 'hoog' werd.
Bovendien wist hij (hij had het ontwerppracticum met goed gevolg doorlopen
en toen gepoogd een dialmemo te maken), dat er van die handige telefoon-
toetsenbordjes bestaan, die heel goed als invoermedium van gegevens kunnen
dienen.\\
Helaas, toen dit allemaal bedacht en besproken was met de directie van het
instituut (waar veel tijd in ging zitten), was zijn stagetijd om.
Maar gelukkig kan een nieuwe op-goep uitkomst brengen door dit
infrarood kanaal te realiseren, door een zend- en een ontvang-chip
voor dit probleem te maken met de onderstaande specificaties.
\subsubsection{Specificaties infrarood-kanaal}
\begin{itemize}
\item Het infrarood-kanaal zal bestaan uit twee chips: een zender-chip,
      die een pulsreeks uitzend om de parameters in te stellen, en een
      ontvanger-chip die de uitgezonden pulsreeks decodeert en de
      parameters de juiste waarden geeft.
\item Als invoermedium voor het kanaal zal gebruik worden gemaakt van een
      telefoon toetsenbord, waarbij de volgende afspraken gelden:
      \begin{itemize}
      \item De nummers op het toetsenbord worden gebruikt om de waarden van
            de parameters op te geven.
      \item De *-toets wordt gebruikt om het device waarvan de parameter
            moet worden veranderd aan te geven. Steeds wanneer op deze
            toets wordt gedrukt zal het volgende device worden aangewezen.
      \item De \#-toets wordt gebruikt om de opgegeven waarde van de parameter
            in het aangegeven device te laden.
      \end{itemize}
      Bovendien dient er rekening mee te worden gehouden, dat het toetsenbord
      een ontdendertijd heeft van 40 ms.
\item De zender moet tevens een uitgang hebben, waarop een 7-segment display
      kan worden aangesloten, waarop de laatst ingedrukte toets is zichtbaar
      gemaakt. Hierbij moet de *-toets worden getoond als een 'A' (advance) en
      de \#-toets als een 'U' (update).
\item De ontvanger moet een 7-bits brede uitgang hebben,
      waarop de 7-bits binaire code
      voor de parameter-waarde wordt geplaatst en een uitgang voor het
      enable bit van ieder device, waarmee de parameter-waarde in dit device
      kan worden geplaatst.
\item De laatst verstuurde waarden van elk device, dit zijn dus de waarden
      waarop de bron staat ingesteld moeten worden zichtbaar gemaakt op een
      LCD. Voorts moeten hierop ook de nieuw in te stellen waarde worden
      getoond, alsmede een indicatie voor het geselecteerde device waarop
      deze waarde zal worden geplaatst.\\
      Hiertoe is een LCD aanwezig (met documentatie), waarop  2 regels van elk
      16 karakters kunnen worden weergegeven.
\item Daar de zender-chip en ontvanger-chip alleen met elkaar communiceren
      via het infrarood-kanaal en er verder geen verbinding tussen de twee
      chips mogelijk is, wordt er dus ook voor het zend- en ontvang-gedeelte
      gebruik gemaakt van een afzonderlijke klokpuls.\\
      Er moet dus rekening mee worden gehouden, dat deze twee klokpulsen
      geen fase-koppeling hebben.
      Ook moet er rekening worden gehouden met een eventueel verschil in
      frequentie van beide pulsen van maximaal 5\%.
\item De pulsduur van de enable-signalen van de devices moet minimaal 100 us
      bedragen, en gedurende minimaal 50 us voor deze puls tot 50 us
      na deze puls moet de waarde op de parameter-uitgangen stabiel zijn.
\end{itemize}

\subsection{Een Reactiemeter}
\subsubsection{Inleiding}
Tijdens een feestelijke bijeenkomst raakten een drietal studenten in gesprek
over hun uiterst snelle reactietijd op een gebeurtenis (zelfs na een aantal
glazen bier).
Natuurlijk vond elk van de drie dat hij toch verreweg de kortste reactietijd
had.
De discussie liep hoog op, tot \'e\'en van de studenten op het idee kwam om een
schakeling te maken, die voor eens en altijd kon uitmaken wie van hen werkelijk het snelst was.
Ze bedachten hierna zo'n schakeling die de hieronder beschreven werking
zou moeten hebben.
Helaas is het tot een realisatie van de schakeling nooit meer gekomen.
Maar hier kan een op-groep wellicht uitkomst brengen.

\subsubsection{Werking reactiemeter}
De reactiemeter moet twee cijfers weergeven: een in te stellen referentiecijfer
en een steeds veranderend reactiecijfer.
Voorts moet voor elke deelnemer een drukknop aanwezig zijn en een
tijdweergave van hun reactietijd.\\
Wanneer het referentiecijfer en het reactiecijfer gelijk zijn moeten de
deelnemers zo snel mogelijk hun drukknop indrukken.
Wie het eerst heeft gedrukt behoud zijn tot dan toe gebruikte reactietijd.
Bij de twee andere deelnemers moet het verschil tussen hun reactietijd en de
reactietijd van de snelste deelnemer bij hun totale reactietijd worden
opgeteld.
Wie zijn toegestane totale reactietijd (bijv. 1 seconde) heeft gebruikt valt af.
Degene die overblijft is de winnaar.\\
Om te voorkomen, dat 'slimme' deelnemers constant hun drukknop indrukken of
ingerukt houden moet ook een straftijd aan de reactietijd van een deelnemer
worden toegevoegd, wanneer hij zijn drukknop indrukt terwijl de cijfers
niet gelijk zijn.

\subsubsection{Specificaties reactiemeter}
In verband met het bovenstaande moet de reactiemeter voldoen aan de volgende
specificaties:
\begin{itemize}
\item Er moet een in te stellen referentiecijfer worden weergegeven.
\item Er moet een reactiecijfer worden gemaakt en getoond, dat om de
      seconde (pseudo-)random verandert.
\item Er moet een drietal drukknoppen aanwezig zijn waarop moet worden gedrukt
      wanneer beide cijfers gelijk zijn.
\item Er moet een drietal tijdweergaven zijn, waarop de geaccumuleerde
      reactietijden van de drie deelnemers zichtbaar zijn.
\item Voor elke deelnemer is een indicatie aanwezig, die aangeeft wanneer
      een deelnemer zijn totale reactietijd heeft overschreden.
\item Er moet een reset knop eenwezig zijn, waarmee de schakeling kan worden
      gereset.
\item De reactietijden moeten worden gemeten met een nauwkeurigheid
      van 0.01 seconde.
\item Voor de klok van de schakeling mag gebruik worden gemaakt van een
      externe pulsgenerator.
\item Voor de totale reactietijd die mag worden gebruikt moet een geschikte
      waarde in de orde van grootte van 1 seconde worden gekozen.
\item De 'straf' voor het foutief indrukken van de drukknop mag door de
      ontwerper(s) van de schakeling worden bepaald.
\end{itemize}
Voor de weergaven van de genoemde grootheden kan worden gekozen uit twee opties:
\begin{itemize}
\item Voor het reactiecijfer en het referentiecijfer wordt gebruik gemaakt
      van een 7-segments display, de tijdsoverschrijding wordt aangegeven
      met een led en de waarden van de reactietijden, worden via (op de chip
      te ontwerpen) digitaal-analoog omzetters door draaispoelmeters aangegeven.
\item Alle grootheden worden getoond op een lcd-schermpje met een afmeting
      van 2 regels met elk 16 karakters.
\end{itemize}

\subsection{De Pong Spelcomputer}

\subsubsection{Inleiding}
De opdracht is om op een Sea-Of-Gates chip een spelcomputer te maken waarvan de uitgangen
kunnen worden aangesloten op de VGA-ingang van een monitor en waarmee een spel gespeeld
kan worden dat populair was in de begintijd van de spelcomputers: ''pong tennis''.
Zie de afbeelding hier\-onder.
\begin{figure}[h]
\centerline{\callpsfig{pongbeeld.eps}{width=.6\textwidth}}
\caption{Het beeld van de Pong spelcomputer}
\label{pongbeeld}
\end{figure}
Voor meer achtergrondinformatie over ''pong'', zie ook http://www.pong-story.com/intro.htm.

\subsubsection{Het spel}
Het idee van het spel is dat twee spelers hun racket omhoog en omlaag bewegen om een balletje
te laten terugkaatsen dat zich schuin over het speelveld beweegt.  Het balletje voert volkomen
elastische botsing\-en uit tegen de boven en onderwand.  Ook tegen de rackets worden volkomen
elastische botsing\-en uitgevoerd, echter met het verschil dat wanneer de bal het racket raakt op de
onderste helft de bal altijd terugkaatst naar beneden en wanneer de bal het racket raakt op de
bovenste helft de bal altijd terugkaatst naar boven. Elke keer wanneer het balletje botst tegen een
wand of racket is het geluid ''pong'' te horen. Wanneer een speler de bal mist en de bal achter hem
passeert krijgt de tegenstander 1 punt erbij.
Wanneer \'e\'en van de spelers 15 punten heeft, heeft deze speler het spel gewonnen.

\subsubsection{De spelbesturing}
Elke speler heeft twee drukknopen:
\'e\'en om het racket omhoog te laten gaan en
\'e\'en om het racket omlaag te laten gaan.
Verder zijn er:
\begin{itemize}
\item
een reset knop om de stand op 0 - 0 te zetten en het spel te starten.
\item
een schakelaar waarmee 2 verschillende balsnelheden kunnen worden ingesteld:
\begin{itemize}
\item
horizontaal: ongeveer 500 pixels/sec, verticaal: ongeveer 250 pixels/sec.
\item
horizontaal: ongeveer 1000 pixels/sec, verticaal: ongeveer 500 pixels/sec.
\end{itemize}
\item
een schakelaar waarmee 2 verschillende afmetingen van de rackets kunnen worden
ingesteld:
\begin{itemize}
\item
64 pixels verticaal
\item
32 pixels verticaal
\end{itemize}
\end{itemize}

\subsubsection{De aansturing van de monitor}
De chip dient een VGA-signaal te genereren voor een beeld van 640x480 pixels.
Een korte schemati\-sche beschrijving van deze standaard is te zien in
figuur \ref{vgatiming}.
Voor verdere informatie,
zie bijvoorbeeld http://www.epanorama.net/documents/pc/vga\_timing.html.
\begin{figure}[th]
\centerline{\callpsfig{VGAtiming.ps}{width=1.1\textwidth}}
\caption{VGA-timing}
\label{vgatiming}
\end{figure}
Merk op dat 5 uitgangssignalen nodig zijn voor een VGA-signaal: De waarde voor het rode pixel,
de waarde voor het groene pixel, de waarde voor het blauwe pixel, een horizontaal sync-signaal
en een verticaal sync-signaal.  De laatste twee signalen zijn digitale signalen van 0 of 5 Volt en
corresponderen dus met de op de Sea-Of-Gates chip gebruikte waarden voor een logische 0 en 1.
De RGB pixelwaarden kunnen in principe analoge waarden hebben tussen 0 en 0.7 Volt.
Voor ons spel zullen we de RGB pixels alleen helemaal aan of helemaal uitzetten
en zullen er 3 externe
level-shift schakelingen beschikbaar zijn waarme
de 5 Volt uitgangswaarde van de Sea-Of-Gates chip
kan worden omgezet naar 0.7 Volt.
Hoewel volgens de definitie van de 640x480 pixel mode VGA-standaard een klok van 25.175
MHz vereist is (om elke klokpuls van 39.7 nsec. een R, G en B pixelwaarde over sturen), blijkt
deze frequentiewaarde niet zo kritisch te zijn en zullen de meeste VGA-monitors nog werken
wanneer er een afwijking van 5 \% in de timing optreedt.
Merk verder op dat wanneer de resolutie
van de objecten in het spel niet te groot wordt gekozen en elke reeks van n (n  = 2, 4 of 8)
achtereenvolgende pixels dezelfde waarde hebben er verder ook volstaan kan worden met een n
keer zo lage klokfrequentie.
\FloatBarrier

\subsubsection{Het ''pong'' geluid}
Via een kleine luidspreker van 50 Ohm die direct op de chip wordt
aangesloten dient het ''pong'' geluid gegenereerd te worden
(een 1kHz blokgolf signaal).

\subsubsection{De overige randvoorwaarden}
Voor dit ontwerp mag een chip-oppervlak van 2 bond\_bars worden gebruikt.

\subsection{Alternatieve opdracht}
Naast bovengenoemde opdrachten kan een groep ook zelf een opdracht bedenken
en zelf de specificaties opstellen.
Enige suggesties voor een alternatieve opdracht zijn:
\begin{itemize}
\item
Een ander soort spelcomputer: bijvoorbeeld Snake, Tetris, Packman, etc.
\item
Een thermostaat.
\end{itemize}
Zo'n soort opdracht dient (1) aan dezelfde algemene randvoorden te voldoen
als de andere opdrachten (zie sectie 'Algemene Randvoorwaarden' hieronder),
en (2) eenzelfde complexiteit te hebben 
(d.w.z. een chip-oppervlak van 1 \`a 2 bond\_bars te omvatten).
Verder dient de opdracht eerst door de begeleiding te worden 
goedgekeurd voordat met de verdere uitwerking wordt begonnen.

\subsection{Algemene randvoorwaarden}

\begin{itemize}
\item
Voor de FSM's (Finite State Machine) mogen alleen die van het
Moore-type gebruikt worden. Meer informatie hierover is te vinden in
\cite{DT}.
\item
Als de schakeling geactiveerd wordt moeten alle FSM's in hun
begintoestand komen door middel van een reset signaal.
\item
Er moet bij het ontwerpen rekening gehouden worden met het feit dat de
schakeling later getest zal worden. Als minimum eis wordt gesteld dat
alle grote blokken apart getest moeten kunnen worden en dat van de
verschillende besturingen ook de huidige toestand gelezen moet kunnen
worden.
\item
Voor de opwekking van het kloksignaal kan gebruik gemaakt worden van
een kristal van 6.144 MHz of 32 kHz.
\item
Het streven is om zo weinig mogelijk componenten extern
te gebruiken. De dissipatie van de chip dient echter ook beperkt
te zijn. Dit geeft een compromis voor de maximale stroom die de elektronica
mag dissiperen voor de aansturing van de LEDs, etc.
\item
De voedingsspanning van het IC bedraagt 5 Volt.
\item
Het IC wordt gemaakt in een semi-custom CMOS proces waarvan de
specificaties in appendix \ref{SoGchip} vermeld zijn.
\item
Het beschikbare chip-oppervlak is circa 0.4 $\,cm^2$,
hetgeen overeenkomt met ongeveer 40.000 transistor-paren,
die gelijkelijk zijn verdeeld over 2 bond\_bars (zie bibliotheek cellen)
\item
Er zijn op het IC, naast de voedingspinnen, per bond\_bar 32 aansluitingen
beschikbaar.
Hoewel een totale schakeling meer dan \'e\'en bond\_bar aan oppervlakte
kan hebben, kunnen bij de afmontage slechts de aansluitingen
van \'e\'en bond\_bar worden aangesloten.
In uitzonderlijke gevallen is het mogelijk om het
aantal aansluitingen te verhogen d.m.v.\ multiplextechnieken.
\end{itemize}

\subsection{Aanpak systeemspecificatie}

In de middagen waarin de systeemspecificatie moet worden vastgesteld kunnen
een aantal taken worden onderscheiden die \'e\'en voor \'e\'en doorlopen moeten worden
om uiteindelijk tot een eenduidige systeemspecificatie te komen. Hieronder zijn deze taken opgesomd.
\begin{itemize}

\item    Randvoorwaarden vaststellen vanuit de opdrachtbeschrijving
(kan thuis worden voorbereid).
\item    Toevoegen van externe randvoorwaarden, complementeren
          van randvoorwaarden en specificaties.
\item    Korte beschrijving in woorden maken wat de schakeling moet doen.
\item    Verdeling van de schakeling in afzonderlijke deelsystemen.
\item    Beschrijving in woorden wat de verschillende deelsystemen
          moeten doen.
\item    Beschrijving van de interface (communicatie tussen de
          verschillende deelsystemen).
\item    Globale uitwerking van de deelsystemen. Vaststellen van de
          deelsysteemspecificaties en randvoorwaarden. Denk hierbij ook aan
          de testbaarheid.
\item    Onderlinge afstemming van de deelsystemen (naamgeving,
          timing, signalen, frequenties). Vaststellen welke signalen naar
buiten worden gevoerd (ook voor testdoeleinden). Zorg ervoor dat de verdeling zo is dat de deelsystemen apart testbaar zijn.
\item    Verdeling van de deelsystemen over de groep, maken van een
          tijdsplanning. Afspraken maken voor onderling overleg.
\end{itemize}

\subsection{Beschikbare ruimte}

Per groep is er op de chip een ruimte beschikbaar die gevormd wordt
door 2 bond\_bars onder elkaar.
Het ontwerp kan bestaan uit 1 geheel, waarbij er slechts 1 bond\_bar is aangesloten
en er dus 1 manier van afmonteren zal zijn, of
het ontwerp kan bestaan uit 2 afzonderlijke gedeelten die elk apart aan een
bond\_bar zijn aangesloten en er dus 2 manieren van afmonteren zullen zijn.
Wanneer er gekozen wordt voor 1 geheel zal deze schakeling als geheel moeten kunnen
functioneren, maar zal deze schakeling het ook moeten toelaten om de afzonderlijke
deelschakelingen te kunnen testen.
Wanneer er gekozen wordt voor 2 gedeelten zal normaal gesproken het eerste deel
als gehele schakeling functioneren en zullen op het twee deel de afzonderlijke
deelschakelingen nog eens aanwezig zijn om apart getest te kunnen worden.

\subsection{Testbaar ontwerpen}
\label{testbaar}
Nadat de schakelingen gefabriceerd zijn zullen zij door de groep getest worden. 
Het testen van de ontworpen schakeling na fabricage heeft een tweeledig doel: 
\begin{itemize}
\item
Nagaan of de schakeling aan de specificaties voldoet.
\item
Indien de schakeling niet aan de specificaties voldoet: 
proberen zo nauwkeurig mogelijk te weten te komen wat de reden is waardoor 
de schakeling niet voldoet,
zodat tijdens een eventueel herontwerp de schakeling wel zal kunnen werken.
\end{itemize}
Het is van groot belang dat al in een vroeg stadium tijdens het ontwerp
rekening wordt gehouden met het feit dat de schakeling later getest zal
worden.
Dit omdat in de eindfase van het ontwerp het meestal lastig is om de 
schakeling hiervoor nog aan te passen.
Daarom wordt in deze sectie besproken hoe het testen zal plaatsvinden
en aan welke eisen de schakeling (en de simulaties) moeten voldoen
om later op goede wijze te kunnen testen.

\subsubsection{De manier van testen bij het OP}
Het testen van de schakeling na fabricage wordt op de volgende 2 manieren uitgevoerd.  
In beide gevallen wordt gemeten aan 2 a 3 stuks afgemonteerde schakelingen in een DIL40 behuizing. 
\begin{itemize}
\item
De schakeling wordt aangesloten op een logic analyzer.
Via de logic analyzer worden logische signalen naar de ingangen van de
schakeling gestuurd en de logische signalen aan de uitgangen worden geobserveerd. 
Dit is vergelijkbaar met het simuleren van de schakeling tijdens het ontwerp,
maar nu met de echte hardware i.p.v. het (\smc{vhdl}) simulatie model.
Om te kunnen beoordelen of de schakeling werkt worden testvectors van 
een eerdere \smc{vhdl} simulatie gebruikt en wordt gekeken of de uitgangen 
van de werkelijke schakeling dezelfde resultaten geven als tijdens simulatie 
(aannemende dat de eerder \smc{vhdl} simulatie goede resultaten gaf).
Voor meer informatie over deze vorm van testen, zie appendix D van de OP handleiding.
De gehele schakeling dient op deze manier te worden getest, maar ook
de deelschakelingen dienen op deze
manier te worden getest.
\item
Aan de schakeling worden
de externe componenten aangesloten 
(schakelaars, oscillator kristal voor de klok, displays, luidsprekers etc) 
en de schakeling wordt real-time uitgeprobeerd.
\end{itemize}
\subsubsection{Overwegingen tijdens het ontwerp}
Vanwege de testbaarheid van het ontwerp en het eventuele localiseren van problemen, 
dient aan de volgende punten aandacht te worden geschonken tijdens het ontwerp:
\begin{itemize}
\item
Zorg dat de uitgangen van de schakeling, bij een gegeven set van ingangsignalen
en na de reset, altijd een bekende, vaste, reeks van waarden gaan doorlopen. 
Alleen dan namelijk kan de logic analyzer een vergelijking maken met de 
referentie set van uitgangsignalen zoals eerder verkregen via simulatie.
\item
Tijdens het testen met de logic analyzer (en ook tijdens simulatie !) 
kan slechts over een beperkt aantal klokcycli gesimuleerd worden: enkele 
miljoenen/honderdduizenden cycli tijdens simulatie en 65000 cycli tijdens 
testen met de logic analyzer. 
Dat wil dus zeggen dat wanneer een kloksignaal van 6.144 Mhz gebruikt wordt 
en men het gedrag van de schakeling na 1 minuut (real-time) wil bekijken 
er 60 x 6144000 is ongeveer 360 miljoen klokcycli nodig zouden zijn, wat dus veel te veel is.
Om dit probleem op te lossen kan men bijvoorbeeld oplossingen bedenken in de volgende richtingen:
(1) de schakeling een bepaalde mode geven waarin hij met een veel langzamere klok 
toch de gewenste toestanden doorloopt of 
(2) het mogelijk maken dat de schakeling van een bepaalde begintoestand start 
waarna de schakeling veel sneller de gewenste eindtoestand bereikt.
\item
Wanneer een schakeling niet goed werkt is aan alleen de uitgangssignalen 
vaak niet altijd te zien waar het probleem zit.  
Probeer daarom zoveel mogelijk ``interessante'' interne signalen op de pinnen van het 
IC aan te sluiten wanneer er nog pinnen beschikbaar zijn. 
Slecht een beperkt aantal signalen kan tijdens het testen bereikt worden worden: 
naast voedingsaansluitingen zijn er slechts 32 pinnen beschikbaar per schakeling voor in 
en uitgangen. 
Om te analyzeren wat er wel werkt en niet werkt en waar het probleem dus zit kan hier 
gedacht worden aan het demultiplexen en multiplexen van in en uitgangssignalen 
zodat via 1 pin meerdere ingangssignalen worden aangeboden of uitgangen worden gelezen. 
Dit kost echter weer wel extra pinnen voor het aansturen van de (de)multiplexers.
\end{itemize}
\subsubsection{Overwegingen tijdens het maken van testbenches}
Voor zowel de gehele schakeling als de verschillende deelschakelingen zal
minstens \'e\'en van de testbenches die gebruikt is tijdens simulatie, tevens 
(na conversie) gebruikt worden voor tijdens het testen met de logic analyzer.
Deze testbench zal aan de volgende eisen moeten voldoen:
\begin{itemize}
\item
Gebruik een simulatie die niet langer duurt dan 65000 klokcycli.
\item
Een ingangsignaal moet gedurende de hele simulatie ook ingangsignaal blijven. 
D.w.z. het is niet toegestaan dat voor bepaalde tijd een logische waarde opdrukt aan
een pin en de rest van de simulatie wordt het signaal ``losgelaten''.
\item
Bij de gespecificeerde ingangsignalen moeten na afloop van de reset altijd 
de gespecificeerde uitgangsignalen optreden 
(de uitgangsignalen mogen na de reset dus niet afhankelijk zijn van een 
toevallige willekeurige begintoestand waarin de schakeling zich bevindt).
\end{itemize}


\subsection{Afronding ontwerp}
Wanneer de uiteindelijke layout gemaakt gaat worden
dient allereerst aan de totale schakeling de bibliotheekcel osc10 aangesloten
te worden om het kloksignaal aan te sturen.
De aansturing van het kloksignaal dient plaats te vinden via terminal F van de osc10 cel.
De terminals XI en XO dienen als aansluitpinnen
voor het kristal naar buiten te worden gevoerd 
(dus aangesloten aan de bond\_bar) 
en de terminal E kan aan VDD worden vastgemaakt.
Van het geheel dient nu een layout gemaakt te worden zodanig dat
deze past op 1 of 2 bond\_bars (afhankelijk van hoe gekozen wordt de
beschikbare ruimte op de chip te gebruiken).
Het is vaak handig om deze laatste plaatsing van de deelschakelingen met de hand
te doen waarbij 1 of 2 bond\_bars op de achtergrond worden geplaatst als
referentiekader.
Vergeet dan echter niet om de bond\_bars na afloop weer te verwijderen.

De samenvoeging van de totale ontworpen layout met een bond\_bar dient te
gebeuren met het commando Add bond\_bar uit het Utility menu van \tool{GoWithTheFlow}.
Hierbij wordt gespecificeerd welke
terminals uit het ontwerp vast zitten aan welke IC pinnen.
Er wordt hierbij een nieuwe circuit cel gegenereerd bestaande uit 2 componenten
(de hierboven ontworpen layout en de bond\_bar) die dan vervolgens met 
Place \& route moet worden omgezet in een layout.
Plaats de ontworpen component hierbij met de hand over de bond\_bar
en route ze aan elkaar met \tool{trout} zonder de optie border terminals te gebruiken.

Om de kans op een werkende schakeling te vergroten is het
verder {\bf belangrijk
dat deze uiteindelijke layout (inclusief bond\_bar)
altijd als laatste nog eens geverifieerd wordt door er een switch-level
simulatie voor uit te voeren}.
Deze switch-level simulatie kan bijvoorbeeld gebaseerd zijn op de testbench
die later ook gebruikt zal worden voor het testen met de logic analyzer.

Maak tenslotte van de totale layout, inclusief bond\_bar, een ldm file
met het commando Make ldm.

De files die dan bij de beleiding moeten worden ingeleverd zijn:
\begin{itemize}
\item
De \smc{ldm} file voor de totale layout,
aangemaakt met het commando Make ldm
\item
De bijbehorende .buf file
met daarin de informatie over hoe de pinnen zijn aangesloten.
(Deze file wordt automatisch gegenereerd
tijdens het uitvoeren van Utilities $\rightarrow$ Add bond\_bar.)
\item
Een .ref file voor elke test die na fabricage verricht zal
gaan worden voor de schakeling.
(Een .ref file
wordt aangemaakt uit een .lst file met behulp van
het commando Generate $\rightarrow$ Reference file.)
Het is gewenst om een voor elke deelschakeling een .ref file 
te hebben en een .ref voor de totale schakeling.
\end{itemize}
\cleardoublepage

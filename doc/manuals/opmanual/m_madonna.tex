\selectlanguage{british}
\title{Automatic Placement with Madonna}
\maketitle
\label{madonna}
\index{madonna@\tool{madonna}|(bold}
\index{automatic!placement|see{\tool{madonna}}}
The job of placing the instances can be performed by an automatic tool which
was called after one of the most famous singers of the 1980'ies\footnote{And
90'ies.}. She places the instances such that the total wire length is
(hopefully) minimized. As we all know, \tool{madonna} is rather unpretentious;
If you like, you can improve her results manually using the other commands in
the instance menu. A circuit which was placed using \tool{madonna} can be routed
automatically, since \tool{madonna} makes sure that all instances are present in the
layout.

\section{What is required before I can call \protect\button{Madonna}?}
\tool{Madonna} is very sensitive to errors in her input.
To prevent offending her the following data must be present:
\begin{itemize}
\item
A proper circuit description in the database. \tool{Seadali}
complains if a circuit description with the specified name
doesn't exist.
Do not forget to convert your
\tool{sls} network description into the database using \tool{csls}.
\item
For each of the son-cells (instances) in the circuit description
a layout must exist which has the same name. We made sure that
this is always the case for all library cells.
\end{itemize}
If there is an error in the netlist description \tool{madonna} opens a window to
explain what you did wrong.

\section{Running \protect\tool{madonna}} \label{s-run-madonna}
Just like \tool{fish}, \tool{madonna} can be called both
from \tool{seadali} and the command shell. 

\subsection{From \protect\tool{seadali}}
Calling her
by pressing the button \button{Madonna} in the instances
menu is the most convenient for you.
Pressing this button will overwrite the design which
your are currently editing in \tool{seadali}. Therefore you will be
asked whether you are sure to continue in case the workspace
is not empty.

As next step you must enter the name of the circuit which
has to be placed. A small menu will appear, in which you
just click \button{DO IT !} to start \tool{madonna}.
In the menu some optional features of can be set for \tool{madonna}:
\begin{description}
\item[\button{set box}]
\index{madonna!set placement box|bold}
Set the box within which \tool{madonna} has to place the instances.  Click the
right-top corner. In this way you can control the shape and the size of the
placement. Specifying a big box will leave quite some unused transistors.
Specifying a very small box will squeeze the instances on minimum area. If the
box is too small, \tool{madonna} will expand it in either the x- or the y-direction,
depending on what you specified. It is better not no make the box much to
small, because \tool{madonna} will have a hard time recovering from that.

If the workspace was empty, \tool{seadali} will make an empty image array
before you can specify the box. In this way you have some reference for the
size of the box.
\item[\button{X-expand}]
Force \tool{madonna} to expand horizontally if the box was too small.
This is default.
\item[\button{Y-expand}]
Expand vertically if the box was too small.
\item[\button{Channels}]
\index{madonna!channels}
\index{channels}
Create a placement with routing channels. If this feature is active,
\tool{madonna} computes the amount of space that \tool{trout} needs 
to complete the routing process successfully. This computation actually
involves a call to
\tool{madonna}'s internal global router. A report of the global routing
process is written to the file \file{seadif/groutes}.

Note that the routing channels increase the layout area in both the X and the Y
direction, regardless of the preferences indicated by means of the
\button{X/Y-expand} buttons or the \button{set box} button.
\item[\button{Options}]
Set any other (unofficial) option which you want to propagate to
\tool{madonna}. Don't touch this button unless you know what you are
doing. 
\item[\button{DO IT!}]
Start \tool{madonna}. This may take anywhere from 30 seconds up to
10 minutes, depending
on the number of instances and the system load. You can kill \tool{madonna}
using the \button{KILL}-button.
\end{description}
You can try to place the circuit with various shapes and sizes.

\attention{Can I pre-place some instances?}
{No, \tool{madonna} is quite possessive. She wants to do everything herself.
She will always place all instances in your circuit description.
Pressing \button{Madonna} a second time will overwrite the
existing placement.
You can only request her to place the instances in a certain box.
Obviously, you can modify \tool{madonna}'s placement manually using the
buttons in the instance menu.}

\subsection{Calling \protect\tool{madonna} from the command line using \protect\tool{sea}}
\label{m-sea}
\index{sea@\tool{sea}}
The non-interactive way to call her is by using the tool \tool{sea} from the
command line: \type{sea -p name} in which {\sl name} is the name of the
circuit cell.
As a result,
the output placement will be written into the layout database.
The program \tool{sea} is a kind of bodyguard to \tool{madonna},
it checks the correctness of your input.
Only if everything is correct,
it calls \tool{madonna} to look at your circuit. 

\tool{Sea} also allows you to place and route a circuit in one step:
\index{automatic!placement and routing in one step|bold}
\type{sea hotelLogic}
for instance, will place and route the cell {\tt hotelLogic} and write the
result into the layout database under the name {\tt hotelLogic} (use the '-o
name' option to write it under a different name).
Just type \type{sea -h} to see all the options.  

\index{madonna@\tool{madonna}|)}
\cleardoublepage

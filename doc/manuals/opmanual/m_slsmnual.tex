\selectlanguage{dutch}
\title{Handleiding SLS}
\maketitle
\label{mansls}
\index{sls|(bold}
\section{Inleiding}

SLS is een afkorting van Switch Level Simulator en is een simulator welke 
gebruikt kan worden voor het simuleren van mos-circuits ter bepaling van
het logische gedrag en timing eigenschappen. De simulator modelleert de
mos-transistoren door geaarde capaciteiten en geschakelde weerstanden. Alle
knooppunten in het netwerk krijgen na simulatie een logische toestand 0, 1 of
X (ongedefinieerd) toegewezen.\\
Het SLS-pakket omvat 2 programma's: \tool{csls} en \tool{sls}. \tool{csls} wordt
gebruikt om een netwerk aan de circuit-database toe te voegen, \tool{sls} is 
de eigenlijke simulator. Het gebruik van beide programma's wordt hieronder
toegelicht.
\index{csls|(bold}
\section{Het gebruik van \tool{csls} en de sls-taal}
Met \tool{csls} wordt een netwerk aan de circuit-database als volgt toegevoegd:
\type{\tool{csls} $<$filenaam$>$}
Het netwerk is in de database in een aantal binaire files opgeslagen.
(Om van dit interne database-formaat weer een netwerk sls-beschrijving
te krijgen moet men het programma \tool{xsls} gebruiken.)
De $<$filenaam$>$ is een tekstfile waarin het toe te voegen netwerk in de
sls-taal beschreven staat. De $<$filenaam$>$ heeft bij voorkeur de naam van
het netwerk met de extensie .sls. De beschrijving bestaat uit twee gedeelten. Eerst \index{externe netwerkdeclaratie}
komt er een declaratie van externe netwerken. Dit zijn netwerken die reeds in
de circuit-database aanwezig zijn en (eventueel) in het nieuwe netwerk als
subcircuit aangeroepen wordt. Het tweede deel bevat de eigenlijke 
netwerkbeschrijving.
De externe declaratie is verplicht. Alle netwerken die in het nieuwe netwerk
gebruikt worden moeten als extern netwerk gedeclareerd zijn.
\paragraph{Externe netwerkdeclaratie} De externe netwerkdeclaratie begint met 
de sleutelwoorden {\bf extern network}, gevolgd door de naam van het externe 
netwerk. Dan tussen ronde haakjes het sleutelwoord {\bf terminal} en alle 
externe aansluitingen gescheiden door komma's. 
De volgorde van deze aansluitingen hoeft niet gelijk
te zijn aan de volgorde zoals deze was bij definitie van het netwerk. 
Wel legt het de volgorde vast van de aansluitingen bij de aanroep van het 
netwerk in het nieuw te beschrijven netwerk. Alle externe netwerkdeclaraties 
van de bibliotheekcellen worden bij het maken van een project 
met \tool{mkopr} \index{mkopr} in een file \fname{oplib.ext} in
de project-subdirectory \fname{sls\_prototypes} gezet.
Deze file kan als header (d.m.v.\ een \fname{\#include} statement)
in een sls-file worden opgenomen.
\paragraph{Netwerkbeschrijving} De netwerkbeschrijving begint met het 
sleutelwoord {\bf network} gevolgd door de naam van het netwerk. Deze naam
mag niet beginnen met een hoofdletter. Dit om verwarring te voorkomen
met namen van netwerken die ontstaan uit extractie. 
Achter de naam van het netwerk
komen tussen ronde haakjes het sleutelwoord {\bf terminal} en alle externe 
aansluitingen gescheiden door komma's. 
De laatste twee aansluitingen moeten altijd vss, vdd zijn.
Dan volgt een regel met accolade-open (\{), gevolgd door een of meerdere 
regels met alle interne en externe subnetwerken waaruit het netwerk 
is opgebouwd,
afgesloten met een punt-komma (;). De beschrijving eindigt met een 
accolade-sluiten (\}).\\
De externe subnetwerken mogen vooraf gegaan worden met tussen accolades een
specifieke naam voor dat subnetwerk. Dit kan handig zijn indien men onderscheid
wil maken tussen subnetwerken die in de beschrijving meerdere keren voorkomen.
Commentaar mag op elke plaats worden toegevoegd en moet staan tussen /* en */.\\
De interne netwerken die gebruikt kunnen worden zijn:

\begin{tabular}{lll}
Naam & parameters & toelichting\\
\\
cap & {\it waarde} (node1,node2) & Capaciteit ter grootte {\it waarde} in Farad\\ \index{cap}
res & {\it waarde} (node1,node2) & Weerstand ter grootte {\it waarde} in $\Omega$\\ \index{res}
nenh & w={\it breedte} l={\it lengte} (gate,drain,source) & enhancement nmos transistor\\ \index{nenh}
&                                of   (gate,source,drain) & De breedte en lengte parameters\\
&& zijn in meters en optioneel\\
penh & w={\it breedte} l={\it lengte} (gate,drain,source) & enhancement pmos transistor\\ \index{penh}
&                                of   (gate,source,drain) & De breedte en lengte parameters\\
&& zijn in meters en optioneel\\
ndep & w={\it breedte} l={\it lengte} (gate,drain,source) & depletion nmos transistor\\
&                                of   (gate,source,drain) & De breedte en lengte parameters\\
&& zijn in meters en optioneel\\
net & \{node1,node2\} & kortsluiting tussen node1 en node2\\\index{net}
\\
\end{tabular} 

Naast deze ingebouwde interne netwerken kunnen een aantal logische functies
als intern netwerk worden aangeroepen.
Deze functies dienen vooraf gegaan te worden door het {\bf @} karakter.
De volgende logische functies zijn mogelijk:\\
\begin{tabular}{lll}
\\
Naam & parameters & toelichting\\
\\
@ invert & (A, Y) & invertor met ingang A en uitgang Y\\
@ nand & ($A_1, A_2,...A_n, Y$) & n-input nand met uitgang Y\\
@ nor & ($A_1, A_2,...A_n, Y$) & n-input nor met uitgang Y\\
@ and & ($A_1, A_2,...A_n, Y$) & n-input and met uitgang Y\\
@ or & ($A_1, A_2,...A_n, Y$) & n-input or met uitgang Y\\
@ exor & ($A_1, A_2,...A_n, Y$) & n-input exclusive or met uitgang Y\\
\end{tabular}
\\
\\
\\
{\bf NB:} De terminal- , netwerk-namen moeten voldoen aan
de volgende voorwaarden:
\begin{itemize}
\item
Een naam moet beginnen met een {\it letter} en mag verder alleen uit
{\it letters}, {\it cijfers} en {\it underscores (\_)} bestaan.
\item
De lengte van een naam mag niet groter zijn dan 14 karakters.
\end{itemize}
\clearpage

\paragraph{Voorbeelden} Hieronder volgend twee voorbeelden van netwerkbeschrijvingen
in de sls-taal.\\

Voorbeeld 1 :\\
\\

\begin{figure}[thb]
\centerline{\callpsfig{no210.cir.eps}{width=0.35\textwidth}}
\caption{\label{nor} Transistornetwerk van een 2-input nor}
\end{figure}

\begin{verbatim}

/* Netwerkbeschrijving van een 2-input nor nor2
   Dit voorbeeld heeft geen externe netwerk aanroep
   omdat alleen gebruik gemaakt wordt van interne netwerken */

network nor2 (terminal A, B, Y, vss, vdd)
{
    nenh l=1.6u w=23.2u (A, vss, Y);
    nenh l=1.6u w=23.2u (B, vss, Y);
    penh l=1.6u w=29.6u (B, 1, Y);
    penh l=1.6u w=29.6u (A, 1, vdd); 
}

\end{verbatim}

\clearpage

Voorbeeld 2 :\\
\\

\begin{figure}[thb]
\centerline{\callpsfig{latch.eps}{width=0.5\textwidth}}
\caption{\label{latch} SR-latch met 1 pF belasting aan beide uitgangen}
\end{figure}


\begin{verbatim}

/* Netwerkbeschrijving van een SR-latch met 1 pF belasting 
   aan beide uitgangen */

extern network nor2 (terminal A, B, Y, vss, vdd)

network sr_latch (terminal S, R, Y, YN, vss, vdd)
{
    {nor1} nor2 (S, Y, YN, vss, vdd);
    {nor2} nor2 (R, YN, Y, vss, vdd);
    cap 1p (Y, vss);
    cap 1p (YN, vdd);
}
\end{verbatim}
\index{csls|)}
\clearpage

\section{Gebruik van \tool{sls} en de commandfile}

De simulator \tool{sls} kan zowel met \tool{simeye} opgestart worden als vanaf
de commandline. De aanroep van de commandline is als volgt:
\type{\tool{sls} $<$netwerknaam$>$ $<$commandfile$>$}
$<$netwerknaam$>$ is de naam van het te simuleren netwerk, $<$commandfile$>$ is de stimuli-file
met bij voorkeur de extensie .cmd.\\
Voor het gebruik van \tool{sls} in \tool{simeye} wordt verwezen naar de handleiding van
\tool{simeye}.\\
In de commandfile worden de stimuli beschreven van de inputsignalen. Tevens kunnen
een aantal opties meegegeven worden voor zowel \tool{sls} als voor \tool{spice}. Dit laatste \index{spice}
kan belangrijk zijn indien het netwerk (b.v.\ vanuit \tool{simeye})
met \tool{spice} gesimuleerd wordt.\\
De structuur van de commandfile is als volgt: \index{commandfile}
\begin{quote}
{\tt
{\it inputstimuli en sls-opties}\\
/$\ast$\\
$\ast$\%\\
{\it Transi\"ent-informatie voor spice}\\
$\ast$\%\\
{\it Directe spice-commando's}\\
$\ast$\%\\
$\ast$/\\
}
\end{quote}

\paragraph{Inputstimuli}
Alle inputsignalen worden gezet met het set-commando:
\begin{quote}
\begin{verbatim}
set A = (l*2 h*2)*2 l*~
\end{verbatim}
\end{quote}
Inputsignaal A is nu twee tijdstappen laag, 2 tijdstappen hoog, dan weer tweemaal laag en
tweemaal hoog, en dan voor altijd laag.
\paragraph{Sls-opties}
Hieronder volgen een paar van de belangrijkste opties. Voor overige opties en instellingen
wordt verwezen naar de "Switch Level Simulator User's Manual"
\index{commandfile!opties}
\begin{itemize}
\item
option level = 1/2/3 (default is 1)
\begin{description}
\item[-] level = 1 : Puur logische simulatie zonder de circuitparameters te betrekken.
\item[-] level = 2 : Logische simulatie gebaseerd op circuitparameters zoals de 
transistor-afmetingen, weerstandwaarden en capaciteitwaarden.
\item[-] level = 3 : Logische simulatie en bepaling van vertragingstijden gebaseerd 
op circuitparameters zoals de transistor-afmetingen, weerstanden en capaciteiten.
\end{description}
\item
option simperiod = n (default n is oneindig)\\
Specificeert het aantal simulatietijdstappen.
\item
option sigunit = getal (default getal is 1)\\
Specificeert de duur van een tijdstap in seconden.
\item
option outunit = macht\_van\_tien (default macht\_van\_tien is macht van 10 het dichtst bij sigunit. Macht\_van\_tien wordt aangeduid met de 
achtervoegsels f,p,n,u,m,k,M,G.
Bijvoorbeeld: 10p, 1u.\\
Specificeert de tijdseenheid welke gebruikt wordt bij de uitvoer.
\item
option outacc = macht\_van\_tien (default macht\_van\_tien is outunit)\\
Specificeert de nauwkeurigheid van de simulatie-uitvoer (outacc $<=$ outunit).
\item
option sta\_file is = on\\
Deze regel specificeert of de file {\it $<$cel$>$.sta} moet worden ingelezen.
Deze file wordt bij de automatische circuitgeneratie vanuit een toestandsdiagram door \tool{kissis} aangemaakt en bevat de definitie van de \index{kissis}
toestandsvariabele State.
\item
option initialize (full) random = on\\
Soms is het nodig een schakeling te initialiseren in een bepaalde begintoestand,
b.v.\ als flipflops zonder reset gebruikt worden.
Dit voorkomt dat sommige knooppunten ongedefinieerd (X) blijven.
Met 'full random' wordt de schakeling iedere keer anders ge{\ii}nitialiseerd.
\item
define XXXX (XXXX is een naam voor een nieuw te defini\"eren variabele)\\
Met dit commando kunnen nieuwe variabelen gedefinieerd worden die
een combinatie zijn van ingangs- of uitgangssignalen. Dit commando
wordt o.a.\ gebruikt bij de definitie van de toestandsvariabele State
bij de automatische circuitgeneratie vanuit een toestandsdiagram
 door \tool{kissis}. State is dan een combinatie van alle flipflop-uitgangen
die de toestand coderen.
\item
print A,b,..  (A,b,.. zijn inputsignalen en/of outputsignalen)\\
Deze regel specificeert welke inputsignalen en/of outputsignalen 
in de sls-uitvoer komen. Dus ook welke signalen bij \tool{simeye} 
op het scherm getekend gaan worden.
\item \index{spice|(}
plot A,b,.. (A,b,.. zijn inputsignalen en/of outputsignalen)\\
Deze regel specificeert welke inputsignalen en/of outputsignalen 
in de spice-uitvoer komen. Dus ook welke signalen bij \tool{simeye} 
op het scherm getekend gaan worden.
\end{itemize}
\paragraph{Transi\"ent-informatie voor spice}
\index{commandfile!spice}
In dit gedeelte kunnen een aantal transi\"entparameters voor \tool{spice} gezet worden:\\
\\
\begin{tabular}{lll}
tstart & getal & starttijd voor de output van de transi\"ent-analyse in seconden \\
tstep & getal & stapgrootte van de transi\"ent-analyse in seconden\\
tfall & getal & daaltijd van de inputsignalen in seconden\\
trise & getal & stijgtijd van de inputsignalen in seconden\\
uic && "use initial conditions"\\
vhigh & getal & spanningsniveau van het hoogsignaal in Volts\\
vlow  & getal & spanningsniveau van het laagsignaal in Volts\\
\end{tabular}

\paragraph{Directe spice-commando's}
In dit gedeelte kunnen commando's worden opgegeven die letterlijk aan de spice-invoerfile
worden toegevoegd zoals:
\begin{quote}
\begin{verbatim}
.options cptime=100
.ic v("3")=0 v("out")=5
\end{verbatim}
\end{quote}
Voor informatie over deze en andere mogelijke commando's wordt u verwezen naar de
desbetreffende spice-handleidingen.\\
\\
\\
{\bf Voorbeeld commandfile sr\_latch.cmd} :
\index{commandfile!voorbeeld}
\begin{verbatim}
set S = l*2 h*2 l*6 h*2 l*8 
set R = l*6 h*2 l*6 h*2 l*2
set vss = l*~
set vdd = h*~
print S ,R, Y, YN
plot  S, R, Y, YN
option level = 3
option sigunit = 1e-8
option simperiod = 20
option outacc = 10n
/*
*%
tfall  0.5n
trise  0.5n
tstep  0.1n
*%
.options cptime = 500
*%
*/
\end{verbatim}
\index{spice|)}
\index{sls|)}
\cleardoublepage













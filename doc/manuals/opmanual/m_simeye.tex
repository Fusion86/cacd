\selectlanguage{british}
\title{Manual Simeye} \index{simeye|(bold}
\maketitle

\section{Synopsys}
{\bf simeye}
[X options]  [-IRLeh]  [-l$|$-m$|$-c]  [-a$|$-b]  [name]

\section{Options}
Beside the standard X Toolkit options (e.g. to set the display,
the geometry, and the foreground color of the program),
the following other options may be specified:
\begin{description}
\item[{\bf -freestateColor {\it color} : }]
Specifies the color for signal values that are in {\bf free state}.
\item[{\bf -high\_lowColor {\it color} : }]
Specifies the color for signal values that are {\bf logic O} or {\bf logic I}.
\item[{\bf -undefinedColor {\it color} : }]
Specifies the color for signal values that are {\bf logic X}.
\item[{\bf -voltsColor {\it color} : }]
Specifies the color for {\bf analog} signal values.
\item[{\bf -pointerColor {\it color} : }]
Specifies another color for the pointer (default: canvas.foreground).
\item[{\bf -std\_bg {\it color} : }]
Specifies another color for the standard background (default: grey).
\item[{\bf -I} :]
Start \tool{simeye} in input (editing) mode.
\item[{\bf -R} :]
Start \tool{simeye} in input (editing) mode
and set circuit name and command file name
in order to prepare a simulation run.
\item[{\bf -L} :]
Load window settings each time when new signals are read.
\item[{\bf -e} :]
Do not add or use extension of specified {\it name}.
\item[{\bf -h} :]
Display only help information.
\item[{\bf -l} :]
Use {\bf sls-logic} as simulation type (default).
\item[{\bf -m} :]
Use {\bf sls-timing} as simulation type.
\item[{\bf -c} :]
Use {\bf spice} as simulation type.
\item[{\bf -a} :]
Use {\bf analog} representation for {\bf sls} signals.
\item[{\bf -b} :]
Use {\bf digital} representation for {\bf sls} signals 
(default).
\end{description}

\section{Description}
\tool{Simeye}
is an X-window (OSF/Motif) based simulation user interface
for viewing the results of the
\tool{sls(1ICD)},and
\tool{spice(1ICD)}
simulators.
The program can be used to inspect the graphical representation of the
output signals of the simulators and
it can be used to edit input signals that
are described in the SLS command file format (see: User's Manual).
Moreover,
from 
\tool{simeye}
simulations can be run by starting either the
\tool{sls} or \tool{spice} simulator
(Note: \tool{spice} via the script \tool{nspice(1ICD)}).
\par
{\bf The following program argument may be specified:}
\begin{description}
\item[{\it name} :]
This name specifies the name of the cell or file for
which simulation signals are displayed and/or edited.
\end{description}
The {\it name} argument should be a file name.
If the {\it name} contains no '.' in it,
the name is assumed to be a cell name.
In that case a standard extension is added, based on the specified
option (precedence order: {\bf -I}, {\bf -c}, {\bf -a}).
Give option {\bf -e}, if no extension must be added.
If a file name is specified with a standard extension (from first '.'
position), this extension type is used (has precedence of options).
Give option {\bf -e}, if you do not want this feature.
\\
\\
The standard extensions are:
\\
''.res''
for {\bf sls digital} (output) signals (option {\bf -b}).
\\
''.plt''
for {\bf sls analog} or waveform (output) signals (option {\bf -a}.
\\
''.ana''
for {\bf spice} analog (output) signals (option {\bf -c}).
\\
''.cmd''
for simulator {\bf commands} and input signals (option {\bf -I}).
\\
\\
Note that the spice output file should contain the signals
of a transient analysis in tabular format.
When running the program \tool{spice}
directly this is achieved by
using the card ''.print tran ...'' in the spice input file.
When running \tool{nspice},
this is achieved by using
sls ''plot ...'' commands in the command file.

\section{Commands}
Commands are organized in six PullDown Menu's (i.e. {\it File},
{\it Edit}, {\it Simulate}, {\it View}, {\it Options} and {\it Help}).
This MenuBar buttons can also be activated by typing the $<${\bf Meta}$>$ key
and the mnemonic character (the mnemonic
is the character that is underlined).
Commands in the PullDown Menu's can also be activated by only typing
a mnemonic character (Note: This mnemonic can be turned off,
if you are using accelerator keys).
\\
Under the MenuBar there is a StatusArea,
which shows the name
of the ''editing'' command file or ''viewing'' simulation results file,
and the simulation setup, and most right the active command (if any).
You can also turn off an active command by clicking on that field.\\
\\
{\bf The commands in the {\it File} PullDown Menu are:}
\begin{description}
\item [{\it New :}]
Start editing of a completely new command file or cell ''cmd'' stream
after an OK-confirmation has been given.
\item [{\it Open :}]
Read {\it cell} command file or stream for editing.
If this {\it cell} or this command file does not exist,
\tool{simeye}
will try to read a default command file called ''simeye.def.d''.
First, it tries to read this file from the current directory
and second it tries to read this file from the process directory.
\item [{\it Save :}]
Save (update) the edited command file or stream under the same file
or cell name after an OK-confirmation has been given.
\item [{\it Save As :}]
Save the edited command file {\it cell} under a
different file or cell name after an OK-confirmation has been given.
\item [{\it Read :}]
Read a new {\it cell} signal (simulation results) file or stream.
If the 'read' {\bf Digital} or {\bf Analog} setting is activated
an \tool{sls} ''res'' or ''plt'' output file/stream is read.
If the 'read' {\bf Spice} setting is activated
the \tool{spice} ''ana'' output file is read.
\item [{\it Read Append :}]
Append the {\it cell} signal (simulation results) file/stream to the current
signals.
Only Digital signals can be appended to Digital signals
and Analog signals to Analog signals (see {\it Read}).
\item [{\it Hardcopy :}]
Generate a hardcopy of the current screen and send it to the printer
by making an X Window pixel dump (fast, but of low resolution).
\item [{\it Print :}]
Make a PostScript-file of the current screen and send it to the printer
after an OK-confirmation has been given (slow, but of high resolution).
\item [{\it Exit :}]
Exit the program after an OK-confirmation has been given.
\end{description}

{\bf The commands in the {\it Edit} PullDown Menu are:}
\begin{description}
\item [{\it Add :}]
Create a new signal. The user has to enter the name of the new
signal, and the signal will be placed at the bottom of
the window.
\item [{\it Change :}]
Change (edit) a particular signal by inserting a new logical level
for a particular time interval (t1, t2).
The signal description may eventually already be defined for this interval.
The insertion of the new logical level is done in two steps:
During the first click of the mouse
the signal, the new logical level and t1 are selected.
During the second click t2 is selected and the new signal
description is drawn.
To insert an interval that has a free state 
one should hold the $<$Shift$>$ key pressed down
while making the first selection.
\item [{\it Delete :}]
Delete one or more signals by selecting them with the cursor.
\item [{\it EraseAll :}]
Erase all signals after an OK-confirmation has been given.
\item [{\it Move :}]
With this command the order of the signals can be changed
or one signal can be placed over another signal
in order to compare them.
First the signal that is moved or that is placed over another signal
is selected,
and then the new position of the signal is selected.
To place the selected signal over another signal one should hold
the $<$Shift$>$ key down while selecting the new position.
To remove a signal that is placed over another signal,
initially select the signal with the $<$Shift$>$ key held down.
\item [{\it Copy :}]
Copy the signal description from one signal to another signal.
\item [{\it ReName :}]
Change or show the (full) name of the selected signal.
\item [{\it Yank :}]
Store a (part of a) signal description in the buffer.
\item [{\it Put :}]
Insert a copy of the signal description that is in the buffer onto
a particular position.
The user has to select the signal to which the contents
of the buffer is added and the time from which on the
new signal description is valid.
The new signal description may (partly) override the existing
signal description for the selected signal.
Furthermore,
the user is asked to type how may periods of the stored signal part
are added.
If a value -1 is specified, the selected signal
will become a periodical signal and (seemingly)
an infinite number of periods are added.
In that case, when the simulation end time is enlarged,
the signal description will automatically
be extended according to the periodicity of the signal description.
A periodical signal is indicated by means of a tilde ($\sim$)
immediately to the right of the signal description.
\item [{\it Grid :}]
Specify the smallest unit for the x-axis (= time axis).
\item [{\it Speed :}]
Speed up the signals by some factor.
If a value $<$ 1 is specified, the signals will be slowed down.
\item [{\it T\_end :}]
Update the end time of the input signals (= end time of simulation).
\end{description}

{\bf The commands in the {\it Simulate} PullDown Menu are:}
\begin{description}
\item [{\it Run :}]
Start the simulation run as set in Prepare menu.
A simulation may be aborted by typing $<$Control$>$C
in the \tool{simeye} window
(place the cursor in the \tool{simeye} window and
type the character C while holding down the $<$Control$>$ key).
\item [{\it Edit :}]
Edit the current command file set in the Prepare menu.
\item [{\it Prepare :}]
Prepare a simulation by using either the \tool{sls} or the \tool{spice}
simulator
after an OK-confirmation has been given.
\\
When 'type' {\bf spice} is set a \tool{spice} simulation is performed.
The \tool{spice} simulator is run by calling the program \tool{nspice}. 
After simulation the 'ana' \tool{spice} output file is read.
\\
When 'type' {\bf sls-logic} (level 1 or 2) or {\bf sls-timing} (level 3) is set,
an \tool{sls} simulation at that level is performed.
When the \tool{sls}
simulator is run, also intermediate simulation
results are displayed.
At the end
the final simulation results
are displayed.
Which results are displayed for \tool{sls} ('res' or 'plt') depends on
the 'read' {\bf Digital} or {\bf Analog} setting.
\item [{\it Command :}]
Display the current simulation command.
\end{description}

{\bf The commands in the {\it View} PullDown Menu are:}
\begin{description}
\item [{\it Measure :}]
Move a vertical scanline over the display window,
according to the position of the pointer,
and show the value of the corresponding x position (and y position).
When clicking the middle (or right) button of the mouse, one
may toggle between displaying
in the right margin (if the number of signals is not too large)
the y values of the signals at the selected x position.
To store the value of a particular position, click the left 
button of the mouse.
This position will then be subtracted from
the next positions of the scanline,
so as to measure delay times and/or voltage differences.
\item [{\it Redraw :}]
Redraw the current window.
\item [{\it Values :}]
Display the value points (for analog simulation results).
\item [{\it Full :}]
Draw all present signals from the beginning of the simulation
time till the end of the simulation time.
\item [{\it ZoomIn :}]
Zoom in on the current window.
When the $<$Shift$>$ key is held down and the window contains only
one analog signal,
also a zoom-in on the y-axis of the signal
is performed.
The ScrollBars can be used to
move the current window left-, right-, up- or down-wards
respectively.
\item [{\it ZoomOut :}]
Zoom out on the current window.
Use zoom out if you want to add new signal states ({\it Change} command)
to signals on the right side of the screen.
\item [{\it Load Window :}]
Load the current window settings (i.e. start-time, stop-time
and the names of the signals that are displayed)
from a file (default: ''simeye.set'').
The name of this file can be specified in the configuration file.
\item [{\it Save Window :}]
Save the current window settings (i.e. start-time, stop-time
and the names of the signals that are displayed)
in a file (default: ''simeye.set'').
The name of this file can be specified in the configuration file.
\end{description}

{\bf The commands in the {\it Options} PullDown Menu are:}
\begin{description}
\item [{\it DetailZoomON :}]
When the DetailZoomON option is set (see also DETAIL\_ZOOM\_ON)
the zoom-in function is also defined for the y-axis of a (single analog) signal.
\item [{\it AutoLoadWin :}]
When the AutoLoadWin option is set (or by the command line option {\bf -L}),
the program \tool{simeye} loads the saved window settings each time
a {\it Read} or {\it Simulate} command is performed.
See also the {\it Load Window} and {\it Save Window} commands.
\item [{\it AutoSaveEdit :}]
When the AutoSaveEdit option is set
the program \tool{simeye} saves automatically the command file (if changed)
when a {\it Run} command is given.
\end{description}

{\bf The working of the {\it Help} MenuBar facility:}
\begin{description}
\item [{\it Manual :}]
When the Manual command is clicked,
the program \tool{icdman} with the manual page of \tool{simeye} is popped up in a window.
Type 'q' to quit this program.
\end{description}

\section{Commandfile}
For simulation with both \tool{sls} and \tool{spice},
\tool{simeye} uses a command file called '{\it cell}.cmd'
This file should contain a description
of the input signals, values for the simulation control variables,
and a listing of the terminals for which output should be generated.
This is explained in more detail in the users manuals of
the \tool{sls} simulator and the \tool{spice} simulator.
\\
The 'set' commands (signals) in this file/stream
may be edited by enabling the {\it Edit} PullDown Menu of
\tool{simeye}.
The new signal descriptions will then be written back
to the command file when updating
the command file.
However,
when the file contains a 'set' command that is followed by
the keyword 'no\_edit' between comment signs,
the signal description of this 'set' command can not be modified.
Additionally,
when the command file contains on a separate line
between comment signs the keyword 'auto\_set',
\tool{simeye}
will automatically turn-on or turn-off these 'non-editable' 
'set' commands according to whether or not
the node the command refers to is part of the terminal list
of the cell that is simulated (by {\it Simulate} command).
This feature may for example be useful to specify default signals for
one or more input terminals.
\\
When the file contains on a separate line
between comment signs the keyword 'auto\_print' ('auto\_plot'),
\tool{simeye}
will automatically add a 'print' ('plot) statement to the
command file for each terminal of the cell that is simulated.
The new 'print' ('plot') commands will have a keyword 'auto' between
comment signs following the keyword 'print' ('plot').
\\
An example of a default command file that can be used for
\tool{sls} and \tool{spice} simulations is:
\begin{verbatim}
/* auto_set */
set /* no_edit */ vdd = h*~
set /* no_edit */ vss = l*~
set /* no_edit */ phi1 = (l*110 h*80 l*10)*~
set /* no_edit */ phi2 = (l*10 h*80 l*110)*~
option sigunit = 1n
option outacc = 10p
option simperiod = 4000
option level = 3
/*
*%
tstep 0.1n
trise 0.5n
tfall 0.5n
*%
*%
*/
/* auto_print */
/* auto_plot */
\end{verbatim}

\section{Configurationfile}
At start-up of the program,
\tool{simeye}
will read some information from a configuration
file called '.simeyerc'.
First, it tries to find this file in the current directory.
Second, it tries to find this file in the home directory of the user.
Thirdly, it looks for this file in process directory.
The configuration file may contain the following keywords, followed
by a specification on the same line if the keyword ends with ':'.

\begin{description}
\item [SLS:]
Specifies the command for running the
\tool{sls}
simulator.
\item[SLS\_LOGIC\_LEVEL:]
Specifies the level of simulation when '{\bf sls-logic}' is selected
(use 1 or 2).
\item [SLS\_LOGIC\_SIGNAL:]
Specifies the default signal representation for {\bf sls-logic} simulations
(use A for {\bf Analog} or D for {\bf Digital}).
\item [SLS\_TIMING\_SIGNAL:]
Specifies the default signal representation for {\bf sls-timing} simulations
(use A for {\bf Analog} or D for {\bf Digital}).
\item [SPICE:]
Specifies the command for running the
\tool{spice}
simulator (use
\tool{nspice}
or a derivative of it).
\item [XDUMP\_FILE:]
Specifies the name of the X Window dump file that is generated
when the {\it Hardcopy} command is clicked.
\item [PRINT:]
Specifies the command that is executed
when the {\it Hardcopy} command is clicked
in order to process the X Window dump file (e.g. to convert the window dump to
PostScript and to send the output to a laser-printer).
\item [PRINT\_LABEL:]
Specifies an optional label that is placed in upper left corner
of the display window when an X Window dump is generated.
\item [SETTINGS\_FILE:]
Specifies the name of the file in (from)
which the window settings are stored (loaded) when using
the {\it Save Window} command ({\it Load Window} command or {\bf -L} option).
\item [DETAIL\_ZOOM\_ON]
If this keyword is specified in the configuration file,
the zoom-in function is also defined for the
y-axis of a (single analog) signal.
In this case it is not necessary to hold the $<$Shift$>$ key down
(see command {\it ZoomIn}).
\item [TRY\_NON\_CAPITAL\_ON]
If this keyword is specified in the configuration file
and \tool{simeye} fails to open a command file that starts with a capital letter,
the program will try to open the same command file but now starting
with a non-capital letter.
If this succeeds and
when performing a simulation, \tool{simeye} will first run the program 
\tool{arrexp}
on a copy of the command file
to expand one dimensional array node names into single node names
(e.g. a[1..3] is converted into a\_1\_ a\_2\_ a\_3\_).
\end{description}
\par
In the above specifications, the strings '\$circuit' and '\$stimuli' must be
used to refer to the selected cell or file name for simulation commands,
'\$file' may be used to refer to the current file on screen
(use '\$cell' for this name without extension (maximum = 22 chars)),
'\$date' and '\$time' may be used to refer to the current date and time,
and '\$user' may be used to refer to your login name.
\par
Example of a configuration file (default values are shown):

\begin{verbatim}
SLS: sls $circuit $stimuli
SLS_LOGIC_LEVEL: 2
SPICE: nspice $circuit $stimuli
PSTAR: npstar $circuit $stimuli
XDUMP_FILE: simeye.wd
PRINT: xpr -device ps -output $cell.ps simeye.wd; lp $cell.ps;
       rm simeye.wd $cell.ps
PRINT_LABEL: $user  $file  $date  $time
SETTINGS_FILE: simeye.set
\end{verbatim}

\section{Files}

\begin{tabular}{ll}
.simeyerc & (default) configuration file\\
HOME/.simeyerc & (altern.) configuration file\\
ICDPATH/share/lib/process/{\it process}/.simeyerc & (altern.) configuration file\}\\
ICDPATH/share/lib/process/{\it process}/.simeyerc & (altern.) configuration file\}\\
{\it cell}.res & (default) input file (SLS logic results)\\
{\it cell}.plt & (opt.) input file (SLS analog results)\\
{\it cell}.ana & (opt.) input file (SPICE analog results)\\
{\it cell}.cmd & (opt.) command file\\
*.cxx & temporary files\\
simeye.def.d & (default) template for command file\\
ICDPATH/share/lib/process/{\it process}/simeye.def.d & (altern.) template for command file\\
ICDPATH/share/lib/process/{\it process}/simeye.def.d & (altern.) template for command file\\
ICDPATH/share/lib/.X2PSfontmapfile & for PostScript output\\
simeye.eps & (default) print output file\\
simeye.set & (default) window-settings file\\
sim.diag & simulation diagnostics\\
pr.diag & print diagnostics
\end{tabular}

\section{X-defaults}
Some examples of ''.Xdefaults'' values are:

\begin{verbatim}
simeye*.borderColor:	white
simeye*StatusArea*.foreground:	yellow
simeye*cellDialog*.foreground:	cyan
simeye*iDialog*.foreground:	red
simeye*pDialog*.foreground:	red
simeye*qDialog*.foreground:	red
simeye*commands*.foreground:	white
simeye*commands*.background:	black
simeye*canvas.background:	black
simeye*canvas.foreground:	white
simeye*canvas.height:	400
simeye*canvas.width:	800
simeye.freestateColor:	red
simeye.high_lowColor:	cyan
simeye.undefinedColor:	blue
simeye.valuesColor:	red
simeye.voltsColor:	green
simeye.pointerColor:	yellow
simeye.std_bg:	grey80

Use the following set of accelerator key bindings:

simeye.keymnemonic:	False
simeye*Add.accelerator:	<Key>a
simeye*Add.acceleratorText:	a
simeye*Change.accelerator:	<Key>c
simeye*Change.acceleratorText:	c
simeye*Command.accelerator:	Shift<Key>c
simeye*Command.acceleratorText:	C
simeye*Copy.accelerator:	<Key>o
simeye*Copy.acceleratorText:	o
simeye*Delete.accelerator:	<Key>d
simeye*Delete.acceleratorText:	d
simeye*Edit.accelerator:	<Key>e
simeye*Edit.acceleratorText:	e
simeye*EraseAll.accelerator:	Ctrl<Key>e
simeye*EraseAll.acceleratorText:	^e
simeye*Exit.accelerator:	<Key>x
simeye*Exit.acceleratorText:	x
simeye*Full.accelerator:	<Key>f
simeye*Full.acceleratorText:	f
simeye*Grid.accelerator:	<Key>g
simeye*Grid.acceleratorText:	g
simeye*Hardcopy.accelerator:	Ctrl<Key>h
simeye*Hardcopy.acceleratorText:	^h
simeye*Load Window.accelerator:	Shift<Key>l
simeye*Load Window.acceleratorText:	L
simeye*Manual.accelerator:	<Key>F1
simeye*Manual.acceleratorText:	F1
simeye*Measure.accelerator:	Shift<Key>m
simeye*Measure.acceleratorText:	M
simeye*Move.accelerator:	<Key>m
simeye*Move.acceleratorText:	m
simeye*New.accelerator:	Ctrl<Key>n
simeye*New.acceleratorText:	^n
simeye*Open.accelerator:	Ctrl<Key>o
simeye*Open.acceleratorText:	^o
simeye*Prepare.accelerator:	<Key>p
simeye*Prepare.acceleratorText:	p
simeye*Print.accelerator:	Ctrl<Key>p
simeye*Print.acceleratorText:	^p
simeye*Put.accelerator:	<Key>u
simeye*Put.acceleratorText:	u
simeye*ReName.accelerator:	<Key>n
simeye*ReName.acceleratorText:	n
simeye*Read Append.accelerator:	Ctrl<Key>q
simeye*Read Append.acceleratorText:	^q
simeye*Read.accelerator:	Ctrl<Key>r
simeye*Read.acceleratorText:	^r
simeye*Redraw.accelerator:	Ctrl<Key>f
simeye*Redraw.acceleratorText:	^f
simeye*Run.accelerator:	<Key>r
simeye*Run.acceleratorText:	r
simeye*Save As.accelerator:	Ctrl<Key>a
simeye*Save As.acceleratorText:	^a
simeye*Save Window.accelerator:	Shift<Key>s
simeye*Save Window.acceleratorText:	S
simeye*Save.accelerator:	Ctrl<Key>s
simeye*Save.acceleratorText:	^s
simeye*Speed.accelerator:	<Key>s
simeye*Speed.acceleratorText:	s
simeye*T_end.accelerator:	<Key>t
simeye*T_end.acceleratorText:	t
simeye*Values.accelerator:	<Key>v
simeye*Values.acceleratorText:	v
simeye*Yank.accelerator:	<Key>y
simeye*Yank.acceleratorText:	y
simeye*ZoomIn.accelerator:	<Key>z
simeye*ZoomIn.acceleratorText:	z
simeye*ZoomOut.accelerator:	Shift<Key>z
simeye*ZoomOut.acceleratorText:	Z
\end{verbatim}

\section{See also}
sls(1ICD),
spice(1ICD),
nspice(1ICD),\\
SLS: Switch-Level Simulator User's Manual,\\
User's Guide for SPICE, Univ. of California at Berkeley,\\
X Window System Documentation. \index{simeye|)}
\cleardoublepage

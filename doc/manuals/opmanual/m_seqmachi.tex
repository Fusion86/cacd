\section{Introductie Sequenti\"ele Machines}
\label{intro_seq_machine}

\subsection{Inleiding}

De Hotelschakeling, die in paragraaf \ref{voorbeeld} besproken wordt, is een voorbeeld van
een {\em sequenti\"ele schakeling\/}: een schakeling met geheugenwerking.
Bij een sequenti\"ele schakeling hangen de uitgangssignalen niet alleen af van
de ingangssignalen: de schakeling kan, afhankelijk van de {\em toestand\/}
waarin hij zich bevindt, verschillende uitgangssignalen produceren bij
dezelfde ingangssignalen.
Dit in tegenstelling tot combinatorische logica, waarbij gegeven
ingangssignalen altijd dezelfde uitgangssignalen zullen opleveren.
Sequenti\"ele schakelingen onderscheiden zich van combinatorische logica
doordat ze ge\-heu\-gen\-ele\-men\-ten bevatten.
Deze geheugenelementen kunnen latches zijn (bij sequenti\"ele schakelingen
volgens het {\em level mode\/} model), of flipflops ({\em clock mode\/}
model).
In de geheugenelementen wordt de toestand
van de schakeling opgeslagen.
Sequenti\"ele schakelingen kunnen beschreven worden d.m.v.\ een
toestandsdiagram.

Bij het practicum zullen we sequenti\"ele schakelingen nodig hebben
om besturingseenheden te realiseren.
Om het ontwerpen van deze besturingseenheden gestructureerd te laten verlopen
en de verificatie van de ontworpen schakelingen te vereenvoudigen
zullen we een aantal afspraken maken, waaraan de besturingen
moeten voldoen:

\begin{itemize}
\item
We maken gebruik van een \'e\'en-fase kloksignaal. De toestand van de schakeling
zal bewaard worden in flipflops, die \'e\'en keer per klokperiode inlezen.
\item
Alleen schakelingen volgens het {\em Moore model\/} zijn toegestaan.
Dit betekent, kort gezegd, dat de uitgangssignalen niet onmiddellijk
af mogen hangen van de ingangssignalen, maar alleen van de toestand
waarin de schakeling zich bevindt.
\item
De schakelingen moeten volledig {\em synchroon\/} zijn. \index{synchroon}
Dat wil zeggen dat alle geheugenelementen op hetzelfde tijdstip
inlezen. Het is dus niet toegestaan om logica op te nemen
in de kloklijnen, omdat dit tot klokverschuiving ({\em clock-skew\/}) \index{clock-skew}
leidt. Ook het gebruik van ongelijke clock-buffers kan
clock-skew veroorzaken.
\item
Alle ongebruikte toestanden moeten gecodeerd worden. Dit om te voorkomen
dat de schakeling kan vastlopen in \'e\'en of meerdere ongebruikte
toestanden.
\end{itemize}

In de volgende paragrafen zullen we een praktische inleiding geven in het
beschrijven
van sequenti\"ele schakelingen.
Een uitgebreidere en meer formele behandeling
is te vinden in \cite{DT}.

\subsection{Het Moore model}
\index{Mealy|(bold}
\index{Moore|(bold}
Voordat we kunnen gaan kijken wat het Moore model nu precies inhoudt zullen
we eerst het algemene wiskundige model van een sequenti\"ele schakeling
bekijken:
de {\em sequenti\"ele machine\/}. Deze kan gedefinieerd worden als een
geordend vijftal:

\begin{equation}
M = (I,U,Q,\delta,\lambda)
\end{equation}

\begin{tabbing}
waarin \= \\
\> $I$ de verzameling van {\em ingangssymbolen\/} is; \\
\> $U$ de verzameling van {\em uitgangssymbolen\/} is; \\
\> $Q$ de verzameling van {\em toestanden\/} is; \\
\> $\delta$ de {\em toestandsfunctie\/}, een afbeelding van $Q \times I$ in $Q$ is; \\
\> $\lambda$ de {\em uitgangsfunctie\/}, een afbeelding van $Q \times I$ op $U$ is.
\end{tabbing}

We zullen hierbij aannemen dat de verzamelingen $I$, $U$ en $Q$
eindig en niet-leeg zijn.
Bovenstaande definitie is schematisch weergegeven in figuur
\ref{figuur:seq_mod}.

\begin{figure}[bth]
\centerline{\callpsfig{seq_mod.ps}{width=1\textwidth}}
\caption{Blokschema sequenti\"ele machine (Mealy model).}
\label{figuur:seq_mod}
\end{figure}

De blokken die aangeduid zijn met $\delta$ en $\lambda$ stellen combinatorische
logica voor; het blok dat aangeduid is met $Q$ bevat de geheugenelementen.
Het model van figuur \ref{figuur:seq_mod}
staat in de literatuur bekend als het {\em Mealy model\/}.
In dit model is de $\lambda$-functie gedefinieerd als een functie van (het
Cartesisch produkt van) de toestandsverzameling $Q$ en
de ingangsverzameling $I$ op de uitgangsverzameling $U$: 

\begin{equation}
\lambda : Q \times I \stackrel{op}{\rightarrow} U
\end{equation}

Het Mealy model staat dus een direct combinatorisch
verband toe tussen de ingangswaarde en de uitgangswaarde.
Dit betekent dat bij schakelingen volgens het Mealy model de uitgangen
direct kunnen veranderen wanneer \'e\'en van de ingangssignalen verandert.

Bij het zogenaamde {\em Moore model\/} is dit directe verband
van ingang naar uitgang niet toegestaan.
Het Moore model definieert de $\lambda$-functie als volgt:

\begin{equation}
\lambda : Q \stackrel{op}{\rightarrow} U
\end{equation}

Dit is schematisch weergegeven in figuur \ref{figuur:moore_mod}.

\begin{figure}[bth]
\centerline{\callpsfig{moore_mod.ps}{width=1\textwidth}}
\caption{Blokschema van het Moore model.}
\label{figuur:moore_mod}
\end{figure}

Wiskundig gezien is het Moore model dus een beperking van
het Mealy model.
Bij het toepassen van beide modellen op praktische schakelingen blijkt
dat schakelingen die volgens het Mealy model gespecificeerd zijn in
het algemeen een kortere reactietijd hebben
(in aantal klokperioden gerekend) dan schakelingen volgens    
het Moore model.
Daarnaast vergt een specificatie volgens het Mealy model meestal
minder toestanden.
Een groot nadeel van het Mealy model is echter dat er
zgn.\ {\em statische lusvorming\/} kan optreden. 
Dit kan gebeuren
wanneer twee apart ontworpen schakelingen op elkaar worden aangesloten,
zoals is weergegeven
in figuur \ref{figuur:modules}.

Figuur \ref{figuur:modules} toont twee modules die met elkaar communiceren onder
besturing van een kloksignaal.
Wanneer beide schakelingen zijn gespecificeerd volgens het Mealy model
is het mogelijk dat een uitgangssignaal van module 1
rechtstreeks de uitgang van module 2 be\"{\i}nvloedt.
Het uitgangssignaal van module 2 gaat naar module 1 en kan zo
op zijn beurt onmiddellijk het uitgangssignaal van module 1 veranderen.
Dit kan weer zijn invloed hebben op de uitgang van module 2, enzovoort.
Er is dan sprake van statische (=~logische) lusvorming tussen 
module 1 en module 2.

\begin{figure}[bth]
\centerline{\callpsfig{modules.ps}{width=6cm}}
\caption{Interactie tussen twee modules.}
\label{figuur:modules}
\end{figure}

Het is duidelijk dat er op deze manier een oscillatie kan ontstaan
in de communicatie tussen beide modules.
Het is in dat geval onzeker wat voor waarde de in- en uitgangssignalen
van beide modules zullen hebben op het moment dat de 
eerstvolgende klokpuls komt.
Hierdoor kan het gebeuren dat twee modules die elk op het eerste gezicht
correct gespecificeerd lijken, toch niet goed functioneren
wanneer ze aan elkaar gekoppeld worden.

Wanneer beide modules daarentegen volgens het Moore model gespecificeerd
zijn, zal het probleem van statische lusvorming zich niet voordoen.
Wanneer module 1 zijn uitgang verandert, kan dit pas na de eerstvolgende klokpuls
van invloed zijn op de uitgang van module 2.
Hierop kan module 1 pas een klokpuls later reageren.
De mogelijkheid van oscillaties binnen \'e\'en klokperiode is hier dus niet
aanwezig.
Dit maakt de verificatie van schakelingen volgens het Moore model een stuk
eenvoudiger dan schakelingen volgens het Mealy model.
Daarom wordt voor het ontwerppracticum ge\"eist dat alle
sequenti\"ele schakelingen van het Moore-type zijn. 
\index{Mealy|)} \index{Moore|)}
\subsection{Toestandsdiagrammen}

Sequenti\"ele schakelingen kunnen worden weergegeven door middel van
toestandsdiagrammen.
We zullen ons hier beperken tot sequenti\"ele schakelingen volgens het Moore model.

In een toestandsdiagram geven we iedere toestand weer met een bolletje.
In het bolletje wordt de naam van de toestand gezet. Ook kan de uitgangscombinatie
die bij die toestand hoort in het bolletje gezet worden.
\footnote{Bij het Moore model is de uitgang alleen afhankelijk van de
toestand waarin de schakeling zich bevindt.
Bij het Mealy model is de uitgang ook afhankelijk van de huidige
ingangscombinatie, zodat de uitgangswaarden dan
bij de pijlen voor de toestandsovergangen komen te staan.}
Overgangen van de ene toestand naar de andere worden weergegeven met pijlen.
Bij iedere pijl staat voor welke ingangscombinatie de overgang optreedt.
De meeste besturingsschakelingen hebben \'e\'en resettoestand
en een dominant resetsignaal.
\footnote{Onder een dominant resetsignaal zullen we hier verstaan: een
signaal dat de schakeling op de eerstvolgende klokpuls in \'e\'en
gedefinieerde toestand brengt, ongeacht de waarde van de overige 
ingangssignalen of de toestand waarin de schakeling zich bevindt.}
Wanneer het resetsignaal geactiveerd wordt, komt de besturing
altijd in de resettoestand, ongeacht de andere ingangssignalen.
De resettoestand kan in een toestandsdiagram aangegeven worden
d.m.v.\ een dubbele pijl.
De aparte pijlen vanuit iedere toestand naar de resettoestand kunnen
dan weggelaten worden.
Dit is weergegeven in figuur \ref{figuur:toestand_vb}.

\begin{figure}[bth]
\centerline{\callpsfig{toestand_vb.ps}{width=8cm}}
\caption{Links: voorbeeld van een stukje toestandsdiagram.
Rechts: aanduiding van een dominant resetsignaal.}
\label{figuur:toestand_vb}
\end{figure}

Het is handig om in toestandsdiagrammen symbolische
namen te geven aan toestanden: een toestandsdiagram
met namen als 'rusttoestand' en 'wacht op input' is meestal duidelijker
dan een toestandsdiagram met namen als 'Q0' en 'Q1'.
Hetzelfde geldt voor de in- en uitgangssignalen: symbolische
benamingen zijn meestal duidelijker dan rijen bits.
In een later stadium wordt dan een toestandstabel opgesteld, waarin
bitcombinaties worden toegekend aan de symbolische namen. 
Ook is het mogelijk om in eerste instantie een groep toestanden weer te
geven als \'e\'en enkele toestand, en deze groep pas in een later
stadium verder uit te werken: zulke toestanden worden {\em meta-toestanden\/}
genoemd.
Een andere reden om een toestand op te splitsen in een aantal toestanden
kan bijvoorbeeld zijn dat bij het uitwerken van een ontwerp blijkt dat een
gevraagde actie
niet in \'e\'en klokperiode te realiseren is, en dus verdeeld moet worden
over meerdere perioden.

\subsection {Voorbeeld: Modeltrein-besturing}
In deze paragraaf zullen we 
het ontwerp van een eenvoudige sequenti\"ele
schakeling uitwerken.
De opdracht is als volgt:

OPDRACHT \\ 
{\em Ontwerp de besturing voor een modeltrein. De trein kan
in twee richtingen rijden en wordt bediend
d.m.v.\ een druktoets en een schakelaar.
Met de druktoets kan de trein op gang of tot stilstand gebracht
worden.
De schakelaar bepaalt of de trein voor- of achteruit
rijdt.
De rijrichting kan alleen veranderd worden door de trein
eerst tot stilstand te brengen. \\
De beide ingangssignalen zijn gesynchroniseerd met de klok.
Het indrukken van de druktoets levert een puls op ter
lengte van \'e\'en klokperiode.
Er is een resetsignaal dat
bij het aanzetten van de modeltrein gedurende enkele
klokperioden hoog is, en daarna verder laag (power-on reset).
De besturing heeft twee uitgangssignalen, die aangeven of
de trein voor- of achteruit rijdt.
Wanneer beide signalen laag zijn, staat de trein stil.}

UITWERKING\\
De besturing heeft drie toestanden: rust, vooruit en
achteruit.
Overgangen tussen deze toestanden treden op
wanneer de druktoets wordt ingedrukt.
De stand van de schakelaar bepaalt of de trein bij
het indrukken van de druktoets in de rusttoestand, vooruit (schakelaar = 1)
of achteruit (schakelaar = 0) gaat rijden.
Het power-on resetsignaal kan gebruikt worden om te voorkomen
dat de trein meteen begint te rijden, wanneer de modelspoorbaan
wordt aangezet.
\footnote{Het is niet bekend wat voor waarde de geheugenelementen aannemen als
de schakeling wordt aangezet. Daardoor is ook de toestand waarin de schakeling
bij het aanzetten terecht komt onbekend.}
Hiertoe wordt de rusttoestand als resettoestand gekozen.
Het resetsignaal is dominant over de andere ingangssignalen.
Het toestandsdiagram is weergegeven in figuur \ref{figuur:modeltrein}.
Merk op dat er in het toestandsdiagram gebruik is gemaakt van
het gegeven dat de druktoets pulsen afgeeft met een tijdsduur
van precies \'e\'en klokperiode.
Wanneer een puls langer aanhoudt, zal de schakeling gaan oscilleren
tussen de rusttoestand en \'e\'en van de andere twee toestanden,
afhankelijk van de stand van de schakelaar.
Dit valt op te lossen door het toestandsdiagram uit te breiden
met drie extra toestanden.
De overgangen in het toestandsdiagram lopen dan alle via zo'n
extra toestand, waarin gewacht wordt op het loslaten van de
druktoets, alvorens door te gaan naar \'e\'en van de huidige
toestanden 'rust', 'vooruit' en 'achteruit'.

\begin{figure}[bth]
\centerline{\callpsfig{modeltrein.ps}{width=10cm}}
\caption{Toestandsdiagram van de modeltrein-besturing.}
\label{figuur:modeltrein}
\end{figure}


We kunnen nu dit toestandsdiagram gaan vertalen naar een schakeling.
De toestand van de schakeling kan bewaard worden in flipflops.
Omdat er drie toestanden zijn, kunnen we de toestand coderen
in twee flipflops: $Q_{1}$ en $Q_{0}$.
Wanneer we flipflops gebruiken met een synchrone reset-ingang,
kunnen we het resetsignaal hier meteen op aansluiten.
Daarmee wordt de rusttoestand gecodeerd met $Q_{1} Q_{0} = 00$.
We kunnen vervolgens bitcombinaties toekennen aan de overige toestanden.
We kunnen de combinaties $01$ en $10$ gebruiken om de toestanden
'achteruit' en 'vooruit' te coderen.
We hebben dan nog \'e\'en combinatie niet gebruikt: $Q_{1} Q_{0} = 11$.
We moeten zorgen dat, mocht de schakeling in deze toestand terechtkomen,
hij er niet in vast blijft zitten, en dat de uitgangssignalen 
in ieder geval 'netjes' zijn 
gedefinieerd\footnote{We kunnen in principe de toestandsvariabele $Q_{1}$ 
gebruiken om het uitgangssignaal
'vooruit' te generen, en $Q_{0}$ voor 'achteruit'.
Dit heeft als voordeel dat er geen extra logica nodig is om
de uitgangssignalen te bepalen, behalve een eventuele buffer
om de flipflops niet te zwaar te belasten.
Wanneer de schakeling dan in de ongebruikte toestand komt, zullen
beide uitgangen echter hoog worden. Hier zou de modeltrein kapot van kunnen
gaan. Daarom is er hier voor gekozen om beide uitgangen in deze
toestand laag te maken, hetgeen ten koste zal gaan van wat extra logica.}.

\begin{table}[hbt]
\begin{center}
\begin{tabular}{cc|l|cc} 
\hline
$Q_{1}$ & $Q_{0}$ & toestand & Vooruit & Achteruit\\
\hline
\hline
0 & 0 & rust & 0 & 0\\
0 & 1 & achteruit & 0 & 1\\
1 & 0 & vooruit & 1 & 0\\
\hline
1 & 1 & ongebruikt & 0 & 0\\
\hline
\end{tabular} \\
\caption{Toestandscodering en uitgangssignalen.
\label{tabel:codering}
}
\end{center}
\end{table}

De codering van de toestanden met de bijbehorende uitgangssignalen is
weergegeven in tabel \ref{tabel:codering}.
Tabel \ref{tabel:toestanden} geeft de bijbehorende functietabel,
waarin de signalen druktoets, schakelaar,
vooruit en achteruit zijn afgekort tot resp. D, S, V, A.
Met de fase $n$ of $n+1$ van een signaal wordt
de waarde die dat signaal aanneemt na
de $n^{e}$ resp. $n+1^{e}$ klokpuls aangeduid.
In de tabel is het resetsignaal weggelaten (dit wordt rechtstreeks aangesloten
op de flipflops).
Deze tabel kan worden omgezet naar combinatorische logica,
om de opvolgertoestanden te bepalen ($\delta$ in figuur
\ref{figuur:moore_mod}) en de uitgangssignalen te
genereren ($\lambda$ in figuur
\ref{figuur:moore_mod}).
Deze omzetting kan handmatig gedaan worden,
of m.b.v synthese software.
Dan kan er een netwerk aangemaakt worden, waarin twee flipflops
gekoppeld worden aan de logica.
Daarbij worden de uitgangen voor $Q^{n+1}$ verbonden met de D-ingangen
van de flipflops.
De netwerkbeschrijving is dan gereed, en de besturingseenheid kan
getest worden.

\begin{table}[bth]
\begin{center}
\begin{tabular}{cccc|cc|cc} 
\hline
$D^{n}$ & $S^{n}$ & $Q_{1}^{n}$ & $Q_{0}^{n}$ & $Q_{1}^{n+1}$ & $Q_{0}^{n+1}$
& $V^{n}$ & $A^{n}$ \\
\hline
\hline
0 & 0 & 0 & 0 & 0 & 0 & 0 & 0\\
1 & 0 & 0 & 0 & 0 & 1 & 0 & 0\\
1 & 1 & 0 & 0 & 1 & 0 & 0 & 0\\
0 & - & 0 & 1 & 0 & 1 & 0 & 1\\
1 & - & 0 & 1 & 0 & 0 & 0 & 1\\
0 & - & 1 & 0 & 1 & 0 & 1 & 0\\
1 & - & 1 & 0 & 0 & 0 & 1 & 0\\
\hline
- & - & 1 & 1 & 0 & 0 & 0 & 0\\
\hline
\end{tabular}
\caption{Functietabel van de modeltrein-besturing.
\label{tabel:toestanden}
}
\end{center}
\end{table}

\cleardoublepage




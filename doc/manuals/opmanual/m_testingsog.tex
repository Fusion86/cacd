\def\figurename{Figure}
\section{Testing SOG chips with the Logic Analyzer}
\label{testingSOG}

\subsection{Introduction}
In this document we describe how a Sea-Of-Gates chip in a DIL40 package can be tested 
with the logic analyzer LA-5580. 
The purpose is to verify if, given a set of input patterns, 
the tested device produces the same output patterns as a set of reference output patterns. 

A logic analyzer is an instrument that is similar to an oscilloscope:
both instruments measure output signals (voltage) as a function of time
after some trigger moment.
However, important difference are:
\begin{itemize}
\item
A logic analyzer usually only measures two different signal levels: high and low.
\item
A logic analyzer is capable of measuring a large number of output signals (e.g. 80).
\item
Besides measuring output signals, a logic analyzer can simultaneously apply input signals.
\end{itemize}
The logic analyzer LA-5580 is a hardware box that is controlled 
(via a USB connection) by a graphical interface program LA5000 that is running a PC.
A picture of the interface is shown in Figure~\ref{la5000}
\begin{figure}[h]
\centerline{\callpsfig{la5000.eps}{width=0.95\textwidth}}
\caption{Graphical Interface for the Logic Analyzer LA-5580}
\label{la5000}
\end{figure}
\subsection{Overview}
As input we need a file with reference input and output patterns (.ref file) and a file with 
information about how the pins of the design are connected to the pins of IO buffers of 
the chip (.buf file).  Typically, both these files have been produced during the design of 
the chip. Using the program test\_flow, these files are converted into a pattern (.csv)
file and an initialization (.ini) file that can be used as input for the logic analyzer 
program,
and a scheme that shows how the pods of the logic analyzer should be connected to 
the pins of the package of the SOG chip. 
Then,
via the the logic analyzer program LA5000,
the logic analyzer is used to send the input patterns to the tested SOG device 
and to collect the resulting output patterns. 
Finally, the 
program test\_flow is used to compare the measured output patterns against the reference 
output patterns.
\begin{figure}[h]
\centerline{\callpsfig{testflow.eps}{width=.60\textwidth}}
\caption{The different steps for testing}
\label{testflow}
\end{figure}

\subsection{Obtaining the Reference Files}
Copy the relevant .ref and .buf files
that you have created during the design phase to
a writable directory on the local disk of your PC 
(e.g. C:$\backslash$wpstmp).
Note that it is important to store the files on the local disk
because otherwise the logic analyzer software may have problems to read it.
Note that the desktop of the PC is usually not on the local disk in our environment.

\subsection{Preparing the Input Data for the Logic Analyzer}
Start the program test\_flow (T:$\backslash$ET2805$\backslash$common$\backslash$test\_flow.tcl).
Next click Commands $\rightarrow$ Read\_reffile and open the .ref 
file.  If the same directory contains a .buf file, you will be asked if it is ok to update the 
information with this file. In that case, click yes. Otherwise click Commands $\rightarrow$ 
Update\_with\_buffile and open the appropriate .buf file. To create a window with input 
patterns and output patterns in csv format, click Commands $\rightarrow$ Make\_csvfile.
In the new csv window, use the command File $\rightarrow$ Save\_csvfile to create the 
data files that can be used as input for the logic analyzer.

A wiring scheme for how to connect the clips of the wires of the pods to the pins of the 
SOG device can be obtained by first selecting the appropriate bondbar number for the 
package in the upper right corner of the test\_flow window and next using the command 
Commands $\rightarrow$ Show\_pod\_connections. 

\subsection{Setting up the Logic Analyzer}
Connect the USB cable of the logic analyzer LA-5580 to the upper USB port at
the front of your computer.  
Plug in the pattern generation pods Pod 1A, Pod 2A, etc. in the lower connector rows 
(label ''board 1'') of the analyzer and the logic (measurement) pods Pod 1B, Pod 2B, 
etc. in the upper connector rows (label ''board 2'') of the analyzer.  
Note that the pods 
have numbers on them. Plug Pod 1 in the leftmost connector, Pod 2 in the second 
leftmost connector etc.

\subsection{Mounting and connecting the SOG chip}
Mount the SOG package on the DIL40 connector box.  From the separate IC testbox,
connect GND (0 Volt) and VDD (5 Volt) to the appropriate connector holes on the 
DIL40 connector box, using cables with 4mm plugs.
Connect the clips of the wires of the pods to the pins of the SOG package according to 
the device-wiring scheme that was generated using test\_flow. Connect at least one black 
wire of each pod to the GND signal.  The pod input Pod 1B(0) at channel 0 will be used 
as a trigger signal for the acquisition of data and should directly be connected to an 
output of a pattern generation pod. 
This connection is given at the bottom of the wiring 
scheme that was obtained using test\_flow.

Switch on the power supply for the SOG chip and switch on the power supply for the 
logic analyzer.  
Be careful to only reconnect wires to the pins of the device when the 
power supply for the SOG chip is switched off. 

\subsection{Using the Logic Analyzer}
Start the Logic Analyzer control program LA5000 Logic Analyzer.
If the Timing view window is not yet shown, click Timing $\rightarrow$ Timing window.

Besides the .csv file that was created by test\_flow, also a .ini file
with the same base name was created by test\_flow.
Click File $\rightarrow$ Open to first open the file .ini file that was created by test\_flow.

Next, click View $\rightarrow$ Clear Data Buffer to clear all signals.
Then, click File $\rightarrow$ Open to open the .csv file that was generated using test\_flow.
The option Type 2 CSV file must be enabled during this.
After reading the .csv file you will see the reference output signals in the Timing window
and you may inspect the input signals when you click Pod $\rightarrow$ Edit Pattern
Generator Data (or button EDIT PAT).
The reference output signals will be overwritten when capturing analysis
data from the logic analyzer.

Then, click Trigger $\rightarrow$ Go or the Go button to start the analysis. 
The output of the analysis will appear in the Timing view window. 

To prepare for a more exact comparison using the program test\_flow, 
export the output signals using File $\rightarrow$ Export. 
For the Data field, select ''Group outputs'' and select Binary, click OK, and then save the file as 
type ''Comma Sep (.CSV)''.

\subsection{Comparing the Analyzer Output with the Reference Output}
In test\_flow, click on Commands $\rightarrow$ Compare. 
Next, in the compare window, click File $\rightarrow$ Compare csv and select
the .csv file that was created using the Logic Analyzer.
After the file has been read,
the measured output signals will be compared against the reference output signals,
just before every rising clock edge. 
On each line you will see from left to right: the time, the clock period number,
the input signals, the reference output signals and the measured output signals. 
In case the reference outputs and measured outputs are different, they have a 
red colour.  
You can click on a line to list the different output vectors below each other.  
If you click on a column in this list, the name of the corresponding output terminal will 
be highlighted at the sub window at the right. 
You can use commands from the 
Commands button in the compare window to navigate through the errors.

Default, all output signals and all inout signals will be compared.
By disabling the toggle button inout, only output signals will be compared.

\subsection{Trouble Shooting}
When the output signals are not according to as expected, carefully check all
pod connections. 
Further you can try to lower the sample rate
Thirdly, you can try another IC.

\subsection{Further Reading}
LA 5000 Logic Analyzer Software Manual (see OP Course Documents on blackboard) \\
Logic Analyzer LA-5000 series, Link Intruments Inc (http://linkinstruments.com)
\def\figurename{Figuur}


\section{Inwerkopdrachten}

\subsection{Inleiding}
Het doel van de inwerkopdracht is om u vertrouwd te maken met de ontwerpomgeving
die nodig is om de groeps\-opdracht te kunnen uitvoeren.
Per groep van 2 studenten dient 1 inwerkopdracht gemaakt te worden.
De voorbeeld inwerkopdracht over de hotel schakelaar
kan doorlopen worden ter voorbereiding
van de eigenlijke inwerkopdracht.

Naast de informatie in dit hoofdstuk is verdere uitleg te vinden in het 
hoofdstuk "Ont\-werptrajekt'' en de appendices. 
Het hoofdstuk "Ontwerptrajekt'' moet u in
ieder geval gelezen hebben voordat u begint aan de inwerkopdrachten.
\subsection{Het ontwerpen van een besturing}

\subsubsection{Inleiding}

In deze opdracht wordt aan de hand van een ontwerp van een eenvoudige besturing
het hele ont\-werptrajekt doorlopen.
Allereerst moet de opdracht bestudeerd worden en de specificaties, indien onvolledig,
verder vastgesteld worden. 
Vervolgens moet er op papier een toestandsdiagram gemaakt worden.
D.w.z.\ er moet vastgesteld \index{toestandsdiagram}
worden in welke verschillende toestanden de te ontwerpen schakeling kan verkeren en hoe,
onder invloed van de stuursignalen,
de overgangen plaatsvinden, zie~\cite{DT}.
Uit het toestandsdiagram kan dan een \smc{vhdl}-beschrijving worden afgeleid en
getest, zie ~\cite{DTVHDL}.
Daarna kan deze beschrijving worden gesynthetiseerd.
De gesynthetiseerde schakeling kan worden ingevoerd in het \smc{ocean/nelsis} systeem
en er kan hiermee een layout worden gegenereerd.
Na een circuit-extractie van de lay\-out
kan ook de afgebeelde schakeling op het
fishbone image getest worden op zijn functionele werking,
zowel op \smc{vhdl} gate-level niveau als transistor niveau.
In appendix \ref{intro_seq_machine} is 
een uitgebreide beschrijving te vinden van het ontwerpen van 
besturingen en aan welke voorwaarden zo'n besturing zou moeten voldoen.
In appendix \ref{good_vhdl} wordt
o.a. beschreven hoe zo'n besturing in \smc{vhdl}
kan worden beschreven.
Het is raadzaam deze appendices te bestuderen voordat u aan uw 
eigen ontwerp gaat beginnen.

\subsubsection{Het opstellen van een toestandsdiagram}
\index{toestandsdiagram}
Bestudeer de aan u toegekende opdracht.\\
Stel een (clock-mode) toestandsdiagram op volgens het Moore-model.
D.w.z.\ dat de uitgangssignalen niet van de ingangssignalen mogen afhangen,
maar alleen van de toestand zelf.
Ga hiervoor eerst na wat de ingangs- en uitgangssignalen zijn en welke toestanden kunnen optreden.
Geef alle toestanden en signalen een functionele naam.
Bepaal wat de uitgangssignalen zijn bij elke toestand.
Geef de overgangen aan als functie van de stuursignalen.
Controleer of alle mogelijke combinaties zijn weergegeven.

\subsubsection{Generatie van de besturing}
In de volgende paragraaf zal  worden besproken welke stappen er moeten
worden genomen om uiteindelijk tot een (werkende) schakeling te komen.\\
Het gehele ontwerp kan worden gedaan m.b.v. de \tool{GoWithTheFlow} interface.
Achtereenvolgens dienen nu de volgende stappen hierin te worden doorlopen:
\begin{itemize}
\item
Maak een \smc{vhdl}-file, die overeenkomt met de gemaakte FSM, en maak tevens
een testbench voor de FSM.
\item
Compileer de gemaakte \smc{vhdl}-files
\item
Simuleer de testbench en kijk of de schakeling zich gedraagt zoals verwacht.
Pas zonodig de \smc{vhdl}-beschrijving aan.
Maak tevens de list-file(s) voor latere vergelijking met simulaties
van de layout en testen aan de gerealiseerde schakeling.
\item
Synthetiseer de \smc{vhdl}-beschrijving van de schakeling.
Deze synthese levert twee beschrijvingen:
\begin{itemize}
\item
Een nieuwe \smc{vhdl}-beschrijving, bestaande uit componenten uit de
\smc{ocean/nelsis}-bibliotheek en hun onderlinge verbindingen.\\
Deze kunnen worden gecompileerd en dit kan via simulatie worden
getest op een correcte werking.
Eventueel moet de \smc{vhdl}-beschrijving worden aangepast.
\item
Een beschrijving van het circuit in het \smc{sls}-formaat.\\
Deze beschrijving kan worden gebruikt als circuit invoer voor het
\smc{ocean/nelsis} systeem. 
\end{itemize} 
\item
Genereer via \tool{madonna} (placer) en \tool{trout} (router) van \tool{seadali} de layout
van de schakeling.
\item
Extraheer de \smc{vhdl}-code uit de gemaakte layout en simuleer deze om
te zien of ook de layout goed functioneert.
\item
Genereer referentie- en commando-files om de schakeling te kunnen testen
op transistor niveau.
\item
Voer met de gegenereerde commando-file een simulatie uit op het circuit.
\item
Vergelijk de uitkomst van deze simulatie met de gegenereerde referentie-file.
Er mogen geen verschillen optreden.
\end{itemize}

\clearpage
\subsection{Voorbeeld: Een hotelschakelaar}
\label{voorbeeld}
In deze sectie zal een voorbeeld worden gegeven van een inwerkopdracht.
We zullen voor het gegeven probleem een toestandsdiagram opstellen,
vervolgens de hieruit afgeleide \smc{vhdl} gedragsbeschrijving invoeren,
daarna deze simuleren en synthetiseren, en als laatste de layout maken.
Voordat dit voorbeeld wordt doorlopen verdient het aanbeveling
de VHDL appendix en de Linux appendix
van de Studentenhandleiding Ontwerppracticum te lezen.

Gevraagd: Ontwerp een schakeling die een lamp aanstuurt met 3 toetsen(s1, s2, s3).
Wanneer 1 of meer toetsen worden ingedrukt, gaat de lamp aan of uit, afhankelijk
van de huidige toestand. Een 'overrule' toets(ov) zorgt dat de lamp in zijn
huidige toestand blijft.
Maak daartoe de schaking voor de black-box van figuur \ref{hotel-opdr}.\\
\begin{figure}[bth]
\centerline{\callpsfig{hotel-opdr.ps}{width=.70\textwidth}}
\caption{Ontwerpopdracht voor een hotelschakelaar}
\label{hotel-opdr}
\end{figure}

Uitwerking:\\
Het toestandsdiagram van figuur \ref{hotel-toest} kan voor de schakeling
worden opgesteld.\\

\begin{figure}[h]
\centerline{\callpsfig{hotel-toest.ps}{width=.90\textwidth}}
\caption{Toestandsdiagram voor de hotelschakelaar}
\label{hotel-toest}
\end{figure}

Zorg eerst dat alle paden naar te gebruiken programmatuur zijn gezet
door eenmalig het commando op\_init uit te voeren in een terminal window
(zie de Linux appendix).

Start nu het programma \tool{GoWithTheFlow} op door op het desbetreffende
icon op de DeskTop te klikken.
Dit zal de interface zoals in figuur \ref{designflow} te zien geven.
\begin{figure}[h]
\centerline{\callpsfig{GoWithTheFlow_opstart.eps}{width=.85\textwidth}}
\caption{Het programma \tool{GoWithTheFlow} na opstarten}
\label{designflow}
\end{figure}
\FloatBarrier

Maak dan een project directory aan m.b.v. het commando File $\rightarrow$ New project.
Een window zoals in figuur \ref{newproj} zal verschijnen.
Specificeer in het vak achter Selection de naam van een nog niet
bestaande directory, en klik op OK.
\begin{figure}[h]
\centerline{\callpsfig{newproj.eps}{width=.45\textwidth}}
\caption{Het aanmaken van een project directory}
\label{newproj}
\end{figure}

Voer dan allereerst de entity beschrijving voor hotel in door op File $\rightarrow$ New entity 
te klikken en het daarop verschijnende widget in te vullen zoals in figuur \ref{newent}.
Gebruik de tab toets om naar de volgende kolom te springen.
\begin{figure}[h]
\centerline{\callpsfig{newhotelent.eps}{width=.35\textwidth}}
\caption{Het invoeren van een nieuwe entity}
\label{newent}
\end{figure}

Na op OK geklikt te hebben verschijnt de bijbehorende \smc{vhdl} beschrijving
in een new window, zie figuur \ref{hotelent}.
\begin{figure}[h]
\centerline{\callpsfig{hotelG.vhd.eps}{width=.6\textwidth}}
\caption{De \smc{vhdl} code voor de hotel entity}
\label{hotelent}
\end{figure}
Klik op Compile. 
Na eerst bevestigend te hebben geantwoord op de vraag of de beschrijving 
moet worden weggeschreven zal deze worden
opgeborgen in de database van \tool{GoWithTheFlow}
en vervolgens worden gecompileerd.
\FloatBarrier

Klik vervolgens op de rechthoek ''hotel'' en selecteer Add Architecture
om de volgende gedragsbeschrijving in te voeren.
Merk op dat (conform wat in VHDL appendix is beschreven)
het eerste process statement (lbl1) een beschrijving bevat van 
de toestandsregisters, en het tweede process statement
(lbl2) een beschrijving van de combinatorische logica
die aan de hand van de huidige toestand en de ingangssignalen
de volgende toestand en de uitganssignalen berekent.
\begin{verbatim}
library IEEE;
use IEEE.std_logic_1164.ALL;

architecture behaviour of hotel is
   type lamp_state is (OFF0, OFF1, ON0, ON1);
   signal state, new_state: lamp_state;
begin
   lbl1: process (clk)
   begin
      if (clk'event and clk = '1') then
         if res = '1' then
            state <= OFF0;
         else
            state <= new_state;
         end if;
      end if;
   end process;
   lbl2: process(state, s, ov)
   begin
      case state is
         when OFF0 =>
            lamp <= '0';
            if (s = '1') and (ov = '0') then
                new_state <= ON1;
            else
                new_state <= OFF0;
            end if;
         when ON1 =>
            lamp <= '1';
            if (s = '0') and (ov = '0') then
                new_state <= ON0;
            else
                new_state <= ON1;
            end if;
         when ON0 =>
            lamp <= '1';
            if (s = '1') and (ov = '0') then
                new_state <= OFF1;
            else
                new_state <= ON0;
            end if;
         when OFF1 =>
            lamp <= '0';
            if (s = '0') and (ov = '0') then
                new_state <= OFF0;
            else
                new_state <= OFF1;
            end if;
      end case;
   end process;
end behaviour;
\end{verbatim}
Klik op Compile. Het programma zal vragen om de beschrijving eerst
weg te schrijven. Antwoord bevestigend, en wanneer er geen fouten
in de beschrijving zitten zal vervolgens een rechthoek ''behaviour'' 
onder het blokje hotel verschijnen ten teken dat
de beschrijving is opgeborgen in de database van \tool{GoWithTheFlow}.
Wanneer er wel compilatie fouten zijn, zullen deze fouten getoond
worden en zullen deze eerst verbeterd moeten worden.

Ga vervolgens op de rechthoek behaviour staan, klik met de linker muisknop
en selecteer Add Configuration.
Er zal nu een window verschijnen met een naam voor de \smc{vhdl} configuratie
file.  
Klik op OK, de \smc{vhdl} beschrijving voor de configuratie file wordt getoond,
schrijf die vervolgens weg en compileer die.

Maak nu op dezelfde wijze ook een entity beschrijving, 
een gedragsbeschrijving en een configuratie file aan
voor een testbench voor de hotel schakeling.
Noem de entity ''hotel\_tb'' (het is niet nodig om terminals te specificeren
voor deze testbench entity) en maak een behaviour beschrijving als volgt:
\begin{verbatim}
library IEEE;
use IEEE.std_logic_1164.ALL;

architecture behaviour of hotel_tb is
   component hotel
      port (clk  : in  std_logic;
            res  : in  std_logic;
            s    : in  std_logic;
            ov   : in  std_logic;
            lamp : out std_logic);
   end component;
   signal clk: std_logic;
   signal res: std_logic;
   signal s: std_logic;
   signal ov: std_logic;
   signal lamp: std_logic;
begin
   lbl1: hotel port map (clk, res, s, ov, lamp);
   clk <= '1' after 0 ns,
          '0' after 100 ns when clk /= '0' else '1' after 100 ns;
   res <= '1' after 0 ns,
          '0' after 200 ns;
   s <= '0' after 0 ns,
         '1' after 600 ns,
         '0' after 1000 ns,
         '1' after 1400 ns,
         '0' after 1800 ns,
         '1' after 2200 ns,
         '0' after 2600 ns,
         '1' after 3000 ns,
         '0' after 3400 ns,
         '1' after 3800 ns;
   ov <= '0' after 0 ns,
         '1' after 1800 ns,
         '0' after 2600 ns;
end behaviour;
\end{verbatim}
Het main window van \tool{GoWithTheFlow} zou nu een beeld moeten geven
zoals in figuur \ref{df1}.
\begin{figure}[h]
\centerline{\callpsfig{GoWithTheFlow1.eps}{width=.80\textwidth}}
\caption{\tool{GoWithTheFlow} na toevoegen behaviour and testbench}
\label{df1}
\end{figure}

Start nu een \smc{vhdl} simulatie door met de linker muisknop
op hotel\_tb\_behaviour\_cfg te klikken
en simulate te selecteren.
Na een simulatie over bijvoorbeeld 4000 ns zal het wave window van ModelSim
een resultaat geven zoals in figuur \ref{sim1}.
\begin{figure}[h]
\centerline{\callpsfig{modelsim1.eps}{width=.80\textwidth}}
\caption{Resultaat van een of \smc{vhdl} simulatie met ModelSim}
\label{sim1}
\end{figure}
Om zo dadelijk ook een switch-level simulatie (een simulatie op
transistor niveau) voor de gemaakte layout te kunnen verrichten 
zullen we nu een .lst file aanmaken
waarin informatie over de in en uitgangssignalen van de simulatie staat.
Selecteer daartoe in het Tools menu van het wave window het commando Make\_list\_file.
Selecteer vervolgens de hotel behaviour beschrijving en klik op OK.
Schrijf vervolgens de file hotel.lst weg.

Vervolgens zullen we een synthese stap doen voor de behaviour beschrijving van 
de hotel entity.
Daarbij wordt de gedragsbeschrijving omgezet in een
netwerkbeschrijving bestaande uit cellen uit de Sea-of-Gates celbibliotheek.
Klik daartoe met de linker muisknop op de configuratie rechthoek van hotel
en selecteer synthesize.
Het synthesizer window zal nu verschijnen.
Klik makeSyntScript en daarna Synthesize.
Eventueel kan op Show circuit worden geklikt om een schema van het 
gesynthetiseerde circuit te zien.
Klik vervolgens op Compile om de \smc{vhdl} beschrijving
te compileren en Parse\_sls om een \smc{sls} circuit beschrijving 
in the backend (\smc{ocean/nelsis}) library te plaatsen.
Er zullen nu rechthoeken voor de gesynthetiseerde \smc{vhdl} beschrijving en
de gesynthetiseerde circuit beschrijving zichtbaar worden in \tool{GoWithTheFlow}.

De gesynthetiseerde \smc{vhdl} beschrijving kan nu ook gesimuleerd worden
met de eerder aangemaakte testbench.
Klik daarvoor op de behaviour beschrijving van de testbench en voeg een
nieuwe configuratie toe.
Geef de nieuwe configuratie een naam die verwijst naar de synthese,
bijvoorbeeld hotel\_tb\_behaviour\_syn\_cfg.
Er zal nu gevraagd worden welke versie van hotel gekozen moet worden 
voor de nieuwe configuratie van hotel\_tb.
Selecteer de gesynthetiseerde versie en klik OK.
Voor de nieuwe configuratie kan nu een simulatie worden opgestart.

Vanuit de gesynthetiseerde circuit beschrijving kan verder een layout gemaakt worden.
Klik daartoe op de rechthoek circuit en selecteer Place \& route.
Het programma \tool{seadali} zal nu gestart worden.
Ga naar het menu automatic tools en voer achtereenvolgens de stappen
plaatsen (m.b.v. \tool{madonna}) en bedraden 
(m.b.v. \tool{trout}) uit.
Na afloop zal een layout zoals in figuur \ref{hotelseadali} te zien zijn.
Schrijf vervolgens de layout weg via het database menu 
(ga terug naar het hoofdmenu via -return-)
en verlaat de layout editor.
\begin{figure}[htbp]
  \centerline{\callpsfig{hotelseadali.eps}{width=13cm}}
  \caption{Een layout voor de hotel schakeling}
  \label{hotelseadali}
\end{figure}

Een alternatief voor Place \& route m.b.v. \tool{seadali}
bestaat uit het eerst plaatsen van de componenten m.b.v. de \tool{row placer}.
Selecteer daartoe op de rechthoek circuit de optie Run row placer.
M.b.v. deze placer kan over het algemeen een betere (compactere) plaatsing
verkregen worden dan m.b.v. \tool{seadali}.
Deze placer kan echter alleen overweg met componenten uit de cellen bibliotheek
(en dus niet met hierarchische ontwerpen waarbij eerder door u ontworpen layouts
als component worden gebruikt)
Verder zal de bedrading altijd nog met \tool{trout} in
\tool{seadali} moeten worden gedaan.
 
De layout kan nu op 2 manieren gecontroleerd worden.
De eerste manier bestaat uit het extraheren van een \smc{vhdl} beschrijving
uit de layout en deze met ModelSim te simuleren.
Klik daartoe met de linker muisknop op de layout rechthoek en selecteer Extract vhdl.
In het nieuwe window, klik Get, vervolgens Write en daarna Compile.
Voeg vervolgens een configuratie toe op de manier
zoals al eerder gedaan.
Maak nu voor de behaviour beschrijving van de testbench voor hotel
opnieuw een nieuwe configuratie waarbij de geextraheerde versie van hotel
geselecteerd wordt en op deze manier kan 
de geextraheerde \smc{vhdl} beschrijving gesimuleerd worden.
Het main window van \tool{GoWithTheFlow} zal nu een beeld zoals in figuur \ref{main-window_end}.
\begin{figure}[htbp]
  \centerline{\callpsfig{GoWithTheFlow_hotelend.eps}{width=12cm}}
  \caption{Het main-window van het programma \tool{GoWithTheFlow} na alle stappen voor het hotel voorbeeld te hebben doorlopen}
  \label{main-window_end}
\end{figure}
\FloatBarrier

Een nog wat nauwkeurige manier van verificatie bestaat uit het simuleren
van een uit de layout geextraheerde transistor beschrijving.
Daartoe moet eerst een geschikte commando file voor de switch-level
simulator gemaakt worden.
Klik op Generate $\rightarrow$ Command file en klik vervolgens 
File $\rightarrow$ Generate from en selecteer
de eerder gemaakte file hotel.lst.
Schrijf de gegenereerde commando file weg met File $\rightarrow$ Write
als hotel.cmd.
Om te simuleren, klik op
de layout rechthoek en selecteer Simulate.
Na op Doit geklikt te hebben zal allereerst een extractie plaatsvinden en
vervolgens een switch-level simulatie.
Klik na afloop op ShowResult om een resultaat te zien zoals in figuur~\ref{slswaves}.
Na inzoomen kunnen hier ook de vertra\-gings\-tijden voor het uitgangssignaal
lamp worden waargenomen.
\begin{figure}[h]
  \centerline{\callpsfig{hotelsimeye.eps}{width=12cm}}
  \caption{Switch-level simulatie resultaat}
  \label{slswaves}
\end{figure}
\FloatBarrier

Hoewel het in dit geval duidelijk te zien is dat de resultaten
van de simulatie op transistor niveau overeenkomen met
de resultaten van de oorspronkelijke \smc{vhdl} simulatie,
kunnen we beide simulatie resultaten ook nog 
door het programma \tool{GoWithTheFlow} laten vergelijken.
We moeten daartoe eerst een simulatie referentie file
(.ref) file aanmaken vanuit de eerder aangemaakte .lst file.
Klik daarvoor op in \tool{GoWithTheFlow} op Generate $\rightarrow$ Reference file.
In het nieuwe window dat verschijnt
klik op File $\rightarrow$ Generate from en selecteer hotel.lst.
Klik vervolgens op File $\rightarrow$ Write
om hotel.ref weg te schrijven en sluit het window af.
Klik daarna op de knop Utilities $\rightarrow$ Compare in
\tool{GoWithTheFlow} om het compare window
te laten verschijnen.
Gebruik
File $\rightarrow$ Read ref
om hotel.ref in te lezen
en 
File $\rightarrow$ Compare res
om hotel.res in te lezen en
deze te vergelijken met hotel.ref.
De uitgangssignalen zullen hierbij vergeleken worden
op tijdstippen vlak voor de volgende opgaande
klokflank.
Als het goed is zullen er alleen afwijkende
uitgangssignalen optreden tijdens de initialisatie periode 
aan het begin van de simulatie (zie figuur \ref{comparewin})
\begin{figure}[h]
  \centerline{\callpsfig{compare.eps}{width=15cm}}
  \caption{Het vergelijken van simulatie resultaten}
  \label{comparewin}
\end{figure}
\FloatBarrier

\clearpage
\subsection{Restricties ontwerpsoftware}
In deze sectie zullen de restricties worden behandeld, die in acht
moeten worden genomen bij het gebruik van de ontwerpsoftware.
\begin{itemize}
\item De 'ports' van entities van \smc{vhdl}-beschrijvingen die moeten worden
      gesynthetiseerd mogen {\it alleen} van het type {\it std\_logic} of
      {\it std\_logic\_vector} zijn.\\
      Dit om te voorkomen dat allerlei conversies moeten worden gemaakt,
      waarbij wellicht niet steeds van dezelfde testbench gebruik kan
      worden gemaakt, en ook conversies van \smc{vhdl} naar sls kunnen mislukken.
\item Gebruik voor de namen van signalen, ports en entities {\it kleine letters}
      Bij de diverse data-omzettingen, die gedurende het ontwerp plaatsvinden
      kunnen anders fouten ontstaan.
\item De namen die worden gebruikt voor terminals en
      circuits dienen onderscheidbaar te zijn in de eerste 14 letters.
\item Voor entities, architectures en configurations moeten aparte files
      worden gemaakt.
\item Geef de layout cellen dezelfde naam als de overeenkomstige circuit cel.
      Dit om bij extractie van
      \smc{vhdl}-code uit het layout-gedeelte naam-problemen te voorkomen.
\item signalen van het type 'inout' mogen alleen worden gebruikt voor
      'analoge' signalen, dus signalen die aan de aangesloten moeten worden
      op een 'direct' buffer aan de rand.
\item Aan ports mogen geen gedeelten van een std\_logic\_vector worden
      aangesloten.
\end{itemize}
De synthese-software voor het genereren van de circuits stelt ook
beperkingen aan de \smc{vhdl}-beschrijvingen.
Naast de algemene aanwijzingen in appendix \ref{good_vhdl} voor het maken van ''synthetiseerbare''
\smc{vhdl}-beschrijvingen noemen we nog de volgende aandachtspunten:
\begin{itemize}
\item Initialisaties van signalen bij hun declaratie worden genegeerd.\\
      Dus bijv. {\it signal a: std\_logic := '0';} wordt gelezen
      als {\it signal a: std\_logic;}.
\item After-clauses in statements worden genegeerd.\\
      Dus bijv. {\it a $<$= '1' after 2 ns;} wordt gelezen als {\it a $<$= '1';}.
\item Process-statements moeten worden gemaakt met een process-list
      waarin, bij een register beschrijving, alleen het klok signaal voorkomt,
      of, bij een beshcrijving van een combinatorische schakeling, {\it alle} 
      signalen voorkomen die tijdens de uitvoering van
      het process worden gelezen.
\item In case- en select-statements moeten {\it alle} mogelijke waarden
      worden behandeld.
\item In iedere tak van case- en if-statements moeten {\it alle}
      uitgangs-signalen een waarde krijgen.
\end{itemize}
Om de tijdvertragingen van de schakelingen te kunnen bekijken is de
nauwkeurigheid van de sls-simulatie default op 100 ps ingesteld.
De maximale tijsduur die dan nog kan worden gesimuleerd, zonder
'overflow'-problemen te krijgen is 100 ms.
{\it Hou hier met het simuleren van de \smc{vhdl} code rekening mee en pas zonodig
de klok-frequentie van de schakeling aan.}

\clearpage

\subsection{Opdrachten ontwerp van een besturing}
\subsubsection{De Ron Brandsteder Lights}

Om t.v.\ shows wat meer aanzien te geven blijkt het noodzakelijk om regelmatig wat
lampen op en rond het podium aan en uit te zetten, waardoor de inhoud van de show
wat minder aandacht behoeft.
De lampen waarom het gaat worden gedacht in een cirkel te staan. Het aantal
lampen is een veelvoud van drie. Elke derde opeenvolgende lamp is met dezelfde
schakelaar verbonden (zie figuur~\ref{lampen}). Hierdoor ontstaan drie groepen
lampen. Elke groep is door een aparte schakelaar aan of uit te zetten.

\begin{figure}[hb]
\centerline{\callpsfig{lampen.eps}{width=0.4\textwidth}}
\caption{Aansluitschema Ron Brandsteder Lights}
\label{lampen}
\end{figure}

U krijgt de opdracht om een besturing voor deze schakelaars te ontwerpen.\\
De besturing moet de volgende mogelijkheden hebben:
\begin{itemize}
\item
Alle lampen staan uit.
\item
De drie groepen worden rechtsom-draaiend na elkaar aan en uit gezet.
\item
De drie groepen worden linksom-draaiend na elkaar aan en uit gezet.
\item
Alle lampen blijven in dezelfde stand staan.
\end{itemize}
De input van de besturing bestaat uit twee stuursignalen s0 en s1.
De output bestaat uit drie outputsignalen L0, L1 en L2, die elk een der
schakelaars bedient.
\clearpage

\subsubsection{De Basket-aanduiding}

Joop van den Beginne heeft voor het nieuwe seizoen weer een nieuw 
televisiespel bedacht. Bij een bepaald onderdeel moeten de deelnemers
een bal in een basketbal-netje gooien.
Als het lukt, gaat er een toeter af. Naast de toeter bevindt zich op 
het bord ook een signaallamp die als het spel begint uit is, gaat
knipperen als \'e\'en keer raak is gegooid en aan gaat als twee keer 
raak is gegooid. De quiz-master kan met een drukknop de schakeling
weer in de begintoestand brengen.


%figuur bus.eps
\begin{figure}[bth]
\centerline{\callpsfig{bus.eps}{width=.3\textwidth}}
\caption{Situatieschets basket}
\end{figure}

Aan de basket-rand is een lichtsluis aangebracht die een logi\-sche '1'
afgeeft als er een bal raak geworpen wordt.\\
U krijgt de opdracht een schakeling te ontwerpen die aan de
gestelde eisen voldoet. Verder geldt er dat:
\begin{itemize}
\item
De aangeboden klok heeft een frequentie die geschikt is om een lamp
te laten knipperen (5 Hz).
\item
Het sensor- en resetsignaal duren \'e\'en klokpuls
en zijn ontdenderd en gesynchroniseerd (met de klok).
\item
De toeter moet gedurende minimaal twee klokpulsen hoorbaar zijn.
\end{itemize}

\clearpage

\subsubsection{De Treinbeveiliging}
Op het trajekt Madurodam-Lutjebroek wordt een nieuw station gebouwd.
Om een goede doorstroom van de sneltreinen te garanderen,
wordt bij de perrons een extra spoor aangelegd. Hierdoor 
blijft het hoofdspoor vrij voor passerende sneltreinen.\\

%figuur trein.eps
\begin{figure}[bth]
\centerline{\callpsfig{trein.eps}{width=1\textwidth}}
\caption{Situatieschets spoorbaan}
\end{figure}

Om ongelukken te voorkomen is ook een seinenstelsel aangebracht.
Er zijn vier seinen geplaatst ($S_{1a}, S_{1b}, S_{2a}, S_{2b}$)
 die worden 
aangestuurd met twee signalen ($S_{1}\ en\ S_{2}$). Als een signaal 
een logi\-sche '1' is, zijn de bijbehorende seinen $S_{na}\ en\ S_{nb}$
respectievelijk geel en rood. In het andere geval zijn ze beide
groen.\\
Om de seinen te kunnen besturen, zijn ook drie detectoren
($d_{1},d_{2},d_{3}$)
aangebracht. Deze geven een logische '1' af als een trein zich in hun
baanvak bevindt.\\
U krijgt de opdracht een schakeling te ontwerpen die ervoor zorgt
dat de seinen een zodanige stand krijgen dat er bij juiste handeling
geen ongelukken kunnen gebeuren. Het ontwerp dient tevens zo te zijn 
dat bij een tegelijk aankomen van zowel een trein op het hoofd- als
zijspoor de trein op het hoofdspoor voorrang krijgt.\\
Verder geldt er dat:
\begin{itemize}
\item
De remweg van een trein is kleiner dan de afstand tussen $S_{na}\ en\ S_{nb}$.
\item
Het is niet mogelijk dat een trein twee detectoren tegelijkertijd activeert.
\item
De treinen rijden op dit baanvak slechts van links naar rechts.
\end{itemize}

\clearpage

\subsubsection{De 'luie' Sluisdeur}
De opdracht is de logica te ontwerpen voor twee deuren 
die als 'anti-tochtschakeling' 
moeten werken, net zoals de achterdeur van het elektro-gebouw. 
De configuratie van de 2 deuren A en B is in onderstaande figuur getekend.

\begin{figure}[bth]
\centerline{\callpsfig{situatie2.eps}{width=0.8\textwidth}}
\caption{Situatieschets sluisdeuren}
\end{figure}

Bij elke deur zijn 3 lichtstraalsensors gemonteerd om de personen te
detecteren. Een sensor geeft een logisch '1' signaal af als de lichtstraal wordt 
onderbroken. 
De logica voor de deurschakeling moet aan de volgende regels voldoen:
\begin{itemize}
\item
In elke situatie is er altijd {\em minimaal} \'e\'en deur gesloten.
\item
De deur kan niet dicht als sensor a3 (of b3) een signaal geeft. Dit
is om nare ongelukken te voorkomen.
\item
Personen moeten zowel van links naar rechts als van rechts naar links
door de deur kunnen. 
\item
Om slijtage aan het deurmechaniek te beperken is de schakeling 'lui',
d.w.z.\ hij gaat niet vanzelf naar de toestand met 2 gesloten deuren terug.
\item
Als tegelijkertijd bij deur A en bij deur B een persoon
aankomt krijgt de persoon bij deur A voorrang.
\item
Bij een reset of bij het aanzetten van de stroom moeten beide 
deuren dicht zijn.
\end{itemize}

\clearpage

\subsubsection{De Personendetector}

Bij veel evenementen blijkt het gewenst om na afloop te weten hoeveel mensen er aanwezig geweest zijn. Om dit te kunnen meten is er een vrij smalle (60 cm) gang gemaakt waar maximaal \'e\'en persoon tegelijk doorheen kan. In deze gang zijn kort achter elkaar twee lichtsluizen (d1, d2) opgenomen die elk weer uit een combinatie van twee infrarood zenders en ontvangers (a, b) bestaan. Als zowel de a als b zender-ontvanger-combinatie onderbroken is, wordt er door de desbetreffende sensor een hoog signaal afgegeven. Signaal d1 wordt dus hoog als zowel d1a als d1b onderbroken zijn.\\
De looprichting is van links naar rechts. De uitgang is een signaal dat ervoor zorgt dat een teller opgehoogd wordt.

\begin{figure}[bth]
\centerline{\callpsfig{persdetect.eps}{width=.8\textwidth}}
\caption{Situatieschets personendetector}
\label{persdetect}
\end{figure}

U krijgt de opdracht een besturing te maken die aan bovengestelde eisen voldoet.\\
Overige voorwaarden:

\begin{itemize}
\item
Een persoon die in tegengestelde richting loopt, mag niet gedetecteerd worden.
\item
Het is niet mogelijk om het tel-signaal op de klok-ingang van de teller aan te sluiten.
\end{itemize}

Facultatief: Maak een schakeling die tevens personen detecteert die van rechts komen en dan een count-down-signaal afgeeft.

\clearpage
\subsubsection{De Garagedeur}

Om niet elke keer uit z'n auto te moeten stappen om de garagedeur open en dicht te doen, wil de heer B. Modaal een garagedeur besturing laten installeren die ervoor zorgt dat zijn garagedeur automatisch opengaat als hij aankomt en sluit als hij vertrekt.\\
Naast de deur is daarvoor een infrarood (IR) sensor (d2) aangebracht die een "hoog'' signaal afgeeft als hij het signaal herkent dat door een zendertje in de auto uitgezonden wordt. Tevens is in de vloer van de garage een metaaldetectorlus (d3) aangebracht waarmee gedetecteerd kan worden of er een auto in de garage aanwezig is. Als laatste is er bij de deur naar de woning nog een drukknop aangebracht waarmee de garagedeur met de hand bediend kan worden.\\


\begin{figure}[bth]
\centerline{\callpsfig{garage.eps}{width=1\textwidth}}
\caption{Situatieschets garage}
\label{garage}
\end{figure}

U krijgt de opdracht een besturing te ontwerpen waarmee met de aanwezige sensoren de garagedeur bediend kan worden.\\
Overige voorwaarden en opmerkingen:

\begin{itemize}
\item
Als de drukknop wordt ingedrukt moet de garagedeur altijd openen of sluiten.
\item
Als de IR-sensor geactiveerd wordt, moet de garagedeur all\'e\'en openen of sluiten als de garage leeg is.
\end{itemize}

\clearpage

\subsubsection{De Frisdrankautomaat}

Ontwerp een besturing voor de afhandeling van het ingeworpen geld in
een frisdrankautomaat.\\ 
De frisdankautomaat kan muntjes van 20 cent en euros verwerken, maar kan niet wisselen.
De prijs van een blikje drank is 1,20 euro.
Een schets van het transportsysteem is gegeven in figuur \ref{geldbox}.

De muntstukken worden na inworp op grootte geselecteerd. De muntjes van 20 cent komen in
het eerste kanaal terecht en worden gedetecteerd door sensor S1. De euros
komen in het tweede kanaal terecht en worden gedetecteerd door sensor S2. 
De sensoren geven na detectie een hoog signaal af ter breedte van minimaal een klokpuls.

Indien er 1 muntje van 20 cent en 1 euro is ingeworpen moet er een 
puls aan een relais R2 worden afgegeven waarmee de frisdrankautomaat 
een blikje vrijgeeft. Dit relaissignaal hoeft slechts 1 klokpuls lang te zijn.
De resetknop zorgt ervoor dat het tot dan toe ingeworpen geld voor een nieuw
blikje weer teruggegeven wordt. Dit wordt bewerkstelligd door een hoog signaal
op een relais R1. Tevens moet relais R1 worden aangestuurd indien er voor een
nieuw blikje 2 muntjes van 20 cent of 2 euros zijn ingeworpen.
D.w.z.\ dat bij foutieve inworp al het ingeworpen geld wordt teruggegeven.
In figuur \ref{geldbox} zijn de relais R1 en R2, die resp. het teruggeven van het geld en het vrijgeven van een blikje besturen, niet weergegeven.\\

\paragraph{Facultatief:} Bedenk een uitbreiding van het systeem waarbij ook
de inworp van 6 muntjes van 20 cent is toegestaan en/of alleen het teveel ingeworpen geld
wordt geretourneerd. De eventueel benodigde extra uitgangssignalen kunnen niet 
naar buiten worden uitgevoerd, maar kunnen in de simulaties wel getest worden.
\begin{figure}[hbt]
\centerline{\callpsfig{frisdrank.eps}{height=0.3\textheight}}
\caption{Schets geld-inwerpsysteem frisdrankautomaat}
\label{geldbox}
\end{figure}

\clearpage
\subsubsection{De Telefoonschakeling}
Een student, die nog niet zo ver is met z'n studie, heeft het volgende probleem:
Hij deelt samen met een huisgenoot de telefoonaansluiting.
Tot nu toe stonden de telefoons parallel aangesloten, maar ze
vinden het vervelend dat ze
door elkaar heen kunnen praten, wanneer ze beiden tegelijk de telefoon opnemen.
Degene die opbelt, begrijpt er dan niets meer van.

\begin{figure}[h]
\centerline{\callpsfig{telefoon2.eps}{width=1\textwidth}}
\caption{Schets schakeling telefoons}
\end{figure}

De studenten hebben bedacht dat ze het probleem met de bovenstaande schakeling
kunnen verhelpen. De telefoons zijn nu beide via een relais
aangesloten op de telefoonlijn.
Daarnaast heeft toestel A een drukknop, waarmee doorverbonden kan worden. 
U krijgt de opdracht om de 'black box' in te vullen met
een schakeling, die moet voldoen aan de volgende eisen:

\begin{itemize}
\item
Wanneer opgebeld wordt, moeten beide telefoons rinkelen.
\item
Wanneer \'e\'en van beiden de telefoon opneemt, wordt het andere
toestel van de lijn afgeschakeld.
\item
Wanneer beiden tegelijk de telefoon opnemen, krijgt toestel~A voorrang.
\item
Als de drukknop wordt ingedrukt, worden beide toestellen met de
telefoonlijn verbonden. Op deze manier kunnen de studenten de opbeller
doorverbinden naar het andere toestel.
\end{itemize}

De input van de schakeling bestaat uit drie signalen.
De signalen A\_bezet en B\_bezet geven een logische '1' af als de hoorn
van het betreffende toestel niet op de haak ligt.
Het signaal druk geeft een logische '1' af als de drukknop 
wordt ingedrukt.
De uitgangssignalen relais\_A en relais\_B zorgen ervoor dat de
toestellen met de telefoonlijn worden verbonden.
Daarnaast hebt u de beschikking over voedingssignalen en een kloksignaal
(niet getekend in de figuur). De schakeling heeft geen apart resetsignaal.        
\clearpage
\cleardoublepage





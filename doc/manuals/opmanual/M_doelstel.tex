\section{Doelstellingen Ontwerppracticum}

Het ontwerppracticum draagt ertoe bij dat de student een groepsproces
ervaart en leert begrijpen. Het accent ligt op het werken naar een
gemeenschappelijk overeengekomen resultaat en het hanteren van de eigen
originaliteit ten behoeve van de groepsactiviteit.
In het ontwerppracticum spelen zowel inhoudelijke (elektrotechnische)
als sociale doelstellingen een rol. De hieronder genoemde doelstellingen
zullen deels een 'leren van', deels een 'kennismaken met' karakter hebben.
\begin{itemize}
\item
{\it Het leren werken in groepsverband.}\\
In de latere beroepspraktijk zal de ingenieur veelal in teamverband aan
een projekt werken. Dit houdt in dat de ingenieur moet kunnen
samenwerken en communiceren met anderen. Dit vereist niet alleen
vakinhoudelijke kennis, maar ook sociale vaardigheden. Dit wil zeggen
een bekwaamheid in het omgaan met mensen in een groep ten behoeve van
een met en door de leden van een groep te verrichten taak. \\
Belangrijke items hierbij zijn o.a.\ het kunnen luisteren naar elkaar, de
eigen idee\"en en gedachten onder woorden brengen, de ander stimuleren
en niet deprimeren, leiding accepteren, leiding kunnen geven, etc.
\item
{\it Het leren werken met randvoorwaarden.}\\
Bij elk projekt waarmee de ingenieur later te maken krijgt zullen
randvoorwaarden een rol spelen. Dit kunnen inhoudelijke randvoorwaarden
zijn (specificaties) alsmede randvoorwaarden t.a.v.\ beschikbare tijd en
geld. In het practicum zal de student met beide aspecten te maken
krijgen. Hij/zij zal de (deel)opdracht goed moeten specificeren en
binnen een strak tijdschema realiseren.
\item
{\it Het leren maken van een eigen ontwerp als onderdeel van het
groepsontwerp.}\\
Opdrachten zullen in het algemeen te groot zijn om alleen of met z'n
twee\"en op te lossen. De ingenieur moet in staat zijn om zo'n groter
probleem op te delen in deelproblemen en de specificaties hiervan vast
te leggen. Vervolgens moet zij/hij in staat zijn om binnen de gestelde
mogelijkheden en afspraken een deelprobleem op te lossen.\\
De opdracht in het practicum is ook van zo'n omvang dat de student
genoodzaakt wordt om het probleem te reduceren tot kleinere
deelproblemen en om tot strakke afspraken te komen waaraan een
deeloplossing moet voldoen. Bij de uitvoering van een deeltaak 
zal hij/zij geconfronteerd worden met de
consequenties van het niet naleven van de afspraken.
\item
{\it Ontwerpgericht denken: analyse als hulpmiddel bij synthese.}\\
De student zal in het ontwerppracticum ontwerpgericht moeten denken.
Het zal zijn/haar taak niet zijn om een bestaand systeem te ontleden,
maar om een nieuw systeem te ontwerpen of eventueel een bestaand
systeem aan te passen. Daarbij is de analyse van het ontwerp een
onmisbaar gereedschap bij de synthese.\\
In het practicum betekent dit dat de student het ontwerp op
verschillende niveaus d.m.v.\ simulatie nauwkeurig zal moeten
analyseren.
\item
{\it Leren systematisch ontwerpen met moderne computerhulpmiddelen.}\\
De studenten leren ontwerpen op een gestructureerde manier. De
systematiek komt sterk naar voren in de hi\"erarchische werkwijze,
waarbij op elk niveau van hi\"erarchie gesimuleerd wordt.
\item
{\it Leren analyseren m.b.v.\ computers op verschillende niveaus.}\\
In het practicum zijn voor de verschillende abstractie-niveaus
simulatieprogramma's beschikbaar: op logisch niveau, switch-level en
transistor niveau. De student zal tijdens het practicum inzicht moeten
krijgen op welk niveau welke simulaties zinvol zijn en welke niet.
\item
{\it Het belang leren onderkennen van modellen.}\\
De student moet het belang leren inzien van het modelleren, dat niet op
elk abstractieniveau alle details meegenomen kunnen worden. Een
model moet zo eenvoudig mogelijk zijn, maar ook weer niet zo eenvoudig
dat effecten die men wil beschrijven niet zichtbaar worden.
\item
{\it Kennismaken met het eigen ontwerp voor en na de fabricage.}\\
In het practicum worden de studenten geconfronteerd met het resultaat
van het eigen ontwerp. Ze leren vaststellen of het produkt geaccepteerd
moet worden volgens de vooraf gestelde criteria.
\item
{\it Aanleren van meetvaardigheid.}\\
De studenten leren omgaan met testprocedures. Ze moeten in staat zijn
om een testplan te genereren en uit te voeren.
\item
{\it Leren maken en gebruiken van documentatie.}\\
De studenten zullen in het practicum documentatie moeten maken waaruit
anderen moeten kunnen opmaken wat de eigenschappen van de schakeling
zijn. Deze documentatie moet ook zelf gebruikt worden bij het testen
van de ontworpen deelschakelingen. Tijdens het practicum moet gebruik
gemaakt worden van gedocumenteerde bibliotheekcellen. 

\end{itemize}

\cleardoublepage

\section{Handleiding Kissis: Automatische Generatie van een Circuit uit een Toestandsdiagram of Functietabel} \index{kissis|(bold}
\label{kissis}
Met het programma \tool{kissis} is het mogelijk om m.b.v.\ een toestandsdiagram of een functietabel een circuit te genereren wat voldoet aan een van te voren
gedefinieerde input/outputrelatie. Hieronder wordt van beide toepassingen een
voorbeeld beschreven. Voor verdere informatie kunt u online de manual page van
\tool{kissis} opvragen (\tool{icdman kissis}).

\subsection{Automatische generatie vanuit een toestandsdiagram}
\index{toestandsdiagram!automatische circuitgeneratie}
Als input voor het maken van een circuit vanuit een toestandsdiagram is een file nodig waarin o.a.\ de toestandsovergangen als functie van de stuursignalen
beschreven zijn.
\tool{Kissis} leest de toestandsovergangen in,
probeert het aantal toestanden te minimaliseren,
kent aan elke toestand een optimale codering toe,
synthetiseert het benodigde logische circuit en beeldt dit af op de bibliotheekcellen.
\tool{kissis} maakt hierbij gebruik van resettable flipflops {\tt dfr11}
om de toestand te onthouden.
Van het logische circuit met geheugen wordt een sls-beschrijving gemaakt,
weggeschreven naar een file en met \tool{csls} toegevoegd aan de database. 
Uiteraard gebeurt dit alles onder behoud van de gedefinieerde
input/output relatie.
Een voorbeeld van zo'n inputfile met bijbehorende toestandsdiagram 
vindt u hieronder.\\
\\
\begin{minipage}[t]{0.4\textwidth}
\begin{verbatim}
.model hotel
.i 2
.o 1
.inputs s ov
.outputs out
.r uit0
-1 uit1 uit1 0
10 uit1 uit1 0
00 uit1 uit0 0
-1 uit0 uit0 0
00 uit0 uit0 0
10 uit0 aan1 0
-1 aan1 aan1 1
10 aan1 aan1 1
00 aan1 aan0 1
-1 aan0 aan0 1
00 aan0 aan0 1
10 aan0 uit1 1
\end{verbatim}
\end{minipage}
\hfill
\begin{minipage}[t]{0.5\textwidth}
Toestandsdiagram:\\
\index{toestandsdiagram!voorbeeld}
\callpsfig{toestand.eps}{width=0.9\textwidth}

\end{minipage}
\\
\\
\\
{\bf Toelichting op de syntax van de inputfile voor kissis.}
\begin{itemize}
\item
De eerste regel begint met het sleutelwoord {\it .model}. 
Daarachter komt de naam van het circuit. Onder deze naam wordt het circuit
opgenomen in de database.
\item
Achter de sleutelwoorden {\it .i} en {\it .o} wordt het aantal inputsignalen 
resp. het aantal outputsignalen opgegeven. Het clocksignaal (CK) en het 
resetsignaal (R) worden hierbij niet meegeteld.
\item
Achter de sleutelwoorden {\it .inputs} en {\it .outputs} worden de namen
van de inputsignalen (inputterminals) resp. de outputsignalen (outputterminals)
opgegeven. Indien deze regels achterwege blijven, maakt 
\tool{kissis} eigen namen.
\item
Indien gewenst kan achter het sleutelwoord {\it .r} de resettoestand 
worden opgeven. Omdat de geheugenwerking (de toestanden) gemaakt wordt 
met resettable flipflops betekent dit dat de resettoestand de 
codering 00..0 krijgt.
\item
Vervolgens worden de toestandsovergangen beschreven. Allereerst de waarden
van de inputsignalen. 
Opgegeven kan worden "1" (hoog), "0" (laag) of "-" (don't care).
Daarna komt de begintoestand en de onder de opgegeven inputwaarden 
opvolgertoestand, gevolgd door de waarden van de outputsignalen. Het aantal inputsignalen en outputsignalen moeten overeenkomen met de achter {\it .i} en {\it .o}
opgegeven waarden. \\
De regel {\bf "00 uit1 uit0 0"} moet dus als volgt worden gelezen : voor {\bf s} is 0 en {\bf ov} is 0 gaat de toestand {\bf uit1} over in toestand {\bf uit0}. De waarde van het outputsignaal {\bf out} behorend bij de begintoestand {\bf uit1} is 0.\\
Omdat alleen Moore-machines zijn toegestaan mogen de outputsignalen niet afhangen van de inputsignalen. Dit houdt in dat de outputsignalen bij een bepaalde begintoestand dezelfde waarden moeten hebben \index{Moore}
(in het voorbeeld is {\bf out} gelijk aan 0 voor de toestanden {\bf uit0} en 
{\bf uit1} en gelijk aan 1 voor de toestanden {\bf aan0} en {\bf aan1}) .
\end{itemize}
Aannemende dat de inputfile {\tt hotel.kiss2} heet,
wordt de generatie van het circuit
gestart met: \type{kissis hotel.kiss2}
De directory test is in dit geval de projectdirectory (aangemaakt met \tool{mkopr}) waarin u het circuit aan de database wil toevoegen.\index{mkopr}\\
Als resultaat ontstaat er een cel {\tt hotel} in de circuit-database, 
een file {\tt hotel.sls} met de sls-beschrijving van de cel {\tt hotel} en de 
codering van de toestanden, en een file {\tt hotel.sta} met een 
{\tt define}-statement van de toestandsvariabele {\bf State}. 
Deze variabele is een combinatie van de uitgangen van de flipflops die
de toestanden coderen.
De file {\tt hotel.sta} wordt bij het simuleren met \tool{sls} ingelezen als in de 
commandfile de regel "{\tt option sta\_file = on}" is opgenomen. Hierdoor wordt in de outputfile van \tool{sls} en in \tool{simeye} de toestand afgedrukt.
\clearpage
{\bf File hotel.sls} :
\begin{verbatim}
/*  This file was automatically created by blif2sls.

    Thu Jul  3 16:42:24 1997
*/

#include<sls_prototypes/oplib.ext>

/*     Latch order:                                          */
/*                          LatchOut_v2 ,                    */
/*                          LatchOut_v3 ,                    */

/*     Code Assignment: uit1  10            */
/*     Code Assignment: the_RS  00            */
/*     Code Assignment: aan1  01            */
/*     Code Assignment: aan0  11            */

network hotel( terminal s, ov, out, R, CK, vss, vdd)
{
 {inst0}    dfr11( n_2359_, R, CK, LatchOut_v2, vss, vdd);
 {inst1}    dfr11( n_2360_, R, CK, LatchOut_v3, vss, vdd);
 {inst2}    iv110( LatchOut_v2, n_2320_, vss, vdd);
 {inst3}    iv110( s, n_2318_, vss, vdd);
 {inst4}    no210( s, ov, n_20_, vss, vdd);
 {inst5}    mu111( LatchOut_v2, LatchOut_v3, n_20_, n_2359_, vss, vdd);
 {inst6}    no210( ov, n_2318_, n_21_, vss, vdd);
 {inst7}    mu111( LatchOut_v3, n_2320_, n_21_, n_2360_, vss, vdd);
            net{LatchOut_v3,out};
}
\end{verbatim}

{\bf File hotel.sta} :
\begin{verbatim}
define  LatchOut_v2 LatchOut_v3 : State  \
            h l : uit1 \
            l l : the_RS \
            l h : aan1 \
            h h : aan0
print State
\end{verbatim}

\clearpage
\subsection{Automatische generatie van combinatoriek m.b.v.\ een functietabel}
Met \tool{kissis} is het ook mogelijk een stukje combinatoriek te genereren aan
de hand van een functietabel. Het circuit bevat dan geen geheugenwerking.
De syntax voor de inputfile is vrijwel gelijk aan hetgeen hierboven is 
beschreven. Alleen de informatie over de toestanden wordt weggelaten en de extensie van de inputfile is nu ".pla". Het voorbeeld hieronder (het maken van een 4-input and) spreekt voor zich. Het genereren van het circuit wordt 
opgestart met: \type{kissis and4.pla}

{\bf Inputfile and4.pla} :
\begin{verbatim}
.model and4
.i 4
.o 1
.inputs A1 A2 A3 A4
.outputs Y
1111 1
---0 0
--0- 0
-0-- 0
0--- 0
\end{verbatim}

{\bf Resultaatfile and4.sls} :\index{sls!voorbeeld}
\begin{verbatim}
/*  This file was automatically created by blif2sls.

    Thu Jul  3 16:47:01 1997
*/

#include<sls_prototypes/oplib.ext>

network and4( terminal A1, A2, A3, A4, Y, vss, vdd)
{
 {inst0}    na210( A1, A2, n_2282_, vss, vdd);
 {inst1}    na210( A3, A4, n_2285_, vss, vdd);
 {inst2}    no210( n_2282_, n_2285_, Y, vss, vdd);
}
\end{verbatim} \index{kissis|)}
\cleardoublepage

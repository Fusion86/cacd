\subsection {NMOS Compound transistor ln3x3}

The NMOS Compoundtransistor ln3x3 consists of 9 NMOS transistors
which are connected in a 3x3 matrix.

Circuit:
%figuur ln3x3.eps
\begin{figure}[h]
\centerline{\callpsfig{ln3x3.cir.eps}{width=.4\textwidth}}
%\caption{}
\end{figure}

Layout:
%figuur ln3x3.lay.eps
\begin{figure}[h]
\centerline{\callpsfig{ln3x3.lay.eps}{width=.5\textwidth}}
%\caption{}
\end{figure}

\clearpage


\subsection {PMOS Compoundtransistor lp3x3}
The PMOS Compoundtransistor lp3x3 consists of 9 PMOS transistors
which are connected in a 3x3 matrix.

Circuit:
%figuur lp3x3.eps
\begin{figure}[h]
\centerline{\callpsfig{lp3x3.cir.eps}{width=.4\textwidth}}
%\caption{}
\end{figure}

Layout:
%figuur lp3x3.lay.eps
\begin{figure}[h]
\centerline{\callpsfig{lp3x3.lay.eps}{width=.5\textwidth}}
%\caption{}
\end{figure}

\clearpage


\subsection {NMOS mirrors mir\_nin en mir\_nout}

The MNOS mirrors are building blocks for a cascoded current mirror.
It consists of the input mir\_nin and the output mir\_nout.
In a mirror, both can be repeated a number of times to achive the
entire circuit.

Circuit:\\
\medskip
\begin{figure}[h]
\centerline{\callpsfig{mir-n.eps}{width=.5\textwidth}}
\end{figure}
(a) mir\_nin (b) mir\_nout
\newpage

\begin{figure}[h]
Layout mir\_nin:\\

\centerline{\callpsfig{mir-nin.eps}{width=1\textwidth}}
\vspace{1cm}
Layout mir\_nout:\\

\centerline{\callpsfig{mir-nout.eps}{width=1\textwidth}}
\end{figure}
\clearpage

\subsection {PMOS mirrors mir\_pin and mir\_pout}
The PNOS mirrors are building blocks for a cascoded current mirror.
It consists of the input mir\_pin and the output mir\_pout.
In a mirror, both can be repeated a number of times to achive the
entire circuit.

Circuit:\\
\medskip
\begin{figure}[h]
\centerline{\callpsfig{mir-p.eps}{width=.5\textwidth}}
\end{figure}
(a) mir\_pin (b) mir\_pout
\newpage
\begin{figure}[h]
Layout mir\_pin:\\

\centerline{\callpsfig{mir-pin.eps}{width=1\textwidth}}
\vspace{1cm}
Layout mir\_pout:\\

\centerline{\callpsfig{mir-pout.eps}{width=1\textwidth}}
\end{figure}

\clearpage

\subsection{bond\_bar}

Functiion: virtual bonding pattern for one quarter of the chip.

Terminals: (t1, t2, t3, t4, t5, t6, t7, t8, t9, t10, t11, t12, 
                  t13, t14, t15, t16, t17, t18, t19, t20, t21, t22, t23, t24, 
                  t25, t26, t27, t28, t29, t30, t31, t32)

Equivalent chip area: 25000

Maximum size of the circuit: 20000

Pins:
\begin{itemize}
\item
2 power pins (vss, vdd)
\item
2 power buffers
\item
32 user-configurable pins (buffered input, buffered output, buffered
bi-directional, unbuffered direct).
\end{itemize}

Four cells of this type can be placed on one chip. The pin positions
connect to an outer ring of interconnect wires, which connects the
pins to the bonding pads on the border of the chip. The pins of one
cell bond\_bar are spread out over the 144 pins in such a way that
they connect to every 4th bonding pad. Each of the 4 virtual chips
has its own vdd power supply, which is also used to drive the
buffers in the bonding pads. In this way the 4 circuits on the chip
are completely independent. All circuits share one common vss ground
supply connection. For larger circuits, multiple instances of
bond\_bar can be used.

%figuur bond_bar.lay.eps
\begin{figure}[bth]
Layout:\\

\callpsfig{bond_bar.lay.eps}{width=1\textwidth}
\end{figure}

\clearpage

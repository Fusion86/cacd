\section{De OP cellenbibliotheek}

\subsection{Inleiding}

In dit hoofdstuk zal een korte beschrijving gegeven worden van de bibliotheekcellen die beschikbaar zijn voor het OP.
De beschrijving bestaat voor elke cel uit:
\begin{itemize}
\item
Functie
\item
Terminal aansluitingen
\item
IEC symbool
\item
Waarheidstabel
\item
Timing parameters
\item
Equivalent chip oppervlak
\item
Fanout
\end{itemize}

Er zijn vier soorten timing parameters:
\begin{tabbing}
xxxxxxxxxxxxxxxxxx\=\kill
$T_{PLH}$ en $T_{PHL}$\> De vertragingstijd van de cel bij eenheidsbelasting (0.12 pF)\\
\\
${\Delta}T_{PLH}$ en ${\Delta}T_{PHL}$\> De vertragings-co\"effici\"ent bij capacitieve belasting\\
\\
$C_{in}$\> De ingangscapaciteit\\
\\
$T_{su}$ en $T_{hold}$\> Setup- en hold-tijden van de flipflops
\end{tabbing}
De vertragings-tijden en -co\"effici\"enten worden afzonderlijk gedefinieerd voor een opgaande (LH) en neergaande (HL) flank van het uitgangssignaal. 
De totale vertragingstijd wordt berekend met de formule:
\begin{description}
\item
$T_{\it P_{tot}} = T_{\it P} + \Delta T_{\it P}(C_{\it load}-C_{\it unit})$
\end{description}
waarbij $C_{load}$ de belastingscapaciteit voorstelt en $C_{unit}$ de eenheidsbelasting.\\
Deze belastingscapaciteit is de som van de ingangscapaciteit van de aangestuurde cellen plus de capaciteit van de verbindingsdraden.
 
De eenheid van chip oppervlakte is gedefinieerd als het kleinst mogelijke stukje van het fishbone image en bestaat uit \'e\'en nmos en \'e\'en pmos transistor. De grootte hiervan is ongeveer 800 $\mu$m$^{2}$.

De fanout is de belasting waarboven de totale vertragingstijd {\it afneemt} als een buffer wordt toegevoegd.\\

\clearpage



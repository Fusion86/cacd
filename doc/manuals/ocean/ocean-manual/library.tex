% -*- latex -*-

\chapter{The sea-of-gates libraries}
\label{s-oplib}
\index{fishbone library|(}
\section{Introduction}
This chapter describes the images and libraries supplied with the
{\smc ocean} system. The 'image' is the basic pattern on a
semi-custom chip. We distribute three types of images:
\begin{itemize}
\item[{\em fishbone}]
\index{image!fishbone}
\index{fishbone image}
A gate-isolation image in a $1.6 \mu$ process with two layers of metal. 
\item[{\em octagon}]
\index{image!octagon}
\index{octagon image}
A remarkable octagonal image in an imaginary $0.8 \mu$ process with three
layers of metal interconnect.
\item[{\em gatearray}]
\index{image!gate array}
\index{gate array image}
An old fashioned gate-array in a single metal layer process.
\end{itemize}
See section~\ref{s-fishstruct} for examples of designs in these
images.  At Delft university we can only process the fishbone
image.  Therefore most of this chapter deals with this image. The
other images are supplied mainly to demonstrate the features of
the placer and the router in the {\sc ocean} system.  They have
only a small cell library which is a subset of the fishbone
library. Also SPICE and SLS simulations will not give realistic
results for {\sl octagon} and {\sl gatearray}.

In the remainder of this chapter we describe the 'fishbone'
Sea-of-Gates library which is used at Delft university of
technology, in the release of august 1993.  The library we supply
is quite small, since it doesn't contain the intricate
combinatorial logic cells. This was done deliberately to keep it
simple for the users. Moreover, in our sea-of-gates style complex
combinational cells can be constructed quite efficiently, so there
is not really a need for such cells in the library.  We split the
libraries according to their function:
\begin{itemize}
\item[primitives:]
Library with the basic description of the Sea-of-Gates image.\\
It contains fishbonec3tu2, Via\_on\_in, Via\_op\_in,
Via\_in\_ins and Error\_Marker
\item[digilib8\_93:]
Library with elementary digital cells.\\
It contains: buf40, de211, de310, dfn10, dfr11, ex210, iv110, mu111
mu210, na210, na310, no210 and no310
\item[analib8\_93:]  
Library with analog cells. It contains: ln3x3, lp3x3, mir\_nin,
mir\_nout, mir\_pin, mir\_pout and osc10.
\item[bonding8\_93:]
Library with elementary bonding patterns, containing bond\_leer,
bond\_bar and bondflap\_8.
\item[exlib8\_93:]
The 'expert' or 'extended' library, containing a few more cells
which are not (yet) intended for practicum users. This library
also contains the old cells which were kicked out of the oplib.
The cells dfr112, dfn102 are the old (and smaller) flipflops.
Contains currently: buf20, buf100, dfr112, dfn102, add10, itb20
\end{itemize}

If you use \tool{mkopr}, 
\index{mkopr@\tool{mkopr}}
the following libraries are imported into the
new project: 'primitives', 'digilib8\_93', 'analib8\_93' and
'bonding8\_93'.

For each library there is a file with the 'external' sls
descriptions. they are useful to help you with inserting the
proper terminal order in your sls circuit files. The fiels with
the external definitions are in the directory 'sls\_prototypes'
\index{slsprototypes@\fname{sls\_prototypes}}
of your project directory.

If you use the command \tool{mkepr} 
\index{mkepr@\tool{mkepr}}
(notice the 'e'), also the
extended library will be imported into the new project. \tool{mkepr}
Imports the following libraries: 'primitives', 'digilib8\_93',
'analib8\_93', 'bonding8\_93' and 'exlib8\_93'.

At the end of this chapter we include some information about the
image description file 'image.seadif' which is required by all
\smc{ocean} programs.

\section{Information of each library cell}

The names of the digital cells were chosen using the convention
which was used by Philips for its gate-array packages. The first
three characters describe the function of the cell. Sometimes the
number of inputs is included in the names: 'na210' is a 2-input
nand, while 'na310' is a 3-input nand. The last 2 digits describe
the fan-out of the output. But sometimes I don't understand this
convention myself.

The description of each cell consists of:
\begin{itemize}
\item
Function 
\item
Terminal connections
\item
IEC symbol
\item
Truth table
\item
Parameters to determine the delay
\item
Equivalent chip area
\item
Fanout
\end{itemize}

The parameters for the circuit delay consist of four parts:
\begin{tabbing}
xxxxxxxxxxxxxxxxxx\=\kill
$T_{PLH}$ en $T_{PHL}$\> the fixed delay times of the cell under
unit load (0.12pF).\\
\\
${\Delta}T_{PLH}$ en ${\Delta}T_{PHL}$\> The delay coefficients with a
capacitative load.\\
\\
$C_{in}$\> The input capacitance.\\
\\
$T_{su}$ en $T_{hold}$\> Setup- and hold-times of the flipflops.
\end{tabbing}

The delay times and delay coefficients are specified for the
rising (LH) as well as the falling (HL) edge of the output signal.
The total delay can be calculated using the formula:
\begin{description}
\item
$T_{\it P_{tot}} = T_{\it P} + \Delta T_{\it P}(C_{\it load}-C_{\it unit})$
\end{description}
in which $C_{load}$ is the load capacitance and $C_{\it unit}$ the
unit load. This load capacitance
is the sum of the input capacitances of the cells which are driven
plus the capacity of the interconnection wires.  The unit of chip
area is defined as the smallest piece of the fishbone image, which
consists of one nmos and one pmos transistor.
\clearpage

\section{digilib8\_93: the basic digital library}
\input celllib/iv110.tex
\input celllib/no210.tex
\input celllib/no310.tex
\input celllib/na210.tex
\input celllib/na310.tex
\input celllib/ex210.tex
\input celllib/buf40.tex
\input celllib/mu111.tex
\input celllib/mu210.tex
\input celllib/de211.tex
\input celllib/dfn10.tex
\input celllib/dfr11.tex
\section{analib8\_93: the analog library}
\input celllib/anabib.tex
\input celllib/osc10.tex
\section{bonding8\_93: the bonding patterns}
\input celllib/bond_leer.tex
\input celllib/bond_bar.tex
\section{exlib8\_93: the extended library}
\input celllib/buf20.tex
\input celllib/dfn102.tex
\input celllib/dfr112.tex

\index{fishbone library|)}
\newpage
\pagestyle{plain}
\section{The file 'image.seadif'}
\index{image.seadif@\fname{image.seadif} file|bold}
Process information is stored in the \smc{nelsis} system in the
directory '\$ICDPATH/lib/process'.  The \smc{ocean} tools read the
information about the image from a file called 'image.seadif',
which is present in '\$OCEANPATH/celllibs/\$OCEANPROCESS'. In some
cases you might want to modify this file to change the behaviour
of the tools. If you copy an 'image.seadif' in your project
directory, it will automatically be read by all tools.

The syntax of the image description file 'image.seadif' is
lisp-like and easy to interpret. Below we include the
'image.seadif' file of the {\sl fishbone} image.

{\small
\begin{verbatim}
/* 
 *
 * This is the file 'image.seadif'
 * Contents: Image description for the the 'fishbone' image. The fishbone
 *           image was developed at Delft University of Technology by
 *           Patrick Groeneveld and Paul Stravers. It requires the
 *           c3tu process, which is a derivative of the 1.6 micron C3DM
 *           process from Philips.
 * Purpose:  All OCEAN tools read this file to get process and image 
 *           specific information. This file is read by fish, sea nelsea, 
 *           ghoti, madonna, trout and seadali. You may edit this file to 
 *           change the behaviour of (for instance) the router. In that 
 *           case you can copy a version of this file into your process 
 *           directory. All OCEAN tools will read your local image 
 *           description file in that case.
 * Created:  by Patrick Groeneveld and Paul Stravers
 *           march 1993
 *
 * (c) The OCEAN/NELSIS Sea-of-Gates design system 
 * Delft University of Technology
 *
 */

(Seadif "Image description file for fishbone image"

   /*
    * Set the size of an empty image if you press 'fish'
    * for fishbone 25 times 1 is nice
    */
   (ChipDescription chip
      (ChipSize 25 1)
   )
 
   (ImageDescription "fishbone"     /* name is important */
      (Technology "fishbone"        /* nelsis technology name */
         (DesignRules
            /* 
             * Specify data about routing layers. First metal layer
             * (which is closest to the solicon) has index 0:
             *    -1  <- polysilicon/diffusion
             *     0  <- metal 1
             *     1  <- metal 2
             *     2  <- metal 3
             *    etc.
             * Take as many layers as you want or have.
             * Layer -1 contains all feautues of the iamge which cannot 
             * be edited by the user
             */
            (NumberOfLayers 2)
            /* set wire width of the layer in lambda */
            (WireWidth 0 12)
            (WireWidth 1 12)
            /* specify which mask corresponds to each layer */
            /* every layer MUST have one unique mask assigned */
            (WireMaskName 0 "in")       /* layer0 name = in */
            (WireMaskName 1 "ins")      /* layer1 name = ins */
            /* dummy layer for markers */
            (DummyMaskName "bb")        /* bb mask as no-op */
            /* set style of the layers for trout, must be alternating */
            (WireOrient 0 horizontal) 
            (WireOrient 1 vertical)

            /* 
             * Declare the vias. For each via there must be an
             * instance containing the mask pattern. Between the
             * metal layers there is one via per adjacent pair of
             * layers. Between metal 1 and the silicon (poly,
             * diffusion, etc) there may be more than one type of via.
             */
            (ViaCellName  0 1 "Via_in_ins")  /* between metal1 and metal 2 */
            (ViaCellName -1 0 "Via_ps_in")   /* polysilicon and metal 1 */
            (ViaCellName -1 0 "Via_on_in")   /* n-diffusion and metal 1 */
            (ViaCellName -1 0 "Via_op_in")   /* p-diffusion and metal 1 */

            /*
             * specify the mask name associated with each via. In this way
             * fish can map a mask pattern to the proper via. Any
             * of the contacts from metal1 down to the image is
             * translated to the proper mask pattern, using the feed
             * statements in the GridConnectList
             */
            (ViaMaskName  0 1 "cos")         /* between metal1 and metal 2 */
            (ViaMaskName -1 0 "cps")         /* polysilicon and metal 1 */
            (ViaMaskName -1 0 "con")         /* n-diffusion and metal 1 */
            (ViaMaskName -1 0 "cop")         /* p-diffusion and metal 1 */
         )

         /*
          * set the spice parameters, which allow ghoti to collapse
          * parallel transistors into one transistor. (not so relevant)
          */
         (SpiceParameters
            (Model "nenh"
               (Parameter "ld"  "0.325e-6") /* ghoti -r needs this parameter */
               (Parameter "vto" "0.7"))
            (Model "penh"
               (Parameter "ld" "0.300e-6")
               (Parameter "vto" "-1.1")))
      )

      /*
       * The next block describes how the grid of a single core cell
       * looks like
       */
      (GridImage

         /*
          * declaration of the image grid. 
          * this image is a rectangular grid which is repeated 
          *
          * Set the size of a single grid cell in grid points 
          * in fishbone this is 1 (x) by 28 (y), which means that
          * the gridpositions range from 0,0 to 0,2
          */
         (GridSize  1 28
            /*
             * Specify how the gridpoints map to the layout coordinates
             * In this case, for instance, the grid point (0, 1) will 
             * map to point (26, 64) lambda in the actual image. In this 
             * way it is possible to specify a non-regular grid.
             */ 
            (GridMapping horizontal    /* 1 point */
                44) 
            (GridMapping vertical      /* contains 28 points */
                17  56  96 128 160 192 232 264 304 336 368 400 432 472
               511 550 590 622 654 686 718 758 790 830 862 894 926 966)

            /*
             * tearlines of the image (currently disabled, not so relevant)
             *
            (TearLine vertical 2
               0 13)
            (TearLine horizontal 1
               0)
             */

            /* 
             * Specify the implicit power lines which are present in
             * the image. This is used by the router.
             * format: 
             * (PowerLine <orient> <ident> <layer_no> <row/column number>) */
            (PowerLine horizontal vss 0 0 ) 
            (PowerLine horizontal vdd 0 14) 

            /* overlap of the repeated image in x and y direction: */
            (ImageOverlap 0 1)
         )

         (Axis
            /* 
             * Mirror axis information for the detailed placer. Format is
             * (MirrorAxis x1 x2 y1 y2) where (x1,y1) and (x2,y2) are two
             * coordinates thru which the mirror axis runs. NOTE: because
             * a mirror axis can run just IN BETWEEN two gridpoints, the
             * arguments of MirrorAxis specify HALF GRIDPOINT UNITS. E.g.
             * a mirror axis running thru the gridpoints (1,1) and (5,5)
             * would be specified as (MirrorAxis 2 10 2 10).
             * used by madonna.
             */
            (MirrorAxis 0 1 28 28)              /* mirror around vdd */
            (MirrorAxis 0 1 0 0 )               /* mirror around vss */
         )

 
         /*
          * The next block specifies the connections/feeds 
          * between grid points.
          */
         (GridConnectList
            /*
             * Default vias are allowed at EACH grid point between metal
             * layers (layer no >= 0) and at NO gridpoint to the core (from
             * layer 0 to -1).  Using the 'block' statement it is possible to
             * remove edges from the graph and therefore also possible vias.
             * In the image of the example no 'cos' vias are allowed at 
             * some positions.
             */
            /* means: no connection between pos (0,1,0) and (0,1,1) 
             *  = no via allowed there */
            (Block 0 1 0 0 1 1) 
            (Block 0 6 0 0 6 1) (Block 0 7 0 0 7 1)
            (Block 0 13 0 0 13 1)
            (Block 0 15 0 0 15 1)
            (Block 0 21 0 0 21 1) (Block 0 22 0 0 22 1)
            (Block 0 27 0 0 27 1)
               
            /*
             * Declare the internal feeds of the image (e.g. feeds
             * through a poly wire).  This declaration also implies
             * that it is possible to make a via at a certain grid
             * position. Therefore the cell name of the required via
             * must be given at each feed position.  The via cell name
             * must be declared previously in the technology block
             * using the 'ViaMaskName' statement.  We recoqnize two
             * types of feeds: UniversalFeed and RestrictedFeed.  For
             * fish there is no difference, but for the router there
             * is.  The router can use universal feeds for arbitrary
             * wire segments.  Restricted feeds have a limited use:
             * they can only indicate that a set of gridpoints (in the
             * image = layer -1) are electrically equivalent.  In the
             * example image the the poly gates can serve as universal
             * feeds, while the diffusion source and drain cannot.  A
             * feed position can also point to a position outside the
             * size of the image cell.  Fish does not use the feed
             * information, but it needs a via cell name for each
             * position where a connection to layer -1 can be made.
             * It is allowed to add dummy feeds: e.g.
             *    (UniversalFeed (Feed "Via_ps_in" 0   1))
             * indicates that at 0,1 a via to core can be placed, with
             * name Via_ps_in.  ExternalFeed - if this exists then
             * this feed has an extension to the neighbor cluster in
             * this direction ("hor" or "ver")
             * BusFeed - power or ground net
             *
             */
            /* e.g. points (0,1) and (0,6) are equivalent */
            (UniversalFeed (Feed "Via_ps_in" 0   1) (Feed "Via_ps_in" 0   6))
            (UniversalFeed (Feed "Via_ps_in" 0   7) (Feed "Via_ps_in" 0  13))
            (UniversalFeed (Feed "Via_ps_in" 0  15) (Feed "Via_ps_in" 0  21))
            (UniversalFeed (Feed "Via_ps_in" 0  22) (Feed "Via_ps_in" 0  27))

            (RestrictedFeed (Feed "Via_on_in" 0   2) (Feed "Via_on_in" 0   3)
               (Feed "Via_on_in" 0   4) (Feed "Via_on_in" 0   5))
            (RestrictedFeed (Feed "Via_op_in" 0   8) (Feed "Via_op_in" 0   9)
               (Feed "Via_op_in" 0  10) (Feed "Via_op_in" 0  11) 
               (Feed "Via_op_in" 0  12))
            (RestrictedFeed (Feed "Via_op_in" 0  16) (Feed "Via_op_in" 0  17)
               (Feed "Via_op_in" 0  18) (Feed "Via_op_in" 0  19) 
               (Feed "Via_op_in" 0  20))
            (RestrictedFeed (Feed "Via_on_in" 0  23) (Feed "Via_on_in" 0  24)
               (Feed "Via_on_in" 0  25) (Feed "Via_on_in" 0  26))

            (RestrictedFeed (Feed "Via_op_in" 0 0) (ExternalFeed "hor"))
            (RestrictedFeed (Feed "Via_on_in" 0 14) (ExternalFeed "hor"))

            (BusFeed (Feed "Via_op_in" 0 0) (ExternalFeed "hor"))
            (BusFeed (Feed "Via_on_in" 0 14) (ExternalFeed "hor"))
         )

         /*
          * The cost parameters to steer the router.
          */
         (GridCostList
            /* 
             * The format is:
             *     <cost> <vector> <startpoint_range> <endpoint_range>
             *
             * GridCost specifies the cost of a metal wire or a via
             * The offset <vector> must be in the unity coordinate system:
             *    1 0 0  = horizontal metal wire to the right
             *   -1 0 0  = horizontal metal wire to the left
             *    0 0 1  = a via to the layer above this layer
             * example:
             *     (GridCost 1   1 0 0  0 0 0  0 27 0) 
             * gives offset vector (1 0 0) = L cost 1 over range 
             * (0 0 0) to (0 27 0) = entire layer 0
             *
             * FeedCost has the same syntax format, but it specified the
             * cost of a feed through the image (a non-metal layer) PLUS
             * the cost of the two vias associated with that feed.
             * The range is implicitly over the image layer (-1).
             * 
             * Tuning the costs may take some experimenting, some 
             * thinking and just some common sense. If you need any help, 
             * contact us at space-support-ewi@tudelft.nl
             */
            /* horizontal cost in layer 0 */
            (GridCost 1   1 0 0  0 0 0  0 27 0)
            (GridCost 1  -1 0 0  0 0 0  0 27 0)
 
            /* horizontal cost over vias in layer 0 
             * (higher cost over poly vias) */
            (GridCost 3  1 0 0  0  0 0  0  1 0)
            (GridCost 3  -1 0 0  0  0 0  0  1 0) 
            (GridCost 3  1 0 0  0  6 0  0  7 0)
            (GridCost 3  -1 0 0  0  6 0  0  7 0)
            (GridCost 3  1 0 0  0 13 0  0 15 0)
            (GridCost 3  -1 0 0  0 13 0  0 15 0)
            (GridCost 3  1 0 0  0 21 0  0 22 0) 
            (GridCost 3  -1 0 0  0 21 0  0 22 0)
            (GridCost 3  1 0 0  0 27 0  0 27 0)
            (GridCost 3  -1 0 0  0 27 0  0 27 0)

            /* vertical cost in layer 0 */
            (GridCost 5  0 1 0  0 0 0  0 27 0) 
            (GridCost 5  0 -1 0  0 0 0  0 27 0)  
            /* vertical cost over vias in layer 0 */
            (GridCost 3  0 1 0  0  0 0  0  1 0) 
            (GridCost 3  0 -1 0  0  0 0  0  2 0)
            (GridCost 3  0 1 0  0  5 0  0  7 0) 
            (GridCost 3  0 -1 0  0  6 0  0  8 0)
            (GridCost 3  0 1 0  0 12 0  0 15 0) 
            (GridCost 3  0 -1 0  0 13 0  0 16 0)
            (GridCost 3  0 1 0  0 20 0  0 22 0)
            (GridCost 3  0 -1 0  0 21 0  0 23 0)
            (GridCost 3  0 1 0  0 26 0  0 27 0)
            (GridCost 3  0 -1 0  0 27 0  0 27 0)
            /* horizontal cost in layer 1 */
            (GridCost 10  1 0 0  0 0 1  0 27 1) 
            (GridCost 10  -1  0 0  0 0 1  0 27 1) 
            /* vertical cost in layer 1 */
            (GridCost 1  0 1 0  0 0 1  0 27 1)  
            (GridCost 1   0 -1 0  0 0 1  0 27 1)
            /* cost of a via between layer 0 and layer 1 */
            (GridCost 5  0 0 1  0 0 0  0 27 0) 
            (GridCost 5  0 0 -1  0 0 1  0 27 1)

            /* cost of a tunnel in poly */
            (FeedCost 1000  0 5 0  0 0 0  0 27 0) 
            (FeedCost 1000  0 -5 0  0 0 0  0 27 0)
            (FeedCost 1000  0 6 0  0 0 0  0 27 0) 
            (FeedCost 1000  0 -6 0  0 0 0  0 27 0)
            /* diff tunnels */
            (FeedCost 80 0 1 0  0 0 0  0 27 0) 
            (FeedCost 80  0 -1 0  0 0 0  0 27 0)
            (FeedCost 3  0 2 0  0 0 0  0 27 0) 
            (FeedCost 3  0 -2 0  0 0 0  0 27 0)
            (FeedCost 3  0 3 0  0 0 0  0 27 0) 
            (FeedCost 3  0 -3 0  0 0 0  0 27 0)
            (FeedCost 3  0 4 0  0 0 0  0 27 0) 
            (FeedCost 3  0 -4 0  0 0 0  0 27 0)
         )
      )

      /*
       * describe the actual layout of the image cell
       */
      (LayoutImage      
         /*
          * The cell is just imported as a model-call.
          */
         (LayoutModelCall "fishbonec3tu2")
         /* 
          * the repetition distance in x is 40 lambda,
          * the repetition distance in y is 988 lambda.
          */
         (LayoutImageRepetition 40 988)
      )
   )
)

\end{verbatim}
}

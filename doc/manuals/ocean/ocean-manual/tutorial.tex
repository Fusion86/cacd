% -*- latex -*-

\chapter{A quick tutorial}
\label{tutorial}
\index{tutorials|(}
\section{Let's get behind the terminal}
This chapter takes you for a guided tour along the {\sc ocean} tools.
Using two examples you will get accustomed to operations like circuit
input, circuit simulation, layout synthesis, and layout verification.  This
tutorial contains three exercises to introduce you to the {\sc ocean}
sea-of-gates layout tools and to the {\sc nelsis} tools for circuit
extraction and simulation.

The first exercise (section~\ref{s-nor3}) deals with the design of a simple
library cell, a 3-input {\sc nor} gate. It shows a powerful aspect of the {\sc
ocean} environment, namely that you can get very close to the sea-of-gates
image to optimize your design for area or speed. You do not depend on a cell
library because you can very easily design a cell library on your own.

The second example (section~\ref{s-hotel}) shows the hierarchical approach of
the {\sc ocean} tools. In contrast to most sea-of-gates systems, {\sc ocean}
does not require a flat circuit description of the design. The hierarchy
present in the designer's circuit is retained in the sea-of-gates layout.

The third example deals with using the logic synthesis tool \smc{sis} with the
\smc{ocean} system. It shown how easy you can translate the description of a
Finite State Machine (FSM) \index{FSM} into a Sea-of-Gates layout.

The examples are intended to be self-contained and can be dealt with
independent from each other. For instance, you can immediately skip to
section~\ref{s-hotel} since it does not assume that you have read
section~\ref{s-nor3}. We advice, however, to do them in the indicated order.

The examples use the {\sl fishbone} image. You can also run them in the other
two images. This will be explained in section~\ref{s-advanced}.

We distinguish four major steps in the design cycle:
\begin{enumerate}
\item
Creating the circuit level
\item
Simulating the circuit on circuit-level.
\item
Creating layout from the circuit.
\item
Extraction and simulation of the layout.
\end{enumerate}
Keeping this design flow is crucial for the hierarchical design strategy. In
figure~\ref{oceantools} on page~\pageref{oceantools} an overview of the system
and its design flow is given. To the right the synthesis tools are displayed
(steps 1 and 3), while the analysis tools can be found on the left (steps 2 and
4).

\section{Designing a simple library cell}
\label{s-nor3}
The purpose of this section is to get you accustomed to circuit
design on the 'fishbone'-image using the \smc{ocean} tool suite.
We will start by designing the simple 3-input nor which is shown in
figure~\ref{nor3}. Normally you'll import such a simple cell from the
library and you don't have to bother so much making wires by hand. It
is very illustrative, however, to take you through all design steps
using a simple example. 

\subsection{Before we start: making a project}
\index{tutorial tool@\tool{tutorial} tool}
In \smc{ocean} we perform the design process in a 'project'\index{project}.
This is nothing other than a \smc{unix} directory tree which contains a
database.  For the tutorial we have a simple command to create the project.
Type\footnote{The string \fname{/user/hillary~\%~} is the prompt, so don't type
it!}: 
\cd{}
\typeb{tutorial hotel}{cd hotel} \cd{/hotel}
This puts you in a project directory that contains all the data
related to both examples which we use.  You must start all programs
of \smc{ocean} and \smc{nelsis} in a project directory.

\index{mkopr@\tool{mkopr}}
\attention{How to make an empty project?}{
The command \tool{mkopr $<$name$>$} creates an empty project for you. This
project automatically imports all library cells. See section ~\ref{mkopr} for
more details.}
\subsection{Creating a circuit description}
\index{creating!circuit descriptions}
\begin{figure}
\centerline{\psfig{figure=\figdir/nor3.eps,width=0.5\linewidth}}
\caption{The schematic of the 3-input nor circuit.}
\label{nor3}
\end{figure}
\begin{itemize}
\item
In figure~\ref{nor3} the schematic is already given. Normally you'd
have to draw the schematic yourself, or create it using other tools.
In this case the schematic of the 3-input nor is very simple: It
consists of 3 parallel n-enhancement MOS-transistors (bottom of
figure~\ref{nor3}) and 3 p-enhancement MOS-transistors in series (on top).
\item
We describe the circuit in the \file{sls}-circuit
\index{sls!language syntax}
language\footnote{At
the moment we do not yet have a stable schematic editor to perform
the task of schematic entry. You'll soon notice, however, that the
\file{sls}-language is powerful enough to describe anything you
want.}.
First we have to identify the terminals:
\begin{itemize}
\item[input terminals:]
Our nor has 3 inputs: \terminal{A}, \terminal{B} and \terminal{C}.
\item[output terminal:]
The output is tagged \terminal{F}.
\item[power terminals:]
Power connections must be called \terminal{vss} and \terminal{vdd}.
\item[internal terminal:]
This circuit has two internal nodes which we called \terminal{v1} and
\terminal{v2}. You can give them any name you want.
\end{itemize}
It would take a lot of words to describe the \file{sls}-syntax, so
let's just look at the example file in figure~\ref{tut-nor3}.
\begin{figure}
\begin{verbatim}
network nor3 (terminal A, B, C, F, vss, vdd)
{
    penh w=29.6u l=1.6u (A, v1, vdd);
    penh w=29.6u l=1.6u (B, v1, v2);
    penh w=29.6u l=1.6u (C, v2, F);
    nenh w=23.2u l=1.6u (A, F, vss);
    nenh w=23.2u l=1.6u (B, F, vss);
    nenh w=23.2u l=1.6u (C, F, vss);
}
\end{verbatim}
\caption{\label{tut-nor3}
The \protect\file{sls} circuit description of the 3-input nor.}
\end{figure}
The first line declares the network 'nor3' and its (external)
terminals\index{terminals}\footnote{also called the
'ports'\index{ports|see{terminals}} of the circuit}. Between the parenthesis
follows the description: 6 transistors.  Each line declares one instance
(in this case a transistor) and the way it is connected. The type, width
and length of the transistors must be specified. Between the brackets are
its terminal connections. For the transistor the order of the connections
is {\em gate}, {\em source} and {\em drain}.  The first transistor, for
example, is a p-type enhancement which has its gate connected to
\terminal{A}, its source to node
\terminal{v1} and its drain to \terminal{vdd}.

You are lucky. The file containing this sls-description already
exists in the tutorial directory. It is called \file{nor3.sls}. You
can check its contents using your favorite editor.

\item
The circuit description must be entered into the Nelsis database.
This is done using the program \tool{csls}\index{csls@\tool{csls}}:
\type{csls nor3.sls}
In this way the contents of the file is translated into the binary
circuit format of the \smc{nelsis} database. 
You can check whether the circuit is there:
\type{dblist}
This command lists the contents of the current database.
\end{itemize}

\subsection{Simulating your circuit}
\label{simcircuit}
\begin{itemize}
\item
Start the simulation interface\index{simeye@\tool{simeye}}::
\type{simeye \&}
A window will appear with many fancy buttons. Move the mouse to the
widget \button{cell:~~~~~} (on the left top). Type there the name of
the circuit ('nor3').

\item
Just start simulating! Click \button{sls-logic} and then
\button{run} to perform logic simulation.
\index{simulation!logic level}
After a short while the simulation results will appear in the window. Check
that the circuit really implements a 3-input nor.

Just play around a bit with the buttons. You can zoom in and out
(\button{in}, \button{out} and \button{full}) and using the arrow
keys you can pan the window.

\item
The simulation requires a 'stimuli' file in which the input voltages and other
simulation commands are specified.  
\index{sls!stimuli file|see{sls command file}}
\index{sls!command file}
The tutorial directory already contains a
proper stimuli-file for this example (called
\file{nor3.cmd}). You can modify the input stimuli in this file, or
graphically in \tool{simeye}. In a later section we will show how you
can do this yourself. Let's first concentrate on the basics.
\item
There are 3 simulators available from \tool{simeye}: 
\index{simulation!on three levels}
a logical simulator, a switch-level simulator and the \tool{spice}
simulator. In this way you can simulate at three levels of accuracy.

The button \button{sls-logic} (which you just pressed)
\index{simulation!logic level}
performs a
logical simulation of the circuit. It assumes that the transistors
are ideal switches and that the network has no parasitics. Therefore
there is no delay in the circuit, and the rising and falling edges
are sharp.

\item
Now try the button \button{sls-waves}, followed by \button{run}.
\index{simulation!switch-level}
This performs a switch-level simulation of the circuit. The
switch-level simulator uses a simple model for the parasitics and
gives a reasonable accurate timing result. Check that the output
signal is now (slightly) delayed.

\item
Finally, try \button{spice} 
\index{simulation!spice-level}
\index{spice@\tool{spice}}
for the most accurate simulation of the
circuit.  Notice that the output waveforms are now continuous, and
look much more like the real-life signal. You will notice that it
will take much more time before the results show up on the screen.
\tool{Spice} is notorious for its load.  From our experience you can
only simulate networks up til several hundred transistors using \tool{spice}.

\item
The small buttons labeled \button{A} and \button{D} 
are toggles that either
cause an ``analogue'' or a ``digital'' interpretation of the signals to be
displayed. \tool{Spice} only allows an analogue interpretation of course.

\item
For now, we've completed the circuit simulation. Iconify the window
and get ready for the next step.
\end{itemize}

\subsection{Creating a layout}
At the hart of the layout generation is \tool{seadali}. 
\index{seadali@\tool{seadali}}
It allows you to create
layout manually, automatically or both.  Chapter~\ref{s-seadali} describes
\tool{Seadali} in greater detail. Here we only outline the minimal commands to
get started.
\begin{itemize}
\item
\index{seadali!starting}
Start \tool{seadali}: \type{seadali \&} 
Click \button{boxes} to enter the box manipulation menu. 
\item
Click \button{fish} to create an empty piece of image. In fact,
you are looking at a small ($20 \times 4$ transistors) portion of
the sea-of-gates chip. Have a good look at the structure. Play around
with \button{visible} (in the main menu) to activate masks. The
bigger transistors (along the middle power rail) are p-type. The
smaller ones are n-type. The white dots denote the grid points.
\item
Now let's think about the layout of our nor3. Try to get some inspiration from
the way in which the nand-2 circuit was implemented (figure~\ref{f-nand2} on
page~\pageref{f-nand2}). The nor-3 is the reverse of the nand-2.  We need 3
n-type transistors (from the bottom row) in parallel. We also need 3 p-type
transistors in series.
\item
Enough philosophy, let's click!. First enable the layer in which you
want to add wires by switching the bulb \button{in} (= metal1 mask
code) on. It is the dark blue one in the bottom row of the window. 
Click \button{APPEND} (in the boxes menu) and try to make some wires.
Start, for instance, in the left-bottom by making a vertical wire of
2 grid points long. The bottom of that wire should touch the lower
power rail.

\item
Add the individual wires. If you want, you can also use the second
metal layer (\mask{ins}, light blue) to cross a wire in \mask{in}
the first layer.
\item
Contacts to the transistors are made in the \mask{cps}-mask. 
\index{creating!contacts}
Click
\button{cps} and disable the other masks to add them. The contacts
are a bit harder to distinguish on the screen because they are black.

You can make contacts between metal1 (\mask{in}) and metal2
(\mask{ins}) by adding boxes in the \mask{cos}-mask. These contacts
are white. Do not stack \mask{cos} contact on top of \mask{cps}
contacts!
\item
Now press \button{fish} to check and purify the layout.
\tool{Fish} will make the layout design-rule correct by rounding all
the wires to the nearest grid points and substituting the proper
contact patterns. It will also warn you for errors, such as stacked
contacts or \mask{cos} contacts on top of a poly gate. You can press
\button{fish} any time and as often as you want. Just try it.

\item
To isolate the cell from its surroundings we add an isolation
transistor on the right side. Just connect the first unused n-type
and p-type transistors on the right to their nearest power line. The
gate of the n-type transistor must be connected to \terminal{vss}, the
gate of the p-type to \terminal{vdd}.  Do not forget the contacts, and
press \button{fish} as often as you want.

\item
Once the layout looks like it is really going to work it is time to add the
terminals. Click \button{return}, \button{terminals} and
\index{seadali!adding terminals}
\index{creating!terminals in layout}
\index{terminals!adding to layout}
\button{add terminal} and switch only the \mask{in}-mask on. Click at the
terminal position\footnote{This must be an \mask{in}-wire or an \mask{ins}
wire, depending on which mask was activated. Other masks cannot be used for
terminals} and enter the terminal name. Give the terminal the same name as in
the circuit description. Only add the external terminals (\terminal{A}, \terminal{B},
\terminal{C}, \terminal{F}, \terminal{vss} and \terminal{vdd}). Put the power
terminals on a spot on the power rails.

\item
Write the layout in the database under the name \fname{nor3}. You'll
find the necessary buttons in the \button{database} menu.
\end{itemize}

\subsection{Extracting the layout and simulating it}
We can verify the correctness of the layout which we just created by
extracting the transistors from the layout using \tool{space} 
\index{space@\tool{space}}
\index{extraction!layout|see{\tool{space}}}
\index{circuit!to layout extraction|see{\tool{space}}}
and then simulating it again:
\begin{itemize}
\item
Start the circuit extractor: \type{space -c nor3}
The result is a circuit cell of which the character of the name has been
capitalized: ``Nor3''. The capitalization is necessary to distinguish
between the original circuit and the extracted circuit.
Just for fun, have a look at it by extracting the sls-description from the
database:
\type{xsls Nor3}
\index{xsls@\tool{xsls}}
\index{extraction!sls-descriptions|see{\tool{xsls}}}
Do you recognize the circuit? Why are there so many transistors? Notice that
the circuit also contains the (parasitic) capacitances.
\item
Let's get rid of the unused transistors:\type{ghoti Nor3}
\index{ghoti@\tool{ghoti}}
\index{circuit!purification|see{\tool{ghoti}}}
\tool{Ghoti} strips all unused transistors from a circuit
description. It is essential to run it before running \tool{spice}. If you skip
ghoti, \tool{spice} aborts with complains about ``singular matrices''. After
running \tool{ghoti} have a look at the circuit again:\type{xsls Nor3} Only 6
transistors and the capacitors remain. With some fantasy you should now be able
to recognize your nor-3.
\item
Start a new \tool{simeye} and simulate the circuit. This works
exactly the same way as was described in \ref{simcircuit}. The only difference
is the name of the circuit ('Nor3' instead of 'nor3'). Compare
the results with the original circuit.

You may want to go to the \button{input} menu of \tool{simeye}. 
There you could
click the \button{speed} 
button, type ``0.5'' return, click \button{ready},
type ``y'' return, and then simulate again. How fast is the nor gate?
\end{itemize}

% Yet another -*- latex -*- document

\section{The second example: a hotel switch}
\label{s-hotel}
This section describes the implementation of a simple circuit {\sl hotel
switch\/} on the fishbone image. The circuit size is about 24 gates, and its
implementation uses a three level hierarchy. You can play with the hotel switch
yourself by typing the following commands.\cd{}
\typeb{tutorial hotel}{cd hotel} \cd{/hotel}
This puts you in a project directory that contains all the data related to the
hotel switch example. (Do not execute the {\tt tutorial} command if you already
did it for the example of section~\ref{s-nor3}.)

\subsection{Functionality}
The function of the hotel switch is to control a light bulb in a hotel room by
means of three switches s1, s2, s3 distributed throughout the room. Whenever
one of the switches is pushed the bulb switches on if it is off and vice versa.
There is also a ``freeze'' switch to prevent the bulb from changing state, see
Figure~\ref{f-hotel-sch}.

\begin{figure}[hbt]
\centerline{\psfig{figure=\figdir/hotel-sch.ps}}
\caption{Functional diagram of the hotel switch}
\label{f-hotel-sch}
\end{figure}

\subsection{A Moore machine}
The functionality of this hotel switch is implemented by a Moore-type state
\index{Moore machine}
machine. First we observe that we can combine the three buttons s1, s2, s3 by
making a logic {\sc or} function s = s1 + s2 + s3. The state diagram of the
hotel switch then looks as depicted in Figure~\ref{f-hotel-state}.

\begin{figure}[hbt]
\centerline{\psfig{figure=\figdir/hotel-state.ps,width=1.0\textwidth}}
\caption{State diagram of the Moore machine}
\label{f-hotel-state}
\end{figure}

\subsection{A three level hierarchy}
\label{s-hierarchy}
With the aid of circuit synthesis tools like Berkeley {\sc sis}, {\sc mis} or
Philips {\sc phaco} the state diagram of Figure~\ref{f-hotel-state} is
translated into a combinational circuit that we call {\sl hotelLogic}. Together
with a 2-flipflop state register and a {\sc nor} gate this implements the hotel
switch as depicted in Figure~\ref{f-hotel-sch2}.

\begin{figure}[hbt]
\centerline{\psfig{figure=\figdir/hotel-sch2.ps,width=0.8\textwidth}}
\caption{Schematic of the highest hierarchical level}
\label{f-hotel-sch2}
\end{figure}

Section~\ref{s-oplib} lists the gates that are available from the fishbone cell
library {\sl oplib}. Figure~\ref{f-hotel-hier} shows the three level hierarchy
of the hotel switch as implemented with the cells from this library.  Since the
library {\sl oplib} does not provide a 3-input {\sc or} gate, it is implemented
with two gates {\sl no310} and {\sl iv110}.

\begin{figure}[hbt]
\centerline{\psfig{figure=\figdir/hotel-hier.ps,width=0.8\textwidth}}
\caption{The three level hierarchy of the hotel switch}
\label{f-hotel-hier}
\end{figure}

The circuit appears in the project directory as 2 ascii files named
\file{hotel.sls} and \file{hotelLogic.sls}. You can use a text editor to view
their contents. Now compile these circuits into the database by typing
\typeb{csls hotelLogic.sls}{csls hotel.sls}
You may want to check that it really worked by extracting a copy of the
circuits from the database:
\typeb{xsls hotel}{xsls hotelLogic}
\index{xsls@\tool{xsls}}
It should look like in figure~\ref{hotelLogic}.
\begin{figure}
{\small
\begin{verbatim}
network hotelLogic(terminal s, freeze, q0, q1, q0new, q1new, vss, vdd)
{
    {inst1}  na210(s, q0, q0new_1, vss, vdd);
    {inst2}  na210(freeze, q0, q0new_2, vss, vdd);
    {inst3}  na310(s_n, freeze_n, q1, q0new_3, vss, vdd);
    {inst4}  na310(q0new_1, q0new_2, q0new_3, q0new, vss, vdd);
    {inst5}  no210(freeze_n, q1_n, h_1, vss, vdd);
    {inst6}  no210(q0, q1_n, h_2, vss, vdd);
    {inst7}  no210(h_1, h_2, q1new_1, vss, vdd);
    {inst8}  na310(s_n, q0, q1, q1new_2, vss, vdd);
    {inst9}  na310(s, freeze_n, q0_n, q1new_3, vss, vdd);
    {inst10} na310(q1new_1, q1new_2, q1new_3, q1new, vss, vdd);
    {inst11} iv110(s, s_n, vss, vdd);
    {inst12} iv110(freeze, freeze_n, vss, vdd);
    {inst13} iv110(q0, q0_n, vss, vdd);
    {inst14} iv110(q1, q1_n, vss, vdd);
}

network hotel(terminal s1, s2, s3, freeze, light, ck, reset, vss, vdd)
{
    {inst1} no310(s1, s2, s3, s_n, vss, vdd);
    {inst2} iv110(s_n, s, vss, vdd);
    {inst3} hotelLogic(s, freeze, q0, q1, q0new, q1new, vss, vdd);
    {inst4} dfr11(q0new, q0, reset, , ck, vss, vdd);
    {inst5} dfr11(q1new, q1, reset, , ck, vss, vdd);
    net {q1, light};
}
\end{verbatim}
}
\caption{ 
The sls circuit descriptions of 'hotel' and 'hotelLogic'.}
\label{hotelLogic}
\end{figure}
\subsection{Layout implementation with the {\sc ocean} tools}
The hierarchy of the layout is similar to the circuit hierarchy of
Figure~\ref{f-hotel-hier}. First we create the layout of {\sl hotelLogic} by
automatically placing and routing it. Then
we use this piece of layout as a sub cell (instance) of the top level cell {\sl
hotel}. To make each cell, we traverse along some tools. This design flow in
the \smc{ocean} environment is depicted in figure~\ref{oceantools2}
(page~\pageref{oceantools2}).

\begin{itemize}
\item
Start \tool{seadali} as described in section~\ref{s-start-seadali}.  Click
\button{inst\_menu}, then \button{Madonna}. 
\index{seadali!madonna}
\index{madonna!example}
Enter the circuit name {\sl
hotelLogic}, then click \button{DO IT}. After a few seconds a placement
appears. If you don't like the result, you may want to read
section~\ref{s-run-madonna} and see what buttons you can click to influence the
placement.
\end{itemize}
\tool{Madonna} has created a placement containing all the instances (= child
cells) 
\index{instances}
in the netlist description. She tried to place them in such a way that
to total wire length and the total area are a small as possible.

\begin{itemize}
\item
Now route the placement. First go to the \button{boxes} menu, then click
\button{trout} 
\index{seadali!trout}
\index{trout!example}
and \button{DO IT}. 
\end{itemize}
You will notice that \tool{trout} created the necessary wires to connect the
terminals according to the netlist specification. The pattern which
\tool{trout} created is indicated in bright colors, the existing layout of the
child-cells (instances) is drawn in a hashed pattern.
\begin{itemize}
\item
Go to \tool{seadali}'s database menu and write the resulting
layout to the database using the name {\sl hotelLogic}.
\item
Just for fun, let's play around a bit with the router. 
Go to the \button{instances} menu and try to move one (or more)
instances (use \button{move}). Go back to the boxes menu and click
\button{trout} again. Click \button{erase wires} before \button{DO IT},
to cause \tool{trout} to remove the existing wires in the layout.  If you would
like to try other features of the router, you can proceed to
section~\ref{s-advanced}.
\end{itemize}

At this point, you may want to verify that the layout cell {\sl hotelLogic}
really implements the corresponding circuit. This is not strictly
necessary, because the {\sc ocean} tools generate layout that is ``correct
by construction''. See subsection~\ref{s-verification} for details on how to
verify a layout cell.

For the moment, we'll just go on and use the layout cell {\sl hotelLogic} to
implement the top level circuit {\sl hotel}. Making the layout of {\sl hotel}
goes in the same way as we just made its child cell {\sl hotelLogic}:
\begin{itemize}
\item
Go to the \button{instances} menu and click \button{madonna}. Enter
as cell name {\sl hotel} and click \button{DO IT} to put \tool{madonna}
to work for you. Verify that \tool{madonna} indeed placed the 5 instances of
{\sl hotel}, including {\sl hotelLogic}. 
\item
Now route the this placement of {\sl hotel}. First go to the \button{boxes}
menu, then click
\button{trout}, followed by \button{DO IT}. 
\item
Verify the result of the routing by clicking 
\index{check nets}
\index{seadali!check nets button}
\button{check nets}. 
This button compares the current layout with the circuit description. It
reports any unconnects or short-circuits. 
\item
If no errors are reported, write the layout under the
name {\sl hotel} to the database.
\end{itemize}

\subsection{Verification}
\label{s-verification}
If everything went well then now the time has come to verify that the layout
has the desired functionality.  Although we say ``verification'' we actually
mean ``simulation''. In order to simulate the behavior of the layout we need 2
items.
\begin{enumerate}
\item {\sl The extracted circuit.\/} This is the circuit that the tool
\tool{space} creates by scanning the actual layout. Depending on the options
passed to \tool{space}, the extracted circuit may or may not contain parasitic
resistors and capacitances.
\item {\sl A stimuli file.\/} This file defines how the input signals for the
extracted circuit change as a function of time. It also defines which output
signals of the extracted circuit must be observed.
\end{enumerate}
The tool \tool{simeye} takes these 2 items, calls the simulator and displays
the simulation results. Figure~\ref{f-hotel-extract} shows this verification
flow.
\begin{figure}[hbt]
\centerline{\psfig{figure=\figdir/hotel-extract.ps,width=1.0\textwidth}}
\caption{The verification flow}
\label{f-hotel-extract}
\end{figure}

Note that after extraction we have {\bf two} circuit representations of {\sl
hotel} in the database! The first {\sl hotel} was created by \tool{csls} as
described in Subsection~\ref{s-hierarchy} and the second {\sl Hotel} (with the
first letter a capital) is created by
\tool{space}. We can tell them apart by looking at the first character of the
circuit name: extracted cells start with a capital.

\subsection{Extracting the circuit}
\label{t-ex}
Now type the following command to extract the circuit {\sl Hotel} from the
layout {\sl hotel}:
\type{space -c hotel}
The option {\tt -c} specifies that \tool{space} must also extract capacitors.
You can have a look at the extracted circuit by typing
\type{xsls Hotel | more}
Watch that capital {\tt H}. Now purify this circuit description by removing all
unconnected transistors:
\type{ghoti Hotel}
Finally, start the simulator by typing:
\type{simeye Hotel \&}
Click the \button{run} button to start the simulation. Within a few seconds the
simulation results are displayed in the \tool{simeye} window.

\subsection{Simulating at various levels of abstraction}
\index{simulation!on three levels|bold}
\tool{Simeye} allows you to simulate a circuit at several levels of
abstraction. The most abstract level is obtained by clicking \button{sls-logic}
before clicking the \button{run} button. This calls a logic simulator that 
models a transistor as an ideal switch. A slightly less abstract
level of simulation is obtained by clicking \button{sls-waves}. This models a
transistor as a switch with a linear resistance and linear capacitances and it
recognizes the parasitic capacitances of the wires. Finally, the most detailed
simulation level is obtained by clicking \button{spice}. This calls the
Berkeley circuit simulator \tool{spice}.

The small buttons labeled \button{A} and \button{D} are toggles that either
cause an ``analogue'' or a ``digital'' interpretation of the signals to be
displayed. Spice only allows an analogue interpretation of course.

\subsection{Graphically editing the stimuli file}
You may want to experiment with different stimuli for {\sl Hotel}. You can
easily edit the stimuli with the graphical editor that is part of
\tool{simeye}. To do this, click the \button{input} button in \tool{simeye}.
This brings you in a graphical editor. The five most important buttons of this
editor are
\begin{itemize}
\item [\button{new}] define a new input signal for the circuit.
\item [\button{edit}] edit an existing input signal.
\item [\button{speed}] change the frequency of the input signals
\item [\button{t\_end}] change the moment that the simulation stops
\item [\button{ready}] exit the editor, back to the simulator
\end{itemize}

\section{The third example: using logic synthesis to make 'hotel'}
\label{kissistutorial}
To demonstrate the use of the \smc{sis}\footnote{You can run this tutorial
with or without having the \smc{sis} logic synthesis package on your
machine.  In the latter case you can just use the pre-synthesized file
'hotel.sls'. We cannot distribute the \smc{sis} package with the \smc{ocean}
tools, unfortunately. You can, however, retrieve \smc{sis} yourself using
anonymous ftp (host ic.berkeley.edu, directory pub/sis). \smc{sis} runs on
many hardware platforms.} logic synthesis program we included another
tutorial directory:
\cd{}\type{tutorial sis} \cd{/sis}

Essentially it contains the same circuit as the previous 'hotel' tutorial. In
this case, however, we will also generate the circuit description
automatically. The main difference is that \smc{sis} generates the circuit {\sl
hotel} in one step without hierarchy.

The program \tool{kissis} 
\index{kissis@\tool{kissis}}
is performing this entire FSM synthesis trajectory
for you. The format of the input table for \smc{sis} is simple. It basically
declares the number inputs and outputs, and state transition conditions. For
example, the kiss2 file in figure~\ref{tut-kiss2} describes a simple hotel 
switch with 4 states.
\begin{figure}
{\small
\begin{verbatim}
.i 2     /* 2 inputs: switch (1st column) and freeze (2nd) */
.o 1     /* 1 output: light (last column) */
.r off1  /* reset to state off1 */
-1 off1 off1 0
10 off1 off1 0
00 off1 off0 0
-1 off0 off0 0
00 off0 off0 0
10 off0 on1  0
-1 on1  on1  1
10 on1  on1  1
00 on1  on0  1
-1 on0  on0  1
00 on0  on0  1
10 on0  off1 1
\end{verbatim}
}
\caption{
A file in \protect\fname{kiss2}-format, describing the hotel circuit.}
\label{tut-kiss2}
\end{figure}
Each line in this table corresponds to a state transition (``an arrow'') in the
state diagram of Figure~\ref{f-hotel-state}, page~\pageref{f-hotel-state}.
Each line in the table contains four fields: the input signals, the current
state, the next state and the output signal.

A file which contains this state transition table is present in the 'sis'
tutorial under the name 'hotel.kiss2'. With this tutorial you can run the logic
synthesis tools yourself\footnote{If you do not have the \smc{sis} package
installed, just type ``csls hotel.sls'' instead to use a pre-synthesized
version of hotel.}:
\type{kissis hotel.kiss2} The output of
\tool{kissis} is a circuit description, which was mapped on the 'fishbone'
Sea-of-Gates library.  The file
\fname{hotel.sls} which was created by \tool{kissis} looks as in
figure~\ref{hotelsis} if you look at it using \tool{xsls hotel}.
\begin{figure}
{\small
\begin{verbatim}
/*  This file was automatically created by blif2sls.
    Thu Jun 24 09:57:52 1993 */
#include<oplib.ext>

/*     Code Assignment: off1  00            */
/*     Code Assignment: off0  10            */
/*     Code Assignment: on1   11            */
/*     Code Assignment: on0   01            */

network hotel( terminal IN_0, IN_1, OUT_0, R, CK, vss, vdd)
{
 {inst0}    dfr11( n_2355_, R, CK, LatchOut_v2, vss, vdd);
 {inst1}    dfr11( OUT_0, R, CK, LatchOut_v3, vss, vdd);
 {inst2}    iv110( LatchOut_v3, n_2314_, vss, vdd);
 {inst3}    iv110( IN_0, n_2315_, vss, vdd);
 {inst4}    no210( IN_0, IN_1, n_18_, vss, vdd);
 {inst5}    mu111( LatchOut_v2, n_2314_, n_18_, n_2355_, vss, vdd);
 {inst6}    no210( IN_1, n_2315_, n_17_, vss, vdd);
 {inst7}    mu111( LatchOut_v3, LatchOut_v2, n_17_, OUT_0, vss, vdd);
}
\end{verbatim}
}
\caption{The output of the logic synthesis program
\protect\smc{sis} for the hotel circuit.} 
\label{hotelsis}
\end{figure}
\smc{Sis} just numbers the inputs and outputs:
\begin{itemize}
\item[IN\_0]
is the switch button input {\sl s}.
\item[IN\_1]
is the {\sl freeze} signal which inhibits the input of {\sl s}.
\item[OUT\_0]
the output of the switch: {\sl light}.
\item[R]
is the reset input for the flip-flops.
\item[CK]
is the clock input.
\end{itemize}

To avoid problems with undefined flipflop states in \tool{sls}, the program
\tool{kissis} instructs \smc{sis} to use the resettable flip-flop 'dfr11'. The
reset-signals of the flip-flops are automatically connected to the terminal 'R'
of hotel. In this way you can force the FSM to a known state at the beginning
of the simulation.

Now simulate the hotel switch using \tool{simeye}. An appropriate command
file is present in the tutorial directory. Click \button{run} to start the
simulation. Notice that there is also a line called 'State' in the output,
which indicates the symbolic state of the Finite State Machine. 
Click on \button{value} and move to the 'State'-bar to see the state as a
function of the time.

When you finished simulating, generate the layout using \tool{seadali}
with \tool{madonna} and \tool{trout}: 
\begin{itemize}
\item
Go to the \button{instances} menu and click \button{madonna}. Enter
as cell name {\sl hotel} and click \button{DO IT} to put \tool{madonna}
to work for you. Verify that \tool{madonna} indeed placed the 5 instances of
{\sl hotel}, including {\sl hotelLogic}. 
\item
Now route the this placement of {\sl hotel}. First go to the \button{boxes}
menu, then click
\button{trout}, followed by \button{DO IT}. 
\item
Verify the result of the routing by clicking 
\index{check nets}
\index{seadali!check nets button}
\button{check nets}\footnote{It is not absolutely necessary to do this.
\tool{trout} automatically performs this connectivity check after routing.}. 
This button compares the current layout with the circuit description. It
reports any unconnects or short-circuits. 
\item
If no errors are reported, write the layout under the
name {\sl hotel} to the database.
\item
A quick way of doing all of the above in one step is by going to the
\button{database} menu and clicking \button{Place and Route}.
\index{automatic!placement and routing in one step}
\end{itemize}

Now you can extract the circuit and simulate it again. This
works in exactly the same way as in the previous section. Please proceed to
section~\ref{t-ex}.

As second example the circuit 'three' is included, which performs essentially
the same as 'hotel', but has 3 button inputs instead of one. 

\section{Some advanced features}
\label{s-advanced}
In this section we demonstrate some advanced features of the \smc{ocean}
system using the hotel circuit which you've just made. 

\subsection{Playing around with \protect\tool{Madonna}}
\index{madonna!example}
Madonna doesn't like to be fooled around with too much. But let's try to see
how far we can go using the circuit {\sl hotelLogic}. For full details on the
options you can have a look in chapter~\ref{madonna} on page~\pageref{madonna}.

First read the layout of the cell {\sl hotelLogic} into \tool{seadali} (the one
which you've made a while ago). Then go to the instances menu and click
\button{Madonna}. Tell \tool{madonna} that you are sure to do that by 
clicking \button{yes} and just press [return] at the cell name to 
select {\sl hotelLogic}.

Now let's ask \tool{madonna} to place the cell in a larger area: use
\button{zoom Out} to zoom out a bit. 
\index{madonna!set placement box}
Then press \button{set box } and
drag a box over a larger area and click \button{DO IT}.
\tool{Madonna} will spread out the modules over the specified area. If you make
the area very large, you will notice that it takes \tool{trout} longer to route
it.

Obviously, we can also do the opposite. If you specify a very small box
\tool{Madonna} will squeeze the cells into the minimum area. If the area of the box
is too small, she will expand into the horizontal direction. If you press
\button{y-expand}, however, the layout will grow taller because she expands in
the vertical direction.

\subsection{Using \protect\tool{trout} in various ways}
\index{trout!example}
\subsubsection{Partial manual and automatic routing}

The router \tool{trout} is a very powerful tool.  Let's try to use
\tool{trout} to complete a partially routed circuit.  Read a routed layout
of {\sl hotelLogic} in and go to the \button{boxes}-menu. Enable the masks
\mask{in}, \mask{ins}, \mask{cps} and \mask{cos} in the bottom of the
screen. Now use \button{DELETE} to erase some part of the existing routing.
Notice that the layout of the son-cells cannot be erased. Now
click\button{trout} and just \button{DO IT} to cause \tool{trout}
to 'repair the damage'. \tool{Trout} automatically discovers which nets are
already connected and which existing wires can be used to make connections
which are as short as possible. In this way you can manually pre-route
certain wires and let \tool{trout} so the rest. Just play around a bit with
this feature.  Do not press \button{erase wires} in this case because that
will remove all existing wires prior to routing.

\subsubsection{Verifying a layout}
\index{design rules!checking connectivity}
\index{seadali!check nets button}
\index{check nets}
If you have created a (partially) manual layout, it is interesting to check
whether it is correct or not. This can be done simply by pressing \button{check
nets} in the \button{boxes}-menu. If no window appears, everything is OK. Just
for fun, make some deliberate errors in the (routed) layout of {\sl
hotelLogic}. Cut one wire, and make a short circuit between two others (use
\button {DELETE} and
\button{APPEND}). If you press \button{check nets} again, a window will warn
you for the problems. The unconnects and shorts are also indicated in the
layout. The indicator arrow shows the terminals of the nets which are involved
in the short-circuit or the unconnect. The shorts are also indicated by the
white pattern in the\mask{bb}-mask. Unfortunately it impossible to show
the exact location of the short.

\subsubsection{Automatic capacitors}
\index{trout!decoupling capacitors}
Now let's go for the really wild stuff!! \tool{trout} has the capability
to convert all unused transistors into capacitors which are switched between
power and ground. In this way, the noise on the power lines can be reduced
effectively. The best way to try this is on a rather empty layout, for
instance the one which you made using \tool{madonna} in a big box. Now press
\button{trout} and \button{Option menu} and select \button{capacitors}. Then
click \button{- return -} and \button{DO IT} and watch what happens!
It should look like figure~\ref{capas}. Notice that the power wires are made as
fat as possible, and that the layout contains many additional vias and wires to
'nail' the power lines properly together.  Notice also that there is now hardly
any way left to route additional wires through this module.

\begin{figure}
\centerline{\psfig{figure=\figdir/capahotel.eps,width=0.85\textwidth}}
\caption{The same circuit as figure \protect\ref{fishhotel}
(page~\protect\pageref{fishhotel}), but routed with the option
\protect\button{capacitors}. This causes~\protect\tool{trout} to connect all
unused transistors and, if possible, to use them as capacitors. In this
way the noise on the power lines is reduced considerably.}
\label{capas}
\end{figure}

\subsection{Inserting your own nor in the circuit}
If you finished the example of section~\ref{s-nor3} successfully, you may want
to edit the files \file{hotel.sls} and \file{hotelLogic.sls}, replacing the
calls to the library cell {\sl no310} by calls to your own {\sl nor3}. You'll
have to use the {\tt extern} declaration of the SLS language to declare the
order of the terminals. For instance, insert the line ``extern network
nor3 (terminal A, B, C, F, vss, vdd)'' on top of the files \file{hotel.sls} and
\file{hotelLogic.sls}. See the \smc{sls} manual for more details.

\subsection{Making the hotel switch on a different image}
\index{image!switch to another image}
In this tutorial we have been using the {\sl fishbone} image. You can run the
same tutorial with the other two images. To make a project in the {\sl octagon}
image, type: \cd{}
\typeb{setenv OCEANPROCESS octagon}{tutorial hotel} \cd{/hotel}
\index{OCEANPROCESS@\fname{OCEANPROCESS} environment variable}
similarly, you can use the {\sl gatearray} image:
type: \cd{}
\typeb{setenv OCEANPROCESS gatearray}{tutorial hotel} \cd{/hotel}
The environment variable \fname{OCEANPROCESS} must be set to the image you're
working in. Also, make sure that no directory hotel already exists in the
directory. In the new hotel directory you are able to run the same hotel
tutorial. Only the
\tool{spice} simulations will most likely not work properly.
Figures~\ref{fishhotel}, \ref{octagonhotel} and \ref{gatehotel}
(page~\pageref{octagonhotel}) show the layout of {\sl hotelLogic} in the {\sl
fishbone}, {\sl octagon} and {\sl gatearray} image.

For now, let's just use the octagon image and try to make the layout of cell
{\sl hotelLogic}:
\typeb{csls hotelLogic.sls}{seadali \&}
Just click \button{instances} followed by \button{madonna} to place the
circuit. Have a good look at the image, and try to move some instances.
Notice that \tool{seadali} is automatically mirroring the instances in the
proper way.  Notice also that \tool{madonna} is clever enough to overlap
some gates in the same 'quarter'.

Next, route it using \button{trout}. This is a tree-layer process, which makes
it sometimes hard to interpret the layout. Use \button{visible} to switch off
some masks. \cd{/myproject}

\subsection{Plotting your layout}
\index{plotting layout|see{getepslay}}
\index{getepslay@\tool{getepslay}}
It is easy to make a plot of your layout. The program \tool{getepslay} converts
the layout into a postscript file, which you can directly send to the
laser printer. For example: 
\type{getepslay hotelLogic}
will create the postscript file \fname{hotelLogic.eps}. See the manual page of
\tool{getepslay} for more details.
\index{tutorials|)}

\subsection{Removing cells and projects}
\index{project!removing}
\index{removing!projects}
To remove an entire project you can just use the normal \smc{unix} command.
For example: \cd{}\type{rm -rf hotel} removes the entire project in
subdirectory
\fname{hotel}. A less rigorous way is to remove individual cells using
\tool{rmdb}: \cd{/hotel}\type{rmdb -c hotel layout}
\index{rmdb@\tool{rmdb}}
\index{removing!cells|see{\tool{rmdb}}}
removes the layout of
cell \fname{hotel} from the database. You must be in a project directory to
call this command. 


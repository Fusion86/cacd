% -*- latex -*-

\chapter{Layout to circuit extraction: 'space' and 'ghoti'}
\label{extraction}
\smc{Ocean} provides powerful tools for verifying a layouts. They are indicated
in the design flow diagram of figure~\ref{oceantools} on
page~\pageref{oceantools}. The most commonly used 
used way is layout-to-circuit extraction using \tool{space}, where a netlist
description (including parasitics) is extracted for a layout.
\index{layout to circuit extraction|see{\tool{space}}}

\section{The extractor \tool{space}}
\index{space@\tool{space}|bold}
\tool{Space} is an advanced layout-to-circuit extractor for analog as well as
digital circuits. From mask-level layout, \tool{space} produces a circuit
netlist consisting of transistors. Optionally, this netlist contains also the
wiring capacitances to substrate (\fname{-c} option) and inter-wire coupling
capacitances (\fname{-C} option). \tool{Space} can also extract wire
resistances, but on the version which is supplied with the \smc{ocean} system
this feature is disabled\footnote{Contact the author Nick van der Meijs if you
are interested in this feature (email him at: n.p.vandermeijs@tudelft.nl)}.
These parasitics are extracted very accurately, which enables you to tune your
circuit for maximum performance with a high reliability. \tool{Space} can
handle very large circuits.

For a more detailed description of \tool{space} we refer to the {\em
SPACE User's manual} by Nick van der Meijs and Arjan van Genderen, which is
included in the \smc{ocean} documentation distribution. 
Examples of how to use \tool{space} can be found in chapter~\ref{tutorial}
(especially section~\ref{t-ex}). 

\section{\tool{space}: Capitalization of the cell name}
\index{capitalization of cell name}
Calling \tool{space} on a cell \fname{mycell} results in a new circuit cell
called \fname{Mycell} (watch the capital 'M'!). The capitalization of the
first character is done to prevent that \tool{space} overwrites the original
circuit description \fname{mycell}. 
This convention is also recoqnized by \tool{simeye}. If you run \tool{simeye}
on the extracted circuit \fname{Mycell}, \tool{simeye} will use the sls-command
\index{sls!command file}
file \fname{mycell.cmd} if the file \fname{Mycell.cmd} does not exist. In this
way you can easily simulate the original and the extracted circuit with the
same command file.

\section{Purifying netlists: \tool{ghoti}}
\index{ghoti@\tool{ghoti}|bold}
\index{removing|unused transistors|see{\tool{ghoti}}}
\tool{Ghoti} is a purifier for extracted circuit descriptions, much in the same
way like \tool{fish} is for layouts. The extracted network which results from
\tool{space} may contain some irrelevant pieces. Especially on a sea-of-gates
there will be many isolation transistors 
\index{isolation transistors}
or totally unused transistors, which
have no (or hardly any) influence on the function of the circuit. These
useless elements, however, may cause the simulator some difficulties. The
\tool{spice} simulator crashes the unconnected transistors (complains about ``singular matrices''). \tool{Spice} will
also consume considerably more \smc{cpu}-time if useless transistors are
present.

\tool{Ghoti} was made to solve these problems by purifying the
netlist. It reads a circuit description, processes it, and write a purified
circuit with the following features:
\begin{itemize}
\item
all unconnected transistors are removed,
\item
all isolation transistors are removed,
\item
power nets which are not yet connected are joined.
\item
(with \fname{-r} option) reduce series and parallel connections of transistors.
This option, where \tool{ghoti} collapses any parallel or series connection or
transistors into one equivalent transistor, is especially useful for analog
circuits which exceed the capacity of \tool{space}.
\end{itemize}
Therefore: {\bf Always \tool{ghoti} before running \tool{spice}!}.
You can call \tool{ghoti} as often as you like on a cell. 

If the circuit is large (e.g. thousands of transistors, or the entire chip)
using \tool{ghoti} also reduces the run time of the switch level simulator
\tool{sls} because the number of elements to keep track of decreases
drastically. See chapter~\ref{tutorial} for examples on how to run
\tool{ghoti}. 




% -*- latex -*-

\chapter{The Sea-of-Gates image}
\label{s-fishstruct}
\section{Why Sea-of-Gates?}
\begin{figure}
\centerline{\psfig{figure=\figdir/fishhotel.eps,width=0.95\textwidth}}
\caption{A circuit on the {\em fishbone}
sea-of-gates image. This image was developed
at Delft University of Technology. Our chip
with this image contains 191,312 transistors
on an area of approx. $10 \times 10 ~ mm^2$. 144
pins are available.}
\label{fishhotel}
\end{figure}

Before we explain how to use the {\sc ocean} layout tools, it is important to
understand the kind of layout that they produce. {\sc Ocean} tools produce {\sl
sea-of-gates} layout. This means that they deal with a prefabricated {\sl
image} \index{image} consisting of {\sc mos} transistors. To implement a
circuit, the tools (or the designer, for that matter) interconnect these
transistors with metal wires. The sea-of-gates layout strategy aims at four
goals:
\begin{enumerate}
\item {\sl Minimization of the fabrication time.}
Because the chips are prefabricated (the transistors are already on the master
image), the silicon foundry only processes the masks related to metal wires. As
compared to full custom layout, the number of masks processed by the silicon
foundry is often reduced by more than 60\%.
\item {\sl Minimization of the design time.}
The time involved in designing a cell layout is reduced dramatically (as
compared to full-custom) because the transistors are preplaced on the image.
Typically, it takes only a few minutes to layout a flipflop or a combinatorial
gate, and the designer does not need to know much about the process design
rules.
\item {\sl Minimization of the chip cost.}
The layout design starts with a prefabricated master image. This is a
semi-manufactured article that can be produced in large quantities.
Consequently, it can be cheap.
\item {\sl Full-custom properties on Semi-Custom chips: Efficient implementation 
of structured logic (RAM, PLA etc.)}. In contrast to the conventional
Gate-Array, a sea-of-gates image does not have pre-defined routing channels.
This enables a much more compact and clean implementation of structured
circuits such as processors.
\end{enumerate}

\begin{figure}
\centerline{\psfig{figure=\figdir/octagonhotel.eps,width=0.95\textwidth}}
\caption{The {\em octagon} image which was developed at Twente University in
the Netherlands. In the lower left corner the same circuit as
in figure~\protect\ref{fishhotel} is implemented. This highly symmetrical image
allows the cells to be rotated.}
\label{octagonhotel}
\end{figure}

The {\sc ocean} suite of tools handles a wide variety of sea-of-gates master
images and process technologies. Three of them are included in the
distribution:
\begin{itemize}
\item[{\em fishbone}]
\index{image!fishbone}
\index{fishbone image}
A gate-isolation image in a $1.6 \mu$ \smc{cmos} process with two layers of metal
(figures~\ref{fishhotel}, \ref{f-nand2} and \ref{f-fishbone}).
\item[{\em octagon}]
\index{image!octagon|bold}
\index{octagon image}
A remarkable octagonal image in a $0.8 \mu$ \smc{cmos} process with three layers
of metal interconnect (figure~\ref{octagonhotel} and \ref{othernand2}).
\item[{\em gatearray}]
\index{image!gate array|bold}
\index{gate array image|bold}
An old fashioned gate-array in a $4 \mu$ single metal layer \smc{cmos} process
(figures~\ref{gatehotel} and \ref{othernand2}).
\end{itemize}

Although the three images differ from each other in many
respects, the {\sc ocean} tools can take advantage of their specific
properties.  For instance, the placer \tool{madonna} uses the
properties of the octagon image to mirror cells with respect to the $45^\circ$
mirror axis. And the router \tool{trout} takes advantage of the vertical poly
strips in the gate array image to find the best route for a wire.
Especially in the single-layer {\em gatearray} image the polysilicon strips are
an essential feature for the routing.

In figures~\ref{f-nand2} and \ref{othernand2} the implementation of a 2-input
nand is shown for each of the three images. Notice that the layout style is
very different, even for such a simple circuit.

Currently, our local foundry {\sc dimes} runs a production line for
metallization of the {\sl fishbone} master image, designed at Delft University
of Technology. It therefore makes sense to concentrate on this image and go
into some of its details. The other images are supplied mainly to demonstrate
the features of the placer and the router in the {\sc ocean} system.

\begin{figure}
\centerline{\psfig{figure=\figdir/gatehotel.eps,width=0.95\textwidth}}
\caption{The single-layer {\em gatearray} image is a traditional row-based
gate-array. The figure shows the same circuit as in the previous figures.}
\label{gatehotel}
\end{figure}

\begin{figure}
\makebox[0.98\textwidth]{
  \hfill
  {\psfig{figure=\figdir/nand2-sch.ps,height=0.4\textheight}}
  \hfill \hfill
  {\psfig{figure=\figdir/nand2-flat.eps,height=0.4\textheight}}
  \hfill
}
\caption{A two-input {\sc nand} gate and its layout on the fishbone sea-of-gates
         image. This is the library cell 'na210'.} 
\label{f-nand2}
\end{figure}

\begin{figure}
\makebox[0.98\textwidth]{
  \hfill
  {\psfig{figure=\figdir/octanand2.eps,width=0.44\textwidth}}
  \hfill \hfill
  {\psfig{figure=\figdir/gatenand2.eps,width=0.44\textwidth}}
  \hfill
}
\caption{
The same two-input nand gate on the octagon (left) and the gatearray (right)
image.}
\label{othernand2}
\end{figure}

\section{The TU-Delft fishbone chip}
\index{image!fishbone|(bold}
\index{fishbone image|(bold}
At Delft university of technology we developed a sea-of-gates chip.
Figure~\ref{f-minimove} (on page~\pageref{f-minimove}) shows one such chip. Its
core contains 191,312 transistors in the {\sl fishbone} image. Each fishbone
row is a series connection of 1087 transistors. There are 88
n-transistor rows and 88 rows with p-transistors. The chip has 144 pins, of
which 128 are user programmable as input, output, bi-directional or direct
pins. The remaining 16 pins are for power connections.  Generally we combine a
number of separate designs on one chip. For this purpose we created a
multi-project chip concept, in which the chip is partitioned into 4, 3 or 2
separate 'pseudo-chips', depending on the sizes of the circuits. The pseudo
chip are isolated from each other and have separate power supply connections.

125 Wafers containing the raw transistor pattern have been pre-produced by a
commercial foundry. At our own chip production facility ({\sc dimes}) the chips
are metalized with a very short turn-around time. Each 4-inch wafer contains 46
usable chips.  In the past year (1992-1993) 12 wafers with different designs
have been produced. At this moment we (that is, the guys from the \smc{ocean}
team) are trying to make this cheap production facility available to other than
university users. Please contact us if you are interested. 


\begin{table}
\begin{center}
\sf
\small
\begin{tabular}{|ll|} \hline  
process: & C3DM (Philips) $1.6\mu$ double layer \smc{cmos}, $\lambda = 0.2 \mu
m$ \\
dimensions: & bruto: $10 \times 10 mm$, {\sl fishbone} core area: $8.7 \times 8.7 mm$ \\
transistors: & 191,312 transistors in 176 rows of 1087, {\sl fishbone} image.\\
pins: & 144 bonding pads, 128 user programmable, 16 power\\
transistor sizes: &
$1.6 \mu m \times 23.2 \mu m$ (nmos) and $1.6 \mu m \times 29.6 \mu m$ (pmos)\\
transistor pitch: & $8.0 \mu m$ (horizontal)\\
metal wire width: & $2.4 \mu m$ (both metal 1 and metal 2)\\
contact size: & $2.0 \times 2.0 \mu m$ (hole), $4.0 \times 4.0 \mu m$ (incl.
metal overlap)\\
poly gate resistance: & $700 Ohm$ (nmos) $950 Ohm$ (pmos)\\
contact resistance: & $22 Ohm$ (\mask{ps}-\mask{in}), $0.13 Ohm$
(\mask{in}-\mask{ins})\\
metal resistance: & $0.056 Ohm/$ (\mask{in}), $0.026 Ohm/$
(\mask{ins})\\
threshold: & $V_t = 0.7 V$ (nmos), $V_t = -1.2 V$ (pmos)\\ \hline
\end{tabular}
\rm
\end{center}
\caption{Summary of the parameters of the TU Delft sea-of-gates chip with the
{\sl fishbone} image.}
\end{table}


\section{The general structure of the fishbone image}
Figure~\ref{f-fishbone} on page~\pageref{f-fishbone} shows the structure of the
fishbone image.  This image is fabricated in the \smc{c3tu} process, the Delft
derivative of the Philips
\smc{c3dm} process. The length of a transistor gate is $1.6 \mu$. The designer
can program two layers of metal and a number of contact layers. Each layer is
referred to as a 2- or 3 character acronym:
\begin{enumerate}
\item
\mask{in} (metal1, this is the metal layer that is closest to the image)
\item
\mask{ins} (metal2, this is the second level metal layer)
\item
\mask{con}, \mask{cop} and \mask{cps} (contacts from \mask{in} to n-type
diffusion, p-type diffusion or poly)\footnote{ For you as a designer, these
three masks are equivalent. The \smc{ocean} tools (\tool{fish}) take care
of selecting the proper one of these three.}
\item
\mask{cos} (contact between \mask{in} and \mask{ins})
\end{enumerate}

The fishbone image consists of alternating rows of n-type and p-type
transistors. Each row is a seemingly endless series connection of transistors.

Wire segments and contacts may only be placed at certain ``grid points''.
\tool{Seadali} optionally marks these grid points with little white dots. Once
you have grown accustomed to the image, you will know these positions by heart.
In figure~\ref{f-fishbone} the gridpoints are indicated with $\times$ marks.
In the vertical direction they have been numbered 0 \ldots 28.

\begin{figure}
\centerline{\psfig{figure=\figdir/fishbone.ps,height=0.8\textheight}}
\caption{The basic structure of the fishbone sea-of-gates image.}
\label{f-fishbone}
\end{figure}

\section{Restrictions on placing contacts}
\label{contacts}
\index{design rules!for contacts|bold}
Unfortunately it is not allowed to place a contact at every grid
position. You have to take the following design rules for contacts
into account:
\begin{enumerate}
\item
\index{stacked contacts}
Contacts may not be stacked. It is therefore not allowed to place a \mask{con},
\mask{cop} or \mask{cps} contact on top of a \mask{cos} contact
(metal1-metal2).  As a result, all connections between diffusion (or poly) to
metal2 have to ``dogleg'' in metal1.
\item
On top of (poly) gate contacts (grid rows 1, 6, 7, 13 in 
fig.~\ref{f-fishbone}) \mask{cos}
contacts (metal1-metal2) are not allowed. 
All \mask{cos} contacts must be placed
on the diffusion tracks (2-5 and 8-12) or on the power supply (0 and 14).
\item
Substrate contacts can be placed under the power
rails (rows 0 and 14). As a rule of thumb, place one substrate contact for
every 3 transistors. The placement of the subtrate contact can be performed
automatically. 
\end{enumerate}
Needless to say that the {\sc ocean} tools take these rules into account.

\section{Transistor layout on fishbone: gate isolation technique}
The gate of each transistor consist of a polysilicon pattern. Each transistor
gate can be reached at two grid points using a \mask{cps} contact. In
figure~\ref{f-fishbone} the n-transistor can be connected in grid rows 1 and 6,
the p-transistor in rows 7 and 13. Notice that the p-transistor is taller than
the n-transistor.

The source and drain of the transistors (diffusion) can be connected to grid
rows 2 through 5 (n-transistor) and 8 through 12 (p-transistor)
using a \mask{con} or \mask{cop} contact\footnote{If you are using
\tool{seadali}, just select an arbitrary mask from \mask{cps}, \mask{con} or
\mask{cop}. The \smc{ocean}-tools will take care of selecting the proper mask}, respectively.

Adjacent transistors (of the same type) share their source/drain connections.
Therefore, in each row, the transistors are arranged in a seemingly endless
series connection.  This does not imply, however, that only series connections
can be programmed on this image.

To isolate circuits from each other, so called ``isolation-transistors'' 
\index{isolation transistors|bold}
are
used.  An n-type isolation transistor is a transistor of which the gate is
connected to \smc{vss}.  Therefore it is always switched off.  In this way the
source and drain of this transistor are ``isolated''.  
\index{gate isolation technique}
Similarly a p-type
isolation transistor can be created by connecting its gate to \smc{vdd}.  A
2-input {\sc nand} gate requires two isolation transistors. This is shown in
figure~\ref{f-nand2} (page~\pageref{f-nand2}), where the two rightmost
transistors serve the purpose of isolation. As a consequence we are able to
place several {\sc nand} gates next to each other, without causing any short
circuits between the gates.

It is clear that a layout cell only needs isolation transistors on either its
left or its right side, but not both, as long as this strategy is applied to
all the cells on the chip. We will always take the right side of a cell to
accommodate the isolation transistors, like it's done in figure~\ref{f-nand2}.
Note, however, that we can only do this because the fishbone image does not
allow to mirror a cell in the y-axis!

\section{Power net strategy}
The fishbone image consists of alternating rows of n-type and p-type
transistors. Each row is a series connection of many, many transistors.
Between the rows there are two power ``rails'' in metal1 which run horizontally
across the entire with of the chip.  Vertically we use a scheme of alternating
\smc{vss} and \smc{vdd} power rails.  Next to a \smc{vss}-power rail there are
two rows of n-type transistors, next to a \smc{vdd}-power rail there are two
rows of p-type transistors.  This results in a nnppnnppnnpp.... order of
transistor rows (figure~\ref{f-fishbone}).

Under the power line well contacts (substrate contacts) are placed to keep the
wells at the appropriate voltage. In our system the well contacts will be added
automatically at the last stage.

\index{image!fishbone|)}
\index{fishbone image|)}
\section{Some practical hints}
You will soon notice that the wiring space is limited. In order to prevent
unconnected points it can be useful to follow these guidelines:
\begin{itemize}
\item
Avoid using long horizontal metal2 wires, because they may block future
vertical metal2 wires. Use metal2 mainly for vertical wires.
\item
Use metal1 mainly for horizontal wires. The layout of the power wires almost
force you to do this.
\item
Design each cell such that its ``permeability''\index{permeability of a cell}
is maximized. That is, leave as much useful free space as possible. Always bear
in mind that the routing capacity of a cell is desperately needed (by you or an
automatic router) at a higher level in the hierarchy.
\item
Often the routing capacity in the horizontal direction is the bottleneck.
Leave an unused transistor row if necessary.
\item
Ensure that all terminals of a cell can be reached from some point outside that
cell. Remember that stacked contacts are not allowed, and that a connection to
metal2 therefore requires at least a contact-less spot on a wire.
\item
Take advantage of the fact that the fishbone image contains series connections
of transistors of the same type (n-mos or p-mos). Try to find chains of
n-mos transistors in the circuit diagram of your cell and lay these chains out
on the fishbone image. Isolate the chains from each other with isolation
transistors. Follow the same procedure for p-transistor chains. If you are very
lucky, the entire circuit can be implemented with only one n-chain and one
p-chain. An example of such a circuit is the D-flipflop depicted in
figure~\ref{f-dff}.
\item
Sometimes, when routing area is in short supply, it may be necessary to use a
poly gate to make a vertical connection. Poly, however, has a high electrical
resistance. In combination with the parasitic capacitance of a long metal wire,
this can significantly slow down signal transmission times on such a wire. As a
rule of thumb, you should only use poly strips on a short distance of signal
inputs and never in the neighborhood of signal outputs.
\item
Do not place substrate contacts. They will be added automatically by the router
in the last stage of the design.
\end{itemize}

\begin{figure}
\makebox[0.98\textwidth]{
  {\psfig{figure=\figdir/dff-sch.ps,height=0.3\textheight}}
  \hfill
  {\psfig{figure=\figdir/dff-lay.eps,height=0.3\textheight}}
}
\caption{Circuit diagram of a D-flipflop and its layout on the fishbone image.}
\label{f-dff}
\end{figure}



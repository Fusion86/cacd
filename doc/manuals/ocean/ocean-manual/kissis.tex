% -*- latex -*-

\chapter{Logic Synthesis with 'kissis'}
\label{kissis}
\index{kissis@\tool{kissis}|(bold}
\index{automatic!logic synthesis|see{\tool{kissis}}}

\protect\tool{SIS} (Berkeley Logic Synthesis System) 
\index{SIS@\tool{SIS}}
is an interactive tool
for synthesis and optimization of sequential circuits. Given a state
transition table, a logic level description of a sequential circuit or PLA
(programmable logic array) description, it produces an optimized net-list in
the target technology while preserving the sequential input-output behavior.

\protect\tool{Kissis} is a front-end to \protect\tool{SIS} which allows
automatic generation of a circuit for given STG (state transition graph) or
PLA. For this purpose it runs SIS in batch mode. The resulting circuit is
automatically placed in
\protect\tool{Nelsis} database. The following steps are performed:

\begin{itemize}
\item state minimization using program called stamina (STG input only).
\item state assignment using program called nova (STG input only).
\item combinatorial optimization.
\item technology mapping. 
\item retiming (STG input only and when -r specified).
\item conversion from blif output format to sls (Switch Level Simulator) using
program \protect\tool{blif2sls}\index{blif2sls@\tool{blif2sls}}.
\item conversion from sls format to the \protect\tool{Nelsis} database using
\protect\tool{csls} program.  
\end{itemize}
 
We cannot distribute the \smc{sis} package with the \smc{ocean}
tools, unfortunately. You can retrieve \smc{sis} yourself using
anonymous ftp (host ic.berkeley.edu, directory pub/sis) or contact us.

\section{How to use \protect\tool{kissis}?}

You start the tool from command line in the following way:

\type{kissis $[$options$]$ $<$filename$>$}	

where $<$filename$>$ is the input file name.


\section{Legal options.}

The kissis can be started with the following options:

\nopagebreak
-h display help info.

-c run combinatorial synthesis (PLA input format).

-k run sequential synthesis (kiss2 input format).

-b run sequential synthesis (blif input format).

-r run retiming (default not)



If none of these options is specified then kind of input 
is determined based on filename extension (default \fname{.kiss2}).  
If  options are specified than extension does not
matter. Default extension is \fname{.kiss2}.

\section{What files are required before I can call \protect\tool{kissis}?}

To run properly the command requires that the following
files are present in your current directory. If the \smc{ocean} system was
properly installed, all the default files should be in the right position.

\begin{itemize}
\item \file{proto\_file}  - a file with sls prototypes (see \protect\tool{blif2sls}).
\item \file{$<$cellname$>.<$kiss2|pla|blif$>$} - input file with STG (PLA)  
                   description.

\end{itemize}
Other files:

\begin{itemize}
\item \file{\$OCEAN/celllibs/\$OCEANPROCESS/<lib\_name>.genlib} - a file with a
library description (to be used during technology mapping).
\item \file{<cell\_name>\_out.blif} -  intermediate file (output in blif format).

\item \file{<cell\_name>.sls} - intermediate file (output in sls format).

\item \file{<cell\_name>.sta} - output from \protect\tool{blif2sls} used by 
                  \protect\tool{simeye}.

\item \file{sis\_logfile} - output from \protect\tool{SIS} (look here if
something goes wrong). 

\end{itemize}

For an example of running \tool{kissis}, please have a look in 
section~\ref{kissistutorial} on page~\pageref{kissistutorial}.

\section{More info about \protect\tool{SIS}.}
\index{SIS@\tool{SIS}}
For more information about \protect\tool{SIS} we refer to the following
document:

"SIS: A System for Sequential Circuit Synthesis" Ellen M. Santovich, Kanwar Jit
Singh, Luciano Lavagno, Cho Moon, Rajeev Murgai, Alexander Saldanha, Hamid
Savoj,  Paul R. Stephan, Robert K. Brayton, Alberto Sangiovanni-Vincentelli.
Department of Electrical Engineering and Computer Science, University of
California, Berkeley, CA 94720. (1992)

It is present in the ocean documentation distribution.

\index{kissis@\tool{kissis}|)}



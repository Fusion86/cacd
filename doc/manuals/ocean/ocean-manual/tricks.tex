% -*- latex -*-

\chapter{Using OCEAN: tips and tricks} 

\section{Simple troubleshooting: the seadif database}
\index{seadif@\fname{seadif} database}
'Seadif' is a special representation for grid-based sea-of-gates layout. It is
stored in the subdirectory 'seadif' of every project.  The seadif
representation of a cell is generated automatically, so you don't need to worry
about it. In some cases, however, the seadif representation cold become corrupt
or or incomplete. This might happen after a program crash or two users writing
at the same time. Fortunately, the seadif representation is redundant, so it
can be re-created automatically as soon as you run \tool{madonna} or
\tool{trout} the next time. Therefore, the following command will solve most
of your troubles:\cd{/myproject}
\type{rm -rf seadif}
This command removes the entire seadif directory and its contents.
Only for the case that you are importing cells from this directory, you should
re-create the seadif database manually using the command \tool{nelsea}:
\type{nelsea}

\section{Assembling cells from various projects}

Combining cells from different projects is quite easy.  Using the commands
\tool{addproj} and \tool{impcell} you can import remote cells into your
project. To make sure that everything will work smoothly, please comply with
the following three simple rules. Suppose that the want to make a cell in the
project called 'top-project', which needs imported son-cells from remote
projects 'project1', 'project2' and 'project3'.  In order to make things more
realistic (and complicated), suppose also that in some of the imported projects
also other projects are imported: the son-cell in 'project1' imports
grandchildren from projects 4 and 5, and the cell in 'project2' imports some
kids from 'project5'.  Confused?  The situation is shown in
figure~\ref{extree}.

\begin{figure}
\begin{verbatim}
                         top-project
                           /  |  \
                          /   |   \
                         /    |    \
                  project1    |    project3
                    /   \   project2
                   /     \    |
                  /       \   |
              project4   project5
\end{verbatim}
\caption{\label{extree}
Example of a project hierarchy}
\end{figure}

\begin{itemize}
\item[{\bf Rule 1}:]
Add ALL projects in the tree with \tool{addproj}, also the projects which are imported
indirectly.
\end{itemize}

For the example, we should not only add projects 1, 2 and 3, but also
the other ones with grandchildren or even the most remote relatives in
the tree:\cd{/top-project}
\begin{itemize}%
\item
{\tt /user/hillary\thedir~\%~ addproj /users/neuzeltje/project1}%
\item
{\tt /user/hillary\thedir~\%~ addproj /users/calimero/project2}%
\item
{\tt /user/hillary\thedir~\%~ addproj /users/prince/project3}%
\item
{\tt /user/hillary\thedir~\%~ addproj /users/neuzeltje/project4}%
\item
{\tt /user/hillary\thedir~\%~ addproj /users/hillary/project5}%
\end{itemize}

With the command \tool{impcell}, on the other hand, you only need to import
the son-cells which you really need in 'top-project'.

\begin{itemize}
\item[{\bf Rule 2}:]
All project names in the imported tree must be unique. 
\end{itemize}

For example, you will run into troubles if you try to import both the
projects '/users/neuzeltje/memory' and '/users/calimero/memory',
because the project name "memory" is used twice. You can prevent name
clashes simply by renaming a project. See the section~\ref{renamingprojects}
(on page~\pageref{renamingprojects}) to find out how to do that.

\begin{itemize}
\item[{\bf Rule 3}:]
All cells in the tree must have a proper seadif representation. 
\end{itemize}

For every son-cell there must be a proper seadif-representation. Normally this is
performed automatically, but problems with the 'seadif' representation are more
likely with imported cells.
\tool{Madonna} or \tool{trout} will tell you exactly which seadif
representation is missing.  The cure is easy. First you go to the project about
which \tool{madonna} (or \tool{trout}) complains. Then you use the program
\tool{nelsea} to force an update of the seadif representation:\cd{/project5}
\index{nelsea@\tool{nelsea}}
\typeb{cd /users/neuzeltje/project4}{nelsea}

You must have write permissions in the directory where you run
\tool{nelsea}.  Without arguments, the command \tool{nelsea} will create or
update the seadif representation of all cells in the project. If there
are many unimportant cells, you can speed things up by specifically
updating one cell. For instance, the command:
\type{nelsea myadder}
will update cell \fname{myadder} and all the children (and grandchildren, if
necessary) of \fname{myadder}. 


\section{Renaming, copying or moving projects: how to proceed}
\index{project!renaming}
\index{project!copying}
\label{renamingprojects}
In some cases it might be necessary to rename project, to duplicate
one, or to move a project to a different place in the file system. In
order to prevent unpleasant surprises, follow the guidelines below.

\begin{itemize}
\item[{\bf Step 1}:]
Make sure that no other project is importing from the project.
\end{itemize}

Since the project will be renamed, any old references to it will point
"ins Blaue hinein", like the Germans say. Generally you should only move 
top-level projects. 

\begin{itemize}
\item[{\bf Step 2}:]
Move or copy the project
\end{itemize}
Within the same file system (that is, the same disk), you can use the
\smc{unix} command \tool{mv}:\cd{}
\type{mv oldproject newproject}

This is the fastest way. To move to a different disk, or if you want
to make a copy, use \tool{cp -r}:
\type{cp -r oldproject /otherdisk/prince/newproject}

\begin{itemize}
\item[{\bf Step 3}:]
Clean up the seadif representation
\end{itemize}

After moving or copying, the subdirectory 'seadif' of the project
still contains old name and path of the project. There is a simple 
way to solve this. First go to the project, remove the existing 
seadif representation, and then create a new one:\cd{}
\type{cd newproject} \cd{/newproject}
\typeb{rm -rf seadif}{nelsea}

Also on other situations where you get strange database errors, you 
can do this to clean things up.


% abstract.tex

\chapter*{Abstract}
\addcontentsline{toc}{chapter}{Abstract}
The current version of the layout-to-circuit extractor SPACE does not have a
user friendly interface to add new process descriptions or change existing
descriptions. This is currently a specialized job.

\bigskip \noindent
This report describes the design and implementation of a user-friendly tool
that can be used to add and manage the technology files that describe a
process.

\bigskip \noindent
Flexibility is an important requirement. It must be possible to easily add new
technology files to the interface. Another important requirement is platform
independence. It must be possible to use the platform on many different flavors
of Unix.

\bigskip \noindent
To meet the demand of flexibility the application uses a configuration file in
which the user interface and the format of the technology files is specified.
The developed application uses the information in the configuration file to
create a user interface and to generate the required technology files.

A parser has been created that can read the configuration file and build a
graphical user interface from the information it encounters. The values entered
by the user into this user interface are then used by the generators specified
in the configuration file to generate the requested technology files.

\bigskip \noindent
The Qt toolkit from TrollTech ensures platform independence of the graphical
user interface. This popular toolkit is available for many types of Unix,
Linux, Windows and even for embedded environments.

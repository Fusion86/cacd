% demands.tex
%%%%%%%%%%%%%%%%%%%%%%%%%%%%%%%%%%%%%%%%%%%%%%%%%%%%%%%%%%%%%%%%%%%%%%%%%%%%%%

\chapter{Application requirements}
\label{chap:demands}

%%%%%%%%%%%%%%%%%%%%%%%%%%%%%%%%%%%%%%%%%%%%%%%%%%%%%%%%%%%%%%%%%%%%%%%%%%%%%%
%\section{Introduction}
As a first step in the design process, the demands and constraints for the
application must be charted. In this chapter the most important demands and
constraints are given.

Firstly, the constraints resulting from the environment in which the
application must function are enumerated in Section
\ref{sect:demands:environment}). Secondly, the functional demands are listed in
Section \ref{sect:demands:functional}. Documentation is also important: demands
related to documentation can be found in Section
\ref{sect:demands:documentation}. Some general wishes are mentioned in Section
\ref{sect:demands:wishes}.

%%%%%%%%%%%%%%%%%%%%%%%%%%%%%%%%%%%%%%%%%%%%%%%%%%%%%%%%%%%%%%%%%%%%%%%%%%%%%%
\section{Application environment constraints}
\label{sect:demands:environment}
\subsection{Platform}
It must be possible to compile the application on various platforms since SPACE
can also be compiled on various platforms. The following platforms must be
supported:
\begin{itemize}
\item Solaris
\item HP UX
\item Linux
\end{itemize}
Any input and/or output files that are read or written by the application must
also be platform independent. This means for example, that it must be possible
to read a file saved with the Solaris version into the Linux version and
vice-versa.

\subsection{Integration}
The application must integrate correctly with the rest of the SPACE package.
This mainly means that the application should honor the directory structure and
the environment settings used by SPACE.

%%%%%%%%%%%%%%%%%%%%%%%%%%%%%%%%%%%%%%%%%%%%%%%%%%%%%%%%%%%%%%%%%%%%%%%%%%%%%%
\section{Functional demands}
\label{sect:demands:functional}
\subsection{Technology file generation}
It should be possible to generate technology files with a plain-text file
format. There must also be a mechanism that will allow the addition (or
removal) of technology file formats. As a proof of concept the application must
be able to generate the following subset of technology files:
\begin{itemize}
\item \emph{maskdata}, which defines the layers present in the process and the
colors used to represent them in the programs and their output.
\item \emph{space.xxx.s}, the element definition files used by SPACE.
\item \emph{space.xxx.p}, the parameter files used by space.
\item \emph{bmlist.gds}, which provides a mapping between the GDS layer format and
the format used by SPACE.
\item \emph{xspicerc}, a control file that specifies which models are used for
the devices.
\end{itemize}

\subsection{Flexibility}
Flexibility is very important. Recompilations as a result of adding, changing
or removing a technology file format should be kept to a minimum to decrease
the maintenance cost of the application.

%%%%%%%%%%%%%%%%%%%%%%%%%%%%%%%%%%%%%%%%%%%%%%%%%%%%%%%%%%%%%%%%%%%%%%%%%%%%%%
\section{Documentation demands}
\label{sect:demands:documentation}
The following documentation must be created:
\begin{itemize}
\item Source code documentation
\item Installation manual
\item Programmers manual
\item Maintainers manual
\end{itemize}

%%%%%%%%%%%%%%%%%%%%%%%%%%%%%%%%%%%%%%%%%%%%%%%%%%%%%%%%%%%%%%%%%%%%%%%%%%%%%%
\section{Additional wishes}
\label{sect:demands:wishes}
The user interface must be consequent and consistent. This means the user
interface should conform to the currently accepted standard.

A ``help'' facility would be welcomed. It should at least be possible to show
``tooltips'' (popup hints).

\bigskip \noindent
If possible, the application will be implemented in C++.

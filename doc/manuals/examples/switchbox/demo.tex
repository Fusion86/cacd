\chapter{switchbox Example of Space Tutorial}
\section{Introduction}
\label{SOintro}
In this example, we will be studying a switchbox design.
We will see how the Space system can be used.
See also the "Space Tutorial" document.
\\[1 ex]
The layout looks as follows, using the layout editor \io{dali} (see \manualpage{dali}):

\begin{figure}[h]
\centerline{\epsfig{figure=switchbox/swbox.eps, width=11cm}}
\end{figure}

\section{Files}
This tutorial is located in the directory \CACDTOP{demo/switchbox}.
Initially, it contains the following files:
\begin{filelist}
\item[README] A file containing information about the demo.
\item[switchbox4.gds] The GDS2 file of the switchbox layout.
\item[switchbox4\_f.gds] The GDS2 file of the flat switchbox layout.
\item[swbox\_ref.spc] SPICE file of the switchbox reference circuit.
\item[script.sh] A file containing the commands for executing all
steps of the demo in sequence.
\end{filelist}

\section{Running a Hierarchical Extraction}
Before we can run a hierarchical extraction on the "switchbox" layout, we need a project directory.
The file \io{script.sh} is a batch file for running all the commands for this example, which you can use.
First, it changes the current working directory '.' into a project directory:
\small
\begin{Verbatim}
% mkpr -p scmos_n -l 1 .
\end{Verbatim}
\normalsize
We use a lambda value (option \io{-l}) of $1 \mu m$.
For the CMOS technology, we use the \io{scmos\_n} process from the technology library.

Second, the layout description is put into the project database.
The layout is supplied in a GDS2 file, which can be converted to
internal database format with the \io{cgi} program:
\small
\begin{Verbatim}
% cgi switchbox4.gds
\end{Verbatim}
\normalsize
Third, we can list the contents of the project database.
To get a hierarchical cell listing, use the \io{dblist} program as follows:
\small
\begin{Verbatim}
% dblist -h
\end{Verbatim}
\normalsize
\small \begin{Verbatim}[frame=single]

layout:
1 - switchbox4      (4)
    2 - dec1of4           4 (6)
        3 - nan4rout          1 (0)
        3 - nan3              4 (0)
        3 - dubinv            1 (0)

circuit:

floorplan:

\end{Verbatim}
\normalsize
You see, that the layout of the "switchbox" design is hierarchical.
There are 3 hierarchical levels.
The top cell is \io{switchbox4}, it calls 4 times sub cell \io{dec1of4}.
The cell \io{dec1of4} has also 3 sub cells (the cell \io{nan3} is used 4 times).
To inspect the layout of the design, use the layout editor/viewer \io{dali}.
Type the following command:
\small
\begin{Verbatim}
% dali
\end{Verbatim}
\normalsize
Click on the "DB\_menu" menu item and use the "read\_cell" command
to read the different cells.
To see more hierarchical levels of the cell, use the number keys.
Type for example '3' to expand the cell to 3 levels deep.
\\[1 ex]
Now, we can extract a circuit description for the layout of each cell.
Because the layout is hierarchical,
we can perform a hierarchical extraction of the \io{switchbox4} top cell.
Each cell is individual extracted, because the flat option is not used.
We start the extractor with the verbose option, as follows:
\small
\begin{Verbatim}
% space -v switchbox4
\end{Verbatim}
\normalsize
Note that \io{space} is using the default technology file \io{space.def.s} and
the default parameter file \io{space.def.p} of the \io{scmos\_n} example process.
The output of the hierarchical extraction is listed below.

\small \begin{Verbatim}[frame=single]
Version 5.3.1, compiled on Fri Feb 03 12:45:53 GMT 2006
See http://www.space.tudelft.nl
parameter file: $ICDPATH/share/lib/process/scmos_n/space.def.p
technology file: $ICDPATH/share/lib/process/scmos_n/space.def.t
extract hierarchy of switchbox4
preprocessing nan4rout (phase 1)
preprocessing nan4rout (phase 2 - removing overlap)
extracting nan4rout

extraction statistics for layout nan4rout:
	capacitances        : 0
	resistances         : 0
	nodes               : 10
	mos transistors     : 8
	...

preprocessing nan3 (phase 1)
preprocessing nan3 (phase 2 - removing overlap)
extracting nan3

extraction statistics for layout nan3:
	capacitances        : 0
	resistances         : 0
	nodes               : 8
	mos transistors     : 6
	...

preprocessing dubinv (phase 1)
preprocessing dubinv (phase 2 - removing overlap)
extracting dubinv

extraction statistics for layout dubinv:
	capacitances        : 0
	resistances         : 0
	nodes               : 6
	mos transistors     : 4
	...

preprocessing dec1of4 (phase 1)
preprocessing dec1of4 (phase 2 - removing overlap)
extracting dec1of4

extraction statistics for layout dec1of4:
	capacitances        : 0
	resistances         : 0
	nodes               : 51
	mos transistors     : 0
	...

preprocessing switchbox4 (phase 1)
preprocessing switchbox4 (phase 2 - removing overlap)
extracting switchbox4

extraction statistics for layout switchbox4:
	capacitances        : 0
	resistances         : 0
	nodes               : 42
	mos transistors     : 0
	...

overall resource utilization:
	memory allocation  : 0.233 Mbyte
	user time          :         0.0
	system time        :         0.0
	real time          :         7.1   0%

space: --- Finished ---
\end{Verbatim}
\normalsize
We make again a hierarchical cell listing of the project database.
\small
\begin{Verbatim}
% dblist -h
\end{Verbatim}
\normalsize
\small \begin{Verbatim}[frame=single]

layout:
1 - switchbox4      (4)
    2 - dec1of4           4 (6)
        3 - nan4rout          1 (0)
        3 - nan3              4 (0)
        3 - dubinv            1 (0)

circuit:
1 - switchbox4      (4)
    2 - dec1of4           4 (6)
        3 - dubinv            1 (4)
        3 - nan3              4 (6)
        3 - nan4rout          1 (8)

floorplan:

\end{Verbatim}
\normalsize
We see, that the extracted circuit cells have also a hierarchical structure.

\section{Running a Circuit Comparison}
The following demonstrates the use of the circuit comparison program \io{match}.

First, add a reference circuit description for the "switchbox" circuit
to the database using the program \io{cspice}.
\small
\begin{Verbatim}
% cspice swbox_ref.spc
\end{Verbatim}
\normalsize
\small \begin{Verbatim}[frame=single]
File swbox_ref.spc:
Parsing network: swbox_ref
\end{Verbatim}
\normalsize
This reference circuit is a flat circuit description of the "switchbox".
Compare this reference circuit with the hierarchical description, type:
\small
\begin{Verbatim}
% match swbox_ref switchbox4
\end{Verbatim}
\normalsize
\small \begin{Verbatim}[frame=single]

match: Succeeded.

\end{Verbatim}
\normalsize
If you want, you can use the \io{-fullbindings} option to get more information.
\\[1 ex]
You can also perform a flat extraction of the "switchbox4" cell and do the compare again.
This must also give a succesfull match.
\small
\begin{Verbatim}
% space -F switchbox4
% match swbox_ref switchbox4
\end{Verbatim}
\normalsize
Second, use \io{cgi} to put a flat version of the "switchbox" layout into the
database and extract the circuit and do the compare again.
\small
\begin{Verbatim}
% cgi switchbox4_f.gds
% space switchbox4_f
% match switchbox4 switchbox4_f
\end{Verbatim}
\normalsize
This must also give a succesfull \io{match} result.
Now, use the program \io{match} to compare the extracted circuit against the
reference circuit.
\small
\begin{Verbatim}
% match swbox_ref switchbox4_f
\end{Verbatim}
\normalsize
\small \begin{Verbatim}[frame=single]

match: Succeeded.

\end{Verbatim}
\normalsize
The result shows that the circuits are identical.
\\[1 ex]
We use the flat layout, because it is more easy to change.
Now, make an error in the layout of "switchbox4\_f" using \io{dali} (e.g. remove
some metal) and run the programs \io{space} and \io{match} again.
\small
\begin{Verbatim}
% dali switchbox4_f
% space switchbox4_f
% match swbox_ref switchbox4_f
\end{Verbatim}
\normalsize
\small \begin{Verbatim}[frame=single]

match: Failed.

\end{Verbatim}
\normalsize
The result shows that the circuits are not more identical.
\\[1 ex]
Now, use the option \io{-fullbindings} to see which network parts are matched.
\small
\begin{Verbatim}
% match -fullbindings swbox_ref switchbox4_f
\end{Verbatim}
\normalsize

\section{Layout Back-Annotation}
You can use the layout back-annotation option \io{-x} of \io{space} to see with the layout viewer \io{dali}
which conductors are high-lighted.
Using \io{match} and after that the high-light tool \io{highlay}, you can write
a cell "HIGH\_OUT" to inspect the unmatched condcutors.
See also section 9 of the "Space Tutorial" document.

\chapter{sub3term Example of Substrate Resistance Extraction}
\section{Introduction}
\label{STintro}
In this example, we will be studying a configuration of 3 substrate terminals.
We will see how Space is used to compute the 3D substrate resistances.
See also the "Space Substrate Resistance Extraction User's Manual".
\\[1 ex]
The layout looks as follows (such a figure can be printed using \io{getepslay}, see
\manualpage{getepslay}):

\begin{figure}[h]
\centerline{\epsfig{figure=sub3term/sub3term.eps, width=8cm}}
\end{figure}

\section{Files}
This tutorial is located in the directory \CACDTOP{demo/sub3term}.
Initially, it contains the following files:
\begin{filelist}
\item[README] A file containing information about the demo.
\item[sub3term.gds] The layout of the sub3term design.
\item[elem.s] The technology file for Space. See Section \ref{STtech}.
\item[param.p] The parameter settings file for Space. See Section \ref{STparam}.
\item[script.sh] A file containing the commands for executing all
steps of the demo in sequence.
\end{filelist}

\section{Technology File}
\label{STtech}
The technology file that we will use is the file \io{elem.s}, see listing below.

\small \begin{Verbatim}[frame=single]
listing of file elem.s
   3    #
   4    # space element definition file for metal substrate terminals
   5    #
   6
   7    colors :
   8        cmf   blue
   9        @sub  pink
  10
  11    conductors :
  12      # name     : condition : mask : resistivity : type
  13        cond_cmf : cmf       : cmf  : 0.0         : m
  14
  15    contacts :
  16      # name     : condition : lay1 lay2 : resistivity
  17        cont_cmf : cmf       : cmf  @sub : 0.0
  18
  19    sublayers :
  20      # name       conductivity  top
  21        epi        6.7           0.0
  22        substrate  2.0e3        -7.0
  23
  24    #EOF
\end{Verbatim}
\normalsize
Please note that the masks to be used in this file have already been defined
in the \io{maskdata} file in the process library \CACDTOP{lib/process}.
In this example, we will only use the first metal mask (\io{cmf}).

When we look to the file, we first see on line \io{7} the \io{colors} specification.
The color of the \io{cmf} mask is chosen to be \io{blue}.
The color for the \io{@sub} mesh is chosen to be \io{pink}.
Note that this info is only used by the visualization tool \io{Xspace}.

Starting on line \io{11} we find the \io{conductors} definitions.
This is an important section, because without this we can not specify
element pins and contacts.
The conductor value is for this example unimportant,
because we don't extract interconnect resistances.

Starting on line \io{15} we find the \io{contacts} definitions.
This is also an important section,
because here is the connection between first metal and the substrate specified.
These nodes are also called substrate terminals.
However, substrate terminals can also be bulk regions or pins of capacitances.

For 3D substrate resistance extraction, we need only one additional section.
Starting on line \io{19} we find the \io{sublayers} definitions.
These definitions specify for each substrate layer the conductivity (in $S / m$)
and the starting position (in $\mu m$).
Note that the last layer (on line \io{22}) extends to minus infinity.

\section{Parameter File}
\label{STparam}
The parameter file \io{param.p} is controlling the extraction, see listing below.

\small \begin{Verbatim}[frame=single]
listing of file param.p
  3
  4  BEGIN sub3d
  5  be_mode         0c   # piecewise constant collocation
  6  max_be_area     1.0  # max. size of interior elements in sq. microns
  7  edge_be_ratio   0.05 # max. size edge elem. / max size inter. elem.
  8  edge_be_split   0.2  # split fraction for edge elements
  9  be_window     inf    # infinite window, all resistances
 10  END sub3d
 11
 12  disp.save_prepass_image   on
\end{Verbatim}
\normalsize
In the above listing, we see a \io{sub3d} block from line \io{4} to line \io{10}.
These parameters are only used for controlling the substrate 3D extraction.
The \io{max\_be\_area} (in $\mu m^2$) and \io{be\_window} (width in $\mu m$) parameter
needs always to be specified.
On line \io{5}, you find also the \io{be\_mode} parameter.
However, this boundary element mode is the default.
On line \io{9}, you see that the \io{be\_window} parameter is chosen to be infinity.
That means, that all substrate terminal nodes are in the same strip,
and are all connected with each other.
See for an explanation of the parameters also the "Space Substrate Extraction User's Manual".

On line \io{12}, you find a parameter for the visualization tool \io{Xspace}.
It means, that the image of the prepass may not be cleared
when starting drawing the image of the last pass.
See for an explanation also the "Xspace User's Manual".

\section{Running the Extractor}
The file \io{script.sh} is a batch file for running all the commands for this example.
First, it changes the current working directory '.' into a project directory:
\small
\begin{Verbatim}
% mkpr -p scmos_n -l 0.1 .
\end{Verbatim}
\normalsize
We use a lambda value (option \io{-l}) of $0.1 \mu m$.
For the CMOS technology, we use the \io{scmos\_n} process from the technology library.
We use the mask names as defined in the \io{maskdata} file of the library.
But, we are not using the default technology
file \io{space.def.s} and parameter file \io{space.def.p} of the library.
Thus, the local technology file \io{elem.s} needs to be compiled with \io{tecc} to the format
(a \io{.t} file) which is used by the extractor:
\small
\begin{Verbatim}
% tecc elem.s
\end{Verbatim}
\normalsize
Second, the layout description is put into the project database.
The layout is supplied in a GDS2 file, which can be converted to
internal database format with the \io{cgi} program:
\small
\begin{Verbatim}
% cgi sub3term.gds
\end{Verbatim}
\normalsize
Now, we can extract a circuit description for the layout of the \io{sub3term} cell, as follows:
\small
\begin{Verbatim}
% space3d -vB -E elem.t -P param.p sub3term
\end{Verbatim}
\normalsize
We use the verbose option (\io{-v}) to see what the extractor program is doing.
Second, we use option \io{-B} to perform a 3D substrate resistance extraction.
The substrate resistances are calculated, using a 3D mesh method (BEM).
The calculated resistance values are assigned between substrate terminals and between the substrate
terminals and the substrate plane (node SUBSTR).
\\[1 ex]
The extracted circuit can be inspected using the \io{xspice} program.
We use option \io{-a} to get alpha-numeric node names.
\small
\begin{Verbatim}
% xspice -a sub3term
\end{Verbatim}
\normalsize
The output of the SPICE circuit is listed below.

\small \begin{Verbatim}[frame=single]
   1    sub3term
   2
   3    * Generated by: xspice 2.39 25-Jan-2006
   4    * Date: 21-Jun-06 14:56:15 GMT
   5    * Path: /users/simon/sub3term
   6    * Language: SPICE
   7
   8    * circuit sub3term c b a
   9    r1 c a 1.265872meg
  10    r2 c b 679.1891k
  11    r3 c SUBSTR 75.86724k
  12    r4 a b 679.1891k
  13    r5 a SUBSTR 75.86724k
  14    r6 b SUBSTR 49.44378k
  15    * end sub3term
\end{Verbatim}
\normalsize
Below, you find a drawing of the listing of the extracted circuit.

\begin{figure}[h]
\centerline{\epsfig{figure=sub3term/circuit.eps, width=9cm}}
\end{figure}

Note that there are substrate resistances between all nodes of the circuit.
This happens, because the chosen \io{be\_window} has a width of infinity.
Each node has also a substrate resistance to the substrate plane (node \io{SUBSTR}).

\section{Running the Visualization Tool}
When you want to see the boundary element mesh,
which is used for the substrate terminal regions,
you must use the visualization tool \io{Xspace}.
It runs the \io{space3d} extractor in the background and requests for graphical output.
The extractor writes this graphical output to a "display.out" file, which \io{Xspace} is reading.
The advantage of the output file is, that you can playback it again without running the extractor.
And it makes zooming and panning on the image very easy.
For additional information, consult the "Xspace User's Manual".

Note that you can playback only the file "display.out" and that \io{Xspace} needs the color
information of the used technology file.
To show something, you need to set items in the "Display" menu.
Note that you can only show items, which are written to the file "display.out".
To start a playback, use hotkey 'a' (or click on "extract again" in the "Extract" menu).

Now, we shall extract a circuit description for the layout of the \io{sub3term} cell,
using the visualization tool as follows:
\small
\begin{Verbatim}
% Xspace -vB -E elem.t -P param.p sub3term
\end{Verbatim}
\normalsize
You don't need to set the cell name in the "Database" menu and
also don't need to set the extract options in the "Options" menu,
because that is already specified on the command line.

\noindent
You need only to set some display options.
Go to the "Display" menu and click on button "DrawBEMesh".
To run the extractor, go to the "Extract" menu and click on "extract" (or use hotkey 'e').

\noindent
You get the following picture:
\begin{figure}[h]
\centerline{\epsfig{figure=sub3term/xspace1.eps, width=12cm}}
\end{figure}

\noindent
You see three substrate terminals, which are meshed into a number of area's.
The complete area of terminal "a" and "c" is $1.0 \mu m^2$.
The area of terminal "b" is $2.0 \mu m^2$.
\\[1 ex]
When you look to the space parameter file "param.p",
you see that the maximum size of an interior element may be $1.0 \mu m^2$
(parameter \io{max\_be\_area}).
The maximum size of an edge element, however, may be $0.05 * 1.0 \mu m^2$
(parameter \io{edge\_be\_ratio}).
When we edit the file "param.p" and change the \io{edge\_be\_ratio} to its default value of $1.0$
(or make the line to be a comment line by placing a '\#' before it),
then we shall see what happens.
\\[1 ex]
We can now run the visualization tool again.
Or better, when you have put it in the background, hit the extract key 'e' again.
\\[1 ex]
You get the following picture:
\newpage
\begin{figure}[h]
\centerline{\epsfig{figure=sub3term/xspace2.eps, width=10cm}}
\end{figure}

\noindent
And indeed, we get another mesh for the 3 substrate terminals.
\\[1 ex]
Below you see what happens, if we don't change the \io{edge\_be\_ratio},
but the \io{edge\_be\_split} parameter instead (set this fraction to its default value of $0.5$).
You get the following picture:
\begin{figure}[h]
\centerline{\epsfig{figure=sub3term/xspace3.eps, width=10cm}}
\end{figure}

\noindent
The accuracy of the extraction result is depended of the mesh you chose.
And, how finer the mesh, how more computation time is needed to calculate the result.

\chapter{suboscil Example of Substrate Resistance Extraction}
\section{Introduction}
\label{SOintro}
In this example, we will be studying a CMOS ring oscillator.
We will see how Space is used to compute substrate resistances.
First using the fast interpolation method and
second using the more accurate 3D BE-method.
See also section 5.5 of the "Space Substrate Resistance Extraction User's Manual".
\\[1 ex]
The layout looks as follows, using the layout editor \io{dali} (see \manualpage{dali}):

\begin{figure}[h]
\centerline{\epsfig{figure=suboscil/oscil.eps, width=9cm}}
\end{figure}

\section{Files}
This tutorial is located in the directory \CACDTOP{demo/suboscil}.
Initially, it contains the following files:
\begin{filelist}
\item[README] A file containing information about the demo.
\item[oscil.gds] The layout of the oscillator design.
\item[oscil.cmd] Command file for circuit simulation.
\item[tech.s] The technology file for Space. See Section \ref{SOtech}.
\item[tech2.s] Alternate technology file for more substrate terminals.
\item[param.p] The parameter settings file for Space. See Section \ref{SOparam}.
\item[param2.p] The parameter settings file for Space.
\item[jun.lib] Local model library for xspice.
\item[xspicerc] Init file for xspice.
\item[script.sh] A file containing the commands for executing all
steps of the demo in sequence.
\end{filelist}

\section{Technology File}
\label{SOtech}
The technology file that we will use is the file \io{tech.s}, see listing below.

\small \begin{Verbatim}[frame=single]
listing of file tech.s
     ...
 17
 18  unit resistance    1     # ohm
 19  unit c_resistance  1e-12 # ohm um^2
 20  unit a_capacitance 1e-6  # aF/um^2
 21  unit e_capacitance 1e-12 # aF/um
 22  unit capacitance   1e-15 # fF
     ...
 38
 39  conductors :
 40    # name    : condition     : mask : resistivity : type
 41      cond_mf : cmf           : cmf  : 0.045       : m    # first metal
 42      cond_ms : cms           : cms  : 0.030       : m    # second metal
 43      cond_pg : cpg           : cpg  : 40          : m    # poly interconnect
 44      cond_pa : caa !cpg !csn : caa  : 70          : p    # p+ active area
 45      cond_na : caa !cpg  csn : caa  : 50          : n    # n+ active area
 46      cond_well : cwn         : cwn  : 0           : n    # n well
 47
 48  fets :
 49    # name : condition    : gate d/s : bulk
 50      nenh : cpg caa  csn : cpg  caa : @sub  # nenh MOS
 51      penh : cpg caa !csn : cpg  caa : cwn   # penh MOS
 52
 53  contacts :
 54    # name   : condition         : lay1 lay2 : resistivity
 55      cont_s : cva cmf cms       : cmf  cms  :   1   # metal to metal2
 56      cont_p : ccp cmf cpg       : cmf  cpg  : 100   # metal to poly
 57      cont_a : cca cmf caa !cpg cwn !csn
 58             | cca cmf caa !cpg !cwn csn
 59                                 : cmf  caa  : 100   # metal to active area
 60      cont_w : cca cmf cwn csn   : cmf  cwn  :  80   # metal to well
 61      cont_b : cca cmf !cwn !csn : cmf  @sub :  80   # metal to subs
 62
 63  junction capacitances ndif :
 64    # name    :  condition                 : mask1 mask2 : capacitivity
 65      acap_na :  caa       !cpg  csn !cwn  :  caa @gnd : 100  # n+ bottom
 66      ecap_na : !caa -caa !-cpg -csn !-cwn : -caa @gnd : 300  # n+ sidewall
 67
 68  junction capacitances nwell :
 69      acap_cw :  cwn                  :  cwn @gnd : 100  # bottom
 70      ecap_cw : !cwn -cwn             : -cwn @gnd : 800  # sidewall
 71
 72  junction capacitances pdif :
 73      acap_pa :  caa       !cpg  !csn cwn      :  caa cwn : 500 # p+ bottom
 74      ecap_pa : !caa -caa !-cpg !-csn cwn -cwn : -caa cwn : 600 # p+ sidewall
 75
 76  capacitances :
 77    # polysilicon capacitances
 78      acap_cpg_sub :  cpg                !caa !cwn :  cpg @gnd : 49
 79      acap_cpg_cwn :  cpg                !caa  cwn :  cpg cwn  : 49
 80      ecap_cpg_sub : !cpg -cpg !cmf !cms !caa !cwn : -cpg @gnd : 52
 81      ecap_cpg_cwn : !cpg -cpg !cmf !cms !caa  cwn : -cpg cwn  : 52
 82
 83    # first metal capacitances
 84      acap_cmf_sub :  cmf           !cpg !caa !cwn :  cmf @gnd : 25
 85      acap_cmf_cwn :  cmf           !cpg !caa  cwn :  cmf cwn  : 25
 86      ecap_cmf_sub : !cmf -cmf !cms !cpg !caa !cwn : -cmf @gnd : 52
 87      ecap_cmf_cwn : !cmf -cmf !cms !cpg !caa  cwn : -cmf cwn  : 52
 88
 89      acap_cmf_caa :  cmf      caa !cpg !cca :  cmf  caa : 49
 90      ecap_cmf_caa : !cmf -cmf caa !cms !cpg : -cmf  caa : 59
 91
 92      acap_cmf_cpg :  cmf      cpg !ccp :  cmf  cpg : 49
 93      ecap_cmf_cpg : !cmf -cmf cpg !cms : -cmf  cpg : 59
     ...
124
125  sublayers :
126    # name       conductivity  top
127      substrate  6.7           0.0
128
129  selfsubres :
130  #  Generated by subresgen on 10:56:15 13-5-2003
131  #    area    perim          r   rest
132       0.64      3.2    81286.8   0.01  # w=0.8 l=0.8
133       0.64        4   73678.39   0.01  # w=0.4 l=1.6
134       0.48      3.2   88205.12   0.01  # w=0.4 l=1.2
135       2.56      6.4    40643.4   0.01  # w=1.6 l=1.6
136       2.56        8    36839.2   0.01  # w=0.8 l=3.2
137       1.92      6.4   44102.56   0.01  # w=0.8 l=2.4
138      10.24     12.8    20321.7   0.01  # w=3.2 l=3.2
139      10.24       16    18419.9   0.01  # w=1.6 l=6.4
140       7.68     12.8   22051.22   0.01  # w=1.6 l=4.8
141      40.96     25.6   10160.85   0.01  # w=6.4 l=6.4
142      40.96       32   9209.799   0.01  # w=3.2 l=12.8
143      30.72     25.6   11025.61   0.01  # w=3.2 l=9.6
144     163.84     51.2   5080.426   0.01  # w=12.8 l=12.8
145     163.84       64   4604.902   0.01  # w=6.4 l=25.6
146     122.88     51.2   5512.804   0.01  # w=6.4 l=19.2
147     655.36    102.4   2540.212   0.01  # w=25.6 l=25.6
148     655.36      128   2302.451   0.01  # w=12.8 l=51.2
149     491.52    102.4   2756.403   0.01  # w=12.8 l=38.4
150    2621.44    204.8   1270.106   0.01  # w=51.2 l=51.2
151    2621.44      256   1151.225   0.01  # w=25.6 l=102.4
152    1966.08    204.8   1378.201   0.01  # w=25.6 l=76.8
153
154  coupsubres :
155  #  Generated by subresgen on 10:56:15 13-5-2003
156  #   area1     area2     dist          r    decr
157       0.64      0.64      1.6     648598  0.873512  # w=0.8 d=1.6
158       0.64      0.64      3.2    1101504  0.925946  # w=0.8 d=3.2
159       0.64      0.64      6.4    1996617  0.959256  # w=0.8 d=6.4
160       0.64      0.64     25.6    7341756  0.988935  # w=0.8 d=25.6
161       2.56      2.56      3.2   324299.1  0.873515  # w=1.6 d=3.2
162       2.56      2.56      6.4   550752.1  0.925953  # w=1.6 d=6.4
163       2.56      2.56     12.8   998307.9  0.959253  # w=1.6 d=12.8
164       2.56      2.56     51.2    3670877  0.988967  # w=1.6 d=51.2
165      10.24     10.24      6.4   162149.6  0.873515  # w=3.2 d=6.4
166      10.24     10.24     12.8     275376  0.925950  # w=3.2 d=12.8
167      10.24     10.24     25.6   499154.2  0.959259  # w=3.2 d=25.6
168      10.24     10.24    102.4    1835439  0.988967  # w=3.2 d=102.4
169     655.36    655.36     51.2   20268.69  0.873515  # w=25.6 d=51.2
170     655.36    655.36    102.4      34422  0.925953  # w=25.6 d=102.4
171     655.36    655.36    204.8   62394.28  0.959257  # w=25.6 d=204.8
172     655.36    655.36    819.2   229429.8  0.988985  # w=25.6 d=819.2
173
\end{Verbatim}
\normalsize
When we look to the file, we first see starting from line \io{18} the \io{unit} specifications.
The 2D capacitance units on line \io{20} and \io{21} are used for the capacitances definitions
starting from line \io{63}.
Note that the \io{colors} specification (not shown) is only used by visualization tool \io{Xspace}.

Starting on line \io{39} we find the \io{conductors} definitions.
This is an important section, because without this we can not specify
element pins for the capacitances and contacts.

Starting on line \io{48} we find the transistor (\io{fets}) definitions.
Here, we see, that the bulk region of the \io{nenh} is connected with the substrate.

Starting on line \io{53} we find the \io{contacts} definitions.
This is also an important section,
because here we find a contact definition between first metal \io{cmf} and the substrate \io{@sub}.
Note that the 2D capacitances or not connected to the substrate pin \io{@sub},
but to the ground node pin \io{@gnd}.

For 3D substrate resistance extraction, we need only one additional section.
Starting on line \io{125} we find the \io{sublayers} definitions.
These definitions specify for each substrate layer the conductivity (in $S / m$)
and the starting position (in $\mu m$).
Note that there is only one layer defined, which extends to minus infinity.

For 2D substrate resistance extraction (using the interpolation method), we need two additional sections.
Starting on line \io{129} we find the \io{selfsubres} table
and on line \io{154} we find the \io{coupsubres} table.
By the way, these tables are generated with tool \io{subresgen},
by performing a number of 3D substrate resistance extractions.
\newpage
\section{Parameter File}
\label{SOparam}
The parameter file \io{param.p} is controlling the extraction, see listing below.

\small \begin{Verbatim}[frame=single]
listing of file param.p
   3
   4    BEGIN sub3d            # Data for the boundary-element method
   5    be_mode          0g
   6    max_be_area      1     # micron^2
   7    edge_be_ratio    0.01
   8    edge_be_split    0.2
   9    be_window        10    # micron
  10    END sub3d
  11
  12    min_art_degree          3      # Data for network reduction
  13    min_degree              4
  14    min_res               100      # ohm
  15    max_par_res            20
  16    no_neg_res             on
  17    min_coup_cap            0.05
  18    lat_cap_window          6.0    # micron
  19    max_obtuse            110.0    # degrees
  20    equi_line_ratio         1.0
  21
  22    disp.save_prepass_image  on    # Data for Xspace
\end{Verbatim}
\normalsize
In the above listing, we see a \io{sub3d} block from line \io{4} to line \io{10}.
These parameters are only used for controlling the substrate 3D extraction.
The \io{max\_be\_area} (in $\mu m^2$) and \io{be\_window} (width in $\mu m$) parameter
needs always to be specified.
See for an explanation of the parameters also the "Space Substrate Extraction User's Manual".

On line \io{22}, you find a parameter for the visualization tool \io{Xspace}.
It means, that the image of the prepass may not be cleared
when starting drawing the image of the last pass.
See for an explanation also the "Xspace User's Manual".

\section{Running the Extractor}
The file \io{script.sh} is a batch file for running all the commands for this example.
First, it changes the current working directory '.' into a project directory:
\small
\begin{Verbatim}
% mkpr -p scmos_n -l 0.1 .
\end{Verbatim}
\normalsize
We use a lambda value (option \io{-l}) of $0.1 \mu m$.
For the CMOS technology, we use the \io{scmos\_n} process from the technology library.
We use the mask names as defined in the \io{maskdata} file of the library.
But, we are not using the default technology
file \io{space.def.s} and parameter file \io{space.def.p} of the library.
Note that the local technology file \io{tech.s} is a copy of the file \io{space.def.s}.
To use the local file, it needs to be compiled with \io{tecc} to the format
(a \io{.t} file) which is used by the extractor:
\small
\begin{Verbatim}
% tecc tech.s
\end{Verbatim}
\normalsize
Second, the layout description is put into the project database.
The layout is supplied in a GDS2 file, which can be converted to
internal database format with the \io{cgi} program:
\small
\begin{Verbatim}
% cgi oscil.gds
\end{Verbatim}
\normalsize
Now, we can extract a circuit description for the layout of the \io{oscil} cell, as follows:
\small
\begin{Verbatim}
% space3d -vF -bC -E tech.t -P param.p oscil
\end{Verbatim}
\normalsize
We use the verbose option (\io{-v}) to see what the extractor program is doing
and option \io{-F} to perform a flat extraction.
Second, we use option \io{-b} to perform fast interpolated substrate res. extraction,
and we use option \io{-C} to extract also the 2D couple caps.
The calculated substrate resistors are assigned between the substrate terminals and between the substrate
terminals and the substrate plane (node SUBSTR).
\\[1 ex]
The extracted circuit can be inspected using the \io{xspice} program.
We use option \io{-a} to get alpha-numeric node names,
and we use option \io{-u} to omit the \io{pbulk} and \io{nbulk} nodes.
\small
\begin{Verbatim}
% xspice -au oscil
\end{Verbatim}
\normalsize
The output of the SPICE circuit is listed below.

\small \begin{Verbatim}[frame=single]
  1   oscil
  2
  3   * Generated by: xspice 2.39 25-Jan-2006
  4   * Date: 23-Jun-06 9:09:24 GMT
  5   * Path: /users/simon/suboscil
  6   * Language: SPICE
  7
  8   * circuit oscil in out vss vdd sens
  9   m1 out in vss 6 nenh_0 w=4.4u l=800n
 10   m2 vdd in out vdd penh_0 w=5.2u l=800n
 11   m3 vdd 13 in vdd penh_0 w=5.2u l=800n
 12   m4 in 13 vss 5 nenh_0 w=4.4u l=800n
 13   m5 12 out vss 4 nenh_0 w=4.4u l=800n
 14   m6 vdd out 12 vdd penh_0 w=5.2u l=800n
 15   m7 vdd 11 13 vdd penh_0 w=5.2u l=800n
 16   m8 13 11 vss 3 nenh_0 w=4.4u l=800n
 17   m9 10 12 vss 2 nenh_0 w=4.4u l=800n
 18   m10 vdd 12 10 vdd penh_0 w=5.2u l=800n
 19   m11 vdd 9 11 vdd penh_0 w=5.2u l=800n
 20   m12 11 9 vss 1 nenh_0 w=4.4u l=800n
 21   m13 8 10 vss 14 nenh_0 w=4.4u l=800n
 22   m14 vdd 10 8 vdd penh_0 w=5.2u l=800n
 23   m15 vdd 7 9 vdd penh_0 w=5.2u l=800n
 24   m16 9 7 vss 16 nenh_0 w=4.4u l=800n
 25   m17 7 8 vss 15 nenh_0 w=4.4u l=800n
 26   m18 vdd 8 7 vdd penh_0 w=5.2u l=800n
 27   c1 vdd 8 12.80448f
 28   c2 vdd 7 12.80448f
 29   c3 vdd 10 12.80448f
 30   c4 vdd 9 12.80448f
 31   c5 vdd 12 12.80448f
 32   c6 vdd 11 12.80448f
 33   c7 vdd out 12.80448f
 34   c8 vdd 13 12.80448f
 35   c9 vdd in 12.80448f
 36   c10 vdd GND 152.1344f
 37   r1 1 16 513.9126k
 38   r2 1 14 2.95793meg
 39   r3 1 2 2.773095meg
 40   r4 1 vss 229.6561k
 41   r5 1 3 513.9126k
 42   r6 1 SUBSTR 39.66184k
 43   r7 2 14 513.9126k
 44   r8 2 vss 229.6561k
 45   r9 2 3 2.95793meg
 46   r10 2 4 513.9126k
 47   r11 2 SUBSTR 39.66184k
 48   r12 3 4 2.773095meg
 49   r13 3 vss 229.6561k
 50   r14 3 5 513.9126k
 51   r15 3 SUBSTR 39.66184k
 52   r16 4 vss 229.6561k
 53   r17 4 5 2.95793meg
 54   r18 4 6 513.9126k
 55   r19 4 SUBSTR 39.66184k
 56   r20 5 6 2.773095meg
 57   r21 5 vss 362.2433k
 58   r22 5 SUBSTR 35.28052k
 59   r23 6 vss 229.6561k
 60   r24 6 SUBSTR 36.68792k
 61   c11 7 GND 10.52699f
 62   c12 8 GND 6.05915f
 63   c13 9 GND 6.05915f
 64   c14 10 GND 6.05915f
 65   c15 11 GND 6.05915f
 66   c16 12 GND 6.05915f
 67   c17 13 GND 6.05915f
 68   c18 out GND 6.05915f
 69   c19 in GND 9.72507f
 70   r25 14 15 513.9126k
 71   r26 14 16 2.773095meg
 72   r27 14 vss 229.6561k
 73   r28 14 SUBSTR 39.66184k
 74   r29 sens 15 2.111757meg
 75   r30 sens vss 6.648565meg
 76   r31 sens 16 3.610183meg
 77   c20 sens GND 396.8e-18
 78   r32 sens SUBSTR 87.51464k
 79   r33 15 vss 387.7774k
 80   r34 15 16 2.95793meg
 81   r35 15 SUBSTR 35.20636k
 82   r36 vss 16 229.6561k
 83   c21 vss GND 46.05344f
 84   r37 vss SUBSTR 13.67904k
 85   r38 16 SUBSTR 37.50382k
 86   * end oscil
   ...
\end{Verbatim}
\normalsize
Below, you find a drawing of the listing of the extracted circuit
(the capacitors are not shown).
\newpage
\begin{figure}[h]
\centerline{\epsfig{figure=suboscil/circuit.eps, width=11cm}}
\end{figure}

Note that the red lines are substrate resistors.
The transistor bulks and substrate contact regions are substrate terminals.
There are not substrate resistors between all substrate terminals,
because of the interpolation method and the used Delaunay triangulation.
However, each substrate terminal has a substrate resistor to the substrate node \io{SUBSTR}.
When you use the visualization tool \io{Xspace},
you can draw an image with the substrate resistors (see below).
Type:
\small
\begin{Verbatim}
% Xspace -bC -E tech.t -P param.p oscil
\end{Verbatim}
\normalsize
Go to the "Display" menu and click on buttons "DrawSubTerm", "FillSubTerm" and "DrawSubResistor".
To start the extraction use hotkey 'e'.
To exit the visualization tool use hotkey 'q'.
\begin{figure}[h]
\centerline{\epsfig{figure=suboscil/xoscil1.eps, width=11cm}}
\end{figure}

\section{Running a Circuit Simulation}
If you have a \io{spice3} simulator available, you can perform a spice3
simulation to inspect the noise on the "sens" terminal node.
The noise is caused by the substrate coupling effects.


You can use the simulation GUI \io{simeye}, to start the analog simulation.
First, check the shell script \io{nspice} in \io{\$ICDPATH/bin} to see
if \io{spice3} is called correctly.
To start \io{simeye}, use the following command:
\small
\begin{Verbatim}
% simeye
\end{Verbatim}
\normalsize
Now, click on the "Simulate" menu and choice item "Prepare".
Select in the "Circuit:" field cell name "oscil" and
in the "Stimuli:" field file name "oscil.cmd".
Choice simulation "Type: spice" and click on button "Run".
You must get the following spice output result:

\begin{figure}[h]
\centerline{\epsfig{figure=suboscil/simeye.eps, width=11cm}}
\end{figure}

You can zoom-in on the "sens" waveform.
Choice the "ZoomIn" item out of the "View" menu.
Click with the mouse pointer in the output window, lower left by the "sens" waveform.
Draw a rubber box with the mouse around the waveform and click again.
To zoom-in even more, set "DetailZoomON" item in the "Options" menu.
To see the individual waveform points, choice "Values" in the "View" menu.
\\[1 ex]
To exit the program, choice item "Exit" in the "File" menu.
\newpage
\section{Running a 3D Substrate Extraction}
By a 3D extraction, the boundary element mesh (BEM) method is used to calculate the resistances.
For more details consult the "Substrate Extraction User's Manual".
To perform the 3D substrate resistance extraction of cell "oscil" in batch mode,
type the following command:
\small
\begin{Verbatim}
% space3d -v -BC -E tech.t -P param.p oscil
\end{Verbatim}
\normalsize
The \io{-B} option is now used in place of the \io{-b} option.
Note that you don't need to give the \io{-F} option,
because substrate extractions are always done in flat mode.
The output of the verbose mode is listed below.

\small \begin{Verbatim}[frame=single]
Version 5.3.1, compiled on Mon Feb 20 11:51:10 GMT 2006
See http://www.space.tudelft.nl
parameter file: param.p
Flat extraction mode turned on!
technology file: tech.t
preprocessing oscil (phase 1 - flattening layout)
preprocessing oscil (phase 2 - removing overlap)
prepassing oscil for substrate resistance prepare
strip 52 152 (subtract)
strip -48 152 (add)
strip 152 252 (subtract)
strip 52 252 (add)
strip 252 352 (subtract)
strip 152 352 (add)
strip 352 452 (subtract)
strip 252 452 (add)
strip 352 500 (add)
computing substrate effects for oscil
running: makesubres -Etech.t -Pparam.p -v oscil
makesubres: See http://www.space.tudelft.nl
makesubres: makesubres -Etech.t -Pparam.p -v oscil
makesubres: parameter file: param.p
makesubres: technology file: tech.t
makesubres: prepassing oscil for substrate resistance calculate
makesubres: strip 52 152 (subtract)
makesubres: Schur dimension 460, maxorder 247
makesubres: strip -48 152 (add)
makesubres: Schur dimension 848, maxorder 423
makesubres: strip 152 252 (subtract)
makesubres: Schur dimension 424, maxorder 211
makesubres: strip 52 252 (add)
makesubres: Schur dimension 884, maxorder 459
makesubres: strip 252 352 (subtract)
makesubres: Schur dimension 460, maxorder 247
makesubres: strip 152 352 (add)
makesubres: Schur dimension 884, maxorder 459
makesubres: strip 352 452 (subtract)
makesubres: Schur dimension 176, maxorder 175
makesubres: strip 252 452 (add)
makesubres: Schur dimension 636, maxorder 423
makesubres: strip 352 500 (add)
makesubres: Schur dimension 212, maxorder 175
makesubres:
makesubres: substrateStatistics for layout oscil:
makesubres: 	total num. of tiles : 72
makesubres: 	substrate tiles     : 19
makesubres: 	substrate terminals : 19
makesubres: overall resource utilization:
makesubres: 	memory allocation  : 1.21 Mbyte
makesubres: 	user time          :         7.1
makesubres: 	system time        :         0.1
makesubres: 	real time          :         7.3 100%
makesubres:
makesubres: makesubres: --- Finished ---
extracting oscil

extraction statistics for layout oscil:
	capacitances        : 21
	resistances         : 27
	nodes               : 23
	mos transistors     : 18
	bipolar vertical    : 0
	bipolar lateral     : 0
	substrate terminals : 19
	substrate nodes     : 11

overall resource utilization:
	memory allocation  : 0.56 Mbyte
	user time          :         0.0
	system time        :         0.0
	real time          :         9.0   0%

space3d: --- Finished ---
\end{Verbatim}
\normalsize
As you see above, the extraction process uses a number of passes.
First, there are two preprocessing phases.
In phase 1, the layout is expanded and flattened using tool \io{makeboxl}.
In phase 2, the expanded mask data is converted with tool \io{makegln} into a new set gln-files,
which geometric lines are sorted and the overlaps are removed.
Note that the preprocessing phases are only done, when the gln-files are out of date.
After this preprocess, prepass 1 is done.
This pass prepares the mask data for the \io{makesubres} program.
This program is started to compute the substrate effects (prepass 1b).
The last pass is always the extraction pass.
This pass reads the substrate results (stored in a "subres" file) and combines it with
the other extracted netlist elements.
\\[1 ex]
Use also the visualization tool \io{Xspace} and
perform an interactive 3D substrate extraction, type:
\small
\begin{Verbatim}
% Xspace -BC -E tech.t -P param.p oscil
\end{Verbatim}
\normalsize
We want to show the following items:
the boundary element mesh,
the substrate terminals
and the substrate resistors.
Go to the "Display" menu and click on buttons "DrawBEMesh", "DrawSubTerm" and "DrawSubResistor".
To start the extraction use hotkey 'e'.
You see 19 substrate terminals.
But not all terminals are connected with each other by substrate resistors
(because of the boundary element window).
And each substrate terminal has a very fine boundary element mesh.
See the picture below.
\newpage
\begin{figure}[h]
\centerline{\epsfig{figure=suboscil/xoscil2.eps, width=11cm}}
\end{figure}

\noindent
Use hotkey 'i' to zoom-in on the cursor position, to show the mesh in more detail.
Use the arrow keys to pan around.
Use the 'o' key to zoom-out and 'b' key to set the bounding-box view again.
Use hotkey 'q', when you want to quit the visualization tool.
The following picture shows a zoom-in example:
\begin{figure}[h]
\centerline{\epsfig{figure=suboscil/xoscil3.eps, width=11cm}}
\end{figure}
\newpage
The 3D substrate method uses a boundary element window (parameter \io{sub3d.be\_window}).
In the x-direction, the layout is split into strips, which each have a width of the \io{be\_window}.
A good choice of the window size is important, because the strips give extra tile splitting.
When you look to the verbose output of the extractor, you see information about the strip positions.
Because the window size is $10 \mu$ and $\lambda = 0.1 \mu$, the window size is $100 \lambda$ units.
The bounding box of the design is from $xl = -48$ to $xr = 500$ units.
Between each two strips are results calculated.
The double calculated results of intermediate single strips are subtracted.
You can use the visualization tool to see the strips by looking to the tile splitting.
Start the tool with the following two display options added to the command line:
\small
\begin{Verbatim}
% Xspace -BC -E tech.t -P param.p -Sdisp.pair_to_infinity \
                                  -Sdisp.coord_in_dbunits oscil
\end{Verbatim}
\normalsize
Before starting the extraction,
go to the "Display" menu and click on buttons "PairBoundary" and "DrawTile".
To stop displaying the image after the first prepass,
go to the "Extract" menu and click on button "pause afterpp".
Now, hit the extract key 'e' to start the extraction.
The result is shown below:
\begin{figure}[h]
\centerline{\epsfig{figure=suboscil/xoscil4.eps, width=11cm}}
\end{figure}

Note that not all strip edges can be shown, because no tile edges are generated.
You see on the design top side some tile edges to infinity, which are from the strips.

To identify the strips,
i have added a number of coordinate crosses to the image.
To add coordinates in database units, you must set parameter \io{disp.coord\_in\_dbunits}.
Because, default \io{Xspace} is using an internal unit, whereby all units are multiplied by 4.
I shall explain, how you can easy enter coordinate crosses.
Use hotkey 'g' and type the x-coordinate number and type a minus sign to negate it.
Type 'Enter' to finish the input.
Note that we have not entered the y-coordinate.
If you want to enter (or change) an y-coordinate, you must begin the number with hotkey ','.
For more information, see also the "Xspace User's Manual".
